\chapter{法文文档}

\begin{quote}
    那人说,你所赐给我,与我同居的女人,她把那树上的果子给我,我就吃了。——《圣经·创世纪》3:12
\end{quote}

了解法文文档要遵循的规则总是好的。严格说来,这些规则并不是无法逃避的准则,而更像是一些使用上的规矩。为了让文档容易可读、避免读者被打断,\emph{建议遵循}这些规则。这些使用上的建议总体上可以让文档看起来更严肃,甚至更专业。有很多关于法文排版的作品,这里给出来自国家印刷局(imprimerie nationale)的汇编材料[7],以及伊夫·佩伊卢梭(Yves Peyrousseaux)的手册[13]。

本章包含一些总结了有关\LaTeX 为实现法文变音符号而使用的字体编码方法的信息,介绍了关于排版的一些规则和用于简化法文输入过程的包\textsf{babel}。本章的末尾介绍了文档类型\dm{letter},其是为信件和传真而设计的。

\section{带有变音符号的字母的问题}

在若干年前,\TeX 的构思阶段完成的时候,其使用的字体不包含带有变音符号的字符。每个字形以7个二进制位区分,这样一共有128个字符可被编码。由于这种方法起源于美国,这128个字符中显然不包括法文中使用的带有变音符号的字符。正因如此,在很长一段时间内,那些优秀的讲法语母语的\TeX 和\LaTeX 用户不得不以一种窘迫方式去录入带有难以输入的字符的法文文档(document en \verb+fran\c{c}ais+ avec des \verb+caract{\`e}res+ assez \verb+p{\'e}nibles+ \verb+{\`a}+ taper)\yz{
    即document en français avec des caractères assez pénibles à taper。原书此处以源代码的方式展示部分单词,以展现其烦琐程度。%原书依然错误渲染了上引号和下引号
}。

今天,这些不悦不再成为人们的糟糕记忆。1990年起,一种容纳了多种语言中带变音符号的字符的字体编码被采用,称为\emph{Cork encoding}或\emph{T1编码(codage T1)}。当然,这种\TeX 编码本身和目前的字符编码标准间存在一定的联系。一些\LaTeX 包就包含了从字符编码(如iso-latin1)到字体编码(如T1编码)的“翻译”操作。

\begin{exclamation}
    于20实际90年代末出现的标准是ISO8859,带有针对称为\emph{latin1}的欧洲语言的编码方案扩展。这也是今天最常用的编码方案。然而,近年来,
\end{exclamation}