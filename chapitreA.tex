\part{附录}
\appendix
\chapter{生成“PDF”}

本附录介绍一种生成PDF(portable document format)文档的新方法。这种格式由Adobe公司创造,优势是使一台计算机向另一台计算机传输文件的过程具有十足的便携性,以及能使不同操作系统间传输文件的通用性。因此,在今天能够才\LaTeX 源代码生成这种文件是十分有趣的。

\section{通用原则}

从\LaTeX 文档生成PDF文件至少有三种方法:

\begin{enumerate}
    \item 借助\textsf{pdflatex}代替\textsf{latex}程序将\LaTeX 源代码翻译为PDF文件;
    \item 借助\textsf{dvipdf}将DVI文件翻译为PDF文件;
    \item 借助\textsf{ps2pdf}将PostScript格式的输出翻译为PDF文件。
\end{enumerate}

\begin{exclamation}
拥有关于上述第一条的一些经验的鄙人将专注于pdflatex。想要正确使用这个软件,有一些前提条件:

\begin{itemize}
    \item 要么使用了包lmodern;
    \item 要么安装了弗拉基米尔·沃洛维奇(Vladimir Volovich)的扩展“CM-Super font”。Debian的发行版Etch包含了开箱即用的包。我们同样可以在用于为Debian的一个发行版——Sarge借助te\TeX 安装该扩展的网站(\wz{http://sravier.free.fr/linux/\linebreak debian\_latex\_cm-super.html})上找到相关文档。
\end{itemize}
\end{exclamation}

\section{更改之处}

为了编译\LaTeX 源文件、生成PDF格式的文件,我们可以以如下方式使用软件\textsf{pdflatex}:

\dmh{pdflatex monfichier.tex}

如果源文档之中没有错误,该指令会创建名为\dm{monfichier.pdf}的文件。如下是几个需要注意的重点。

\begin{description}
    \item[图形] 对于图片,应当以PNG或JPEG格式包含在文档中;对于绘制的图画,应当以PDF格式包含\jz{
        软件\textsf{Xfig}的软件可以被转换为PDF。%TODO 啥?
    }。
    \item[链接] 在包含了包\textsf{hyperref}的情况下,PDF文档会在指令\verb|\ref|出现时、目录中、索引中等情况下自动包含链接。此外,适用于软件\textsf{Acrobat Reader}的可折叠目录也会生成。
\end{description}

\section{一些技巧}

考虑到我们经常使用同一个源文件生成DVI或PDF,且根据具体情况应当包含不同的图像文件,可以借助包\textsf{ifpdf}来实现这样的技巧:

\begin{dmd}
\begin{verbatim}
\ifpdf
% 针对输出PDF而特定的内容
\else
% 针对输出DVI而特定的内容
\fi  
\end{verbatim}
\end{dmd}

\subsection{处理图像}

我们可以编写如下的内容:

\begin{dmd}
\begin{verbatim}
\ifpdf
\graphicspath{{pngs/}{{pdfs/}}
\else
\graphicspath{{epss}}
\fi
\end{verbatim}
\end{dmd}

这对应于我们将图像文件按\dm{pngs}、\dm{pdfs}、\dm{epss}等目录整理的情况。这个新的“if”语句也可以写成如下结构:

\begin{dmd}
\begin{verbatim}
\ifpdf
\includegraphics[pdftex]{graphicx}
\else
\includegraphics{graphicx}
\fi
\end{verbatim}
\end{dmd}

在最新的\LaTeX 版本中,这样写不是必需的。

\subsection{缩略图}

\textsf{pdflatex}近期的版本支持了为包括但不仅限于\textsf{evince}、\textsf{Acrobat Reader}等浏览器创建缩略图(vignettes;英:\emph{thumbnail})。以前,则需要使用包\textsf{thumbpdf}:

\begin{dmd}
\verb+\usepackage{thumbpdf}+
\end{dmd}

并且执行:

\dmh{thumbpdf monfichier.pdf}

该指令会创建名为\dm{monfichier.tpt}的文件,会在接下来使用\textsf{pdflatex}编译时被包含。

\subsection{页码}

为了在浏览器\textsf{Acrobat Reader}中展示页码,需要在包含包\textsf{hyperref}(参见\S A.4)添加选项\dm{pdfpagelabels}。

\subsection{书签}

PDF浏览器中的书签(signet;英:\emph{bookmarks})是一种“目录浏览器”,可以直接跳转到某一指定级别的某一节。在生成目录上,需要绕过两个难点。

\begin{enumerate}
    \item 将“收尾”的内容(参考文献、术语字典、索引)设置为该浏览器中各部分的同一级别。默认情况下,这些信息会“掩盖”在附录部分中,因为它们与各\verb|\chapter|处于同一深度。
    \item 确保书签中指向索引的链接确实指向索引……
\end{enumerate}

对于第一个问题,你只需要使\LaTeX 相信“收尾”内容中的\emph{章}与目录、各\emph{部分}处于同一深度。这里的咒语是:

\begin{dmd}
\verb|\renewcommand{\toclevel@chapter}{-1}|
\end{dmd}

它需要放置在风格文件中的适当位置。我们重定义\verb|\backmatter|\celan{\S 10.4.4}的地方无疑是很好的选择。

\begin{qquestion}
作为\emph{书签}部分的结尾,为了让指向索引的书签实际指向索引(!),我们这次要念的是萨满教的咒语:

\begin{dmd}
\begin{verbatim}
\let\printindexORIG\printindex 
\renewcommand{\printindex}{%
    \cleardoublepage
    \phantomsection% 创建一个假节
    \addcontentsline{toc}{chapter}{Index} 
    \printindexORIG}
\end{verbatim}
\end{dmd}

该咒语通过包hyperref提供的指令\verb|\phantomsection|为指令\verb|\printindex|添加了一个并不存在的节,使其过载。别问我更多了\dm{:-)}
\end{qquestion}

\section{超链接}

