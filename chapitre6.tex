\chapter{科技文档}

\begin{quote}
    智慧人积存知识,愚妄人的口速致败坏。——《圣经·箴言》10:14
\end{quote}

现在,终于到了跟你聊聊所谓\emph{科技}文档特性的时刻。虽然关于数学式和其他方程的问题已经在第3章妥善解决,但还有一块骨头要啃:参考文献。对于这个问题,虽然不能一口吃个胖子,但接下来的内容可以让你大幅简化工作。在本章,我们还会解释生成索引的机制。

本章首先会介绍起草文章的几点特别之处,然后展示参考文献的生成和索引的生成,最后介绍将大篇幅文档拆解成几个小部分的实用方法。

\section{文章(article)}

为了起草一篇文章,没有什么新内容可以介绍的,我们目前为止见过的所有内容都适用。只需要注意,在文前部分中,可以使用以下指令:

\begin{itemize}
    \item \verb|\title|,定义标题;
    \item \verb|\date|,定义日期;
    \item \verb|\author|,定义作者团队;
    \item \verb|\thanks|,定义作者单位。
\end{itemize}

若要利用这些定义来插入标题,需要在\verb|\begin{document}|之\textbf{后}插入指令\verb|\maketitle|:

\begin{dmd}
\begin{verbatim}
\documentclass{article}
\title{Le seuillage à 128 : une révolution !}
\author{M. C. Orlanrien\\
        Institut du Pixel\\
        42007 Saint-Etienne---FRANCE}
\date{2 Avril 1927}
\begin{document}
\end{verbatim}
\backslash maketitle\textsl{\% 标题插到此处}\\
...\\
\verb|\end{document}|
\end{dmd}

此处重复一遍\jz{
    因为传授知识就是重复的过程。
}:标题是由指令\verb|\maketitle|生成并插入的,而不是文前部分的定义。

通常来说,会议或期刊提供的模板文件中会引入一些变化(例如使用\verb|\address|分隔作者和其地址),但基本原理是一致的。

\section{参考文献}

由两种方式可以使用\LaTeX 起草文章的参考文献部分。其中,可以称得上是“手动”的方式是,在文章中插入环境\dm{thebibliography}。另一种方式,即此处要介绍的方式,是使用\bib ,主要分为如下步骤。

\begin{enumerate}
    \item 创建一个或多个参数文件,包含\bib 格式的各条参考文献入口(entrée;文章、会议……)。这个步骤不可避免地需要我们去\emph{输入}。
    \item 在文档中,使用指令\verb|\cite|去引用这些入口。
    \item 参考文献会自动根据你选择的特殊风格排版。
\end{enumerate}

这种方法的优势是,对于每条参考文献,你只需输入一次。此外,考虑到可以使用\emph{风格文件},你不用去担心它的版式。有几十种风格文件,对应各种标准,包含期刊和其他会议所使用的标准。我们也可以在互联网上找到\bib 格式的参考文献数据库,可以在文档中直接使用。

我们重复一遍:参考文献有多种标准。但不幸的是,一些期刊偏偏喜欢指定属于自己的参考文献格式。有朝一日你在这种期刊上发表文章时,就需要去创建或调整风格文件。为了实现这一点,可以去查找工具\textsf{makebst}。

\subsection{\dm{.bib}文件}

第一个操作是构建\jz{
    \textbf{Emacs}的Auc\TeX 组件包含了很好用的\bib 模式。
}参考文献文件,其扩展名最好为\dm{.bib}。该文件需要遵循特殊的语法。首先需要知道,\bib 通过\emph{类型(type)}区分每个入口。这样一来,每个入口都带有一个文档类型:图书、文章、会议、科技报告……一共有二十多种不用的文档类型。

\begin{ii}
正常来说,我们可以在伴随\LaTeX 发行版提供的文件找到名为\bib ing的文件(命名为\dm{btxdoc.pdf}),由奥兰·帕塔什尼克(Oran Patashnik)在约二十年前创作。该文件包含有关构建\bib 格式文件方法的重要信息来源。
\end{ii}

每个入口\emph{类型}都包含一定数量描述该入口的\emph{字段(champ)}。参考文件入口的结构如下:

\begin{dmd}
@\codereplace{入口}\{\codereplace{关键描述},\\
\verb|  |\codereplace{字段$_1$}\ =\ \{...\},\\
\verb|  |\codereplace{字段$_2$}\ =\ \{...\},\\
\verb|  |...\\
\verb|  |\codereplace{字段$_n$}\ =\ \{...\},\\
\}
\end{dmd}

其中,\codereplace{入口}表示文档类型(\dm{article}、\dm{inproceedings}等),\codereplace{字段$_1$}、\codereplace{字段$_2$}……\codereplace{字段$_n$}表示参考文献入口的不同字段。这些\bib 的保留字可以以大写或小写形式输入。

符号\codereplace{关键描述}需要以唯一方法描述该文档,以备通过用于识别标签的符号\verb|\label|来重新找到。为了你能够快速上手\bib ,接下来的示例综合了三个你需要使用的基本入口。

\subsubsection{期刊文章}

有一篇期刊文章需要以如下形式输入:

\begin{dmd}
\begin{verbatim}
@article{qtz:UchArb,
    author ={Uchiyama, Toshio and Arbib, Michael A.},
    title = {Color Image Segmentation
            Using Competitive Learning},
    journal=pami,
    volume =16, number=2, pages={1197--1206},
    month=dec, year=1994}
\end{verbatim}
\end{dmd}

有以下几点需要注意。

\begin{enumerate}
    \item 字段\dm{author}、\dm{title}、\dm{journal}、\dm{year}是必需的。
    \item 对于作者\jz{
        此处关于作者的注意事项对于其他入口(会议、书等)也同样适用。
    }姓名,需要遵循\codereplace{姓}、\codereplace{名}的顺序。\textbf{所有}作者姓名都需要以\dm{and}分隔。
    \item 对于复合姓或其他特殊作者名,可以以如下形式输入:
    \begin{center}
        \dm{author="de la Motte Beuvron, Alain"}
    \end{center}
    其中遵循的顺序为:\codereplace{特殊组成部分}、\codereplace{姓}、\codereplace{名}。逗号作为分割符,起到与上例中相似的作用。
    \item 所有月份可以以字符串的形式给出,如\dm{jan}、\dm{feb}、\dm{mar}等。
\end{enumerate}


在\dm{.bib}文件的开头,为简洁起见,我们已经创建了\emph{缩写}\dm{pami}:

\begin{dmd}
\begin{verbatim}
@string{pami="IEEE transactions on Pattern Analysis and
              Machine Intelligence"}
\end{verbatim}
\end{dmd}

\subsubsection{会议析出文章}

没错,\bib 会区分对待\emph{期刊}和\emph{会议}中的文章。格式结构与上例很相似,只不过\dm{booktitle}用于会议标题而不是期刊标题:

\begin{dmd}
\begin{verbatim}
@Inproceedings{qtz:BouOrch,
    author={Bouman, Charles A. and Orchard, Michael T.}
    title={Color Image Display with a Limited Palette Size},
    booktitle={SPIE Conference on Visual Communications
               and Image Processing},
    volume=1199,pages={522--533},
    year=1989}
\end{verbatim}
\end{dmd}

这里,\dm{author}、\dm{title}、\dm{booktitle}、\dm{year}是必填字段,我们可以选用\dm{volume}和\dm{number}。

\subsubsection{图书片段}

相比于整本书,我们经常指参考其中的一个片段——若干章、若干页:

\begin{dmd}
\begin{verbatim}
@inBook{col:McA,
    author =    {MacAdam, David L.},
    title =     {Color Measurement},
    chapter =   4,
    pages   =   {48--49},
    publisher = {Springer-Verlag},
    year =      1985}
\end{verbatim}
\end{dmd}

强制字段为:\dm{author}、\dm{title}、\dm{chapter}或\dm{pages}、\dm{publisher} (出版方),以及\dm{year}。

\begin{ii}
我们再次强烈建议你使用Emacs组件Auc\TeX 的\textsc{Bib}\TeX 模式。特别是该模式为你提供了包含所有入口类型的菜单。选择菜单中的一项,就可以在你的文件中插入入口“骨架”。该组件可以在\wz{ftp.lip6.fr/pub/TeX/CTAN/support/auctex}下载,也可以以包Debian的形式获取。
\end{ii}

\subsection{参考文献的标注}

一旦参考文献建立完成,就可以即刻在文档中借助关键描述使用指令\verb|\cite|来标明引用:

\begin{dmd}
\backslash cite\{\codereplace{关键描述}\}
\end{dmd}

