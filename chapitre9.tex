\chapter{必要工具}

\begin{quote}
    我所爱的,你何其美好。何其可悦,使人欢畅喜乐。你的身量好像棕树。你的两乳如同其上的果子,累累下垂。

    \hfill《圣经·雅歌》7:7
\end{quote}

在本章,我们会介绍创造比第4章中介绍的指令和环境更复杂的指令和环境所需准备的工具。此外,借着本章的介绍,我们旨在说明,此处提到的第4章需要正确消化,才能继续这一部分的阅读。本章也会介绍一些关于字体的机制,以及挖掘\LaTeX 资源的方法。

\section{赫尔克里·波洛}

\section{在文件中挖掘信息}

首先,为了让使用\LaTeX 写成的文档带有一些个人特色,需要知道组成你使用的\TeX 或\LaTeX 的发行版的文件的组织方法。鄙人使用了UNIX平台的发行版\TeX Live(\wz{http://www.tug.org/texlive})。在这个发行版中,我们可以在第一时间在以下目录中查阅各种包的文档:

\begin{dmd}
/usr/share/texmf-texlive/doc/latex/
\end{dmd}

这个目录中包含其他子目录,通常每个子目录对应一个包,其中就以DVI或PostScript文件的形式提供了文档。在一些情况下,需要去检查这些包的源代码。在发行版te\TeX 中,这些源代码位于:

\begin{dmd}
/usr/share/texmf-texlive/tex/latex
\end{dmd}

在同样的位置,我们通常可以为每个包找到一个目录,包含文本形式且带有扩展名\dm{sty}的源代码,在必要时也会包含相关文件。最后,为了独立于我们可以包含的包而了解\LaTeX 的默认行为,可以借助以下位置的\LaTeX 源代码:

\begin{dmd}
/usr/share/texmf-texlive/tex/latex/base/latex.ltx
\end{dmd}

对于文档类型\dm{book},还可以借助以下位置的文档类型源代码:

\begin{dmd}
/usr/share/texmf-texlive/tex/latex/base/book.cls
\end{dmd}

\subsection{检查宏}

查找指令定义的一个非常便捷的方法是在交互式会话中求助于\LaTeX 。可以直接在操作系统的命令行终端中执行以下指令:

\dmh{latex}

我的系统是这样冷冰冰地回答的:

\begin{dmd}
\begin{verbatim}
This is e-TeXk, Version 3.14159-2.1 (Web2C 7.4.5)
%&-line parsing enabled.
**
\end{verbatim}
\end{dmd}

在“赤裸裸的\TeX ”呼喊出这个冷峻的提示(\dm{**})的邀请下,我勇敢地回复了\verb|&latex|来要求加载\LaTeX 格式。没有丝毫延迟,就得到了答复:

\begin{dmd}
\begin{verbatim}
**&latex
entering extended mode
LaTeX2e <2001/06/01>
Babel <v3.7h> and hyphenation patterns for american,
french loaded.
*
\end{verbatim}
\end{dmd}

注意,提示中少了一个星号。从现在开始,我们就可以交互式地编写\LaTeX 文档了。从绝对意义商来说,这样做的乐趣不大,但从获取指令的定义和语法上来说,却很有帮助。因此,举例来说,我们可以写下这样的指令:

\begin{dmd}
\verb|*\show|\codereplace{指令}
\end{dmd}

这样可以获取\codereplace{指令}的定义。例如:

\begin{dmd}
\begin{verbatim}
*\show\mbox
> \mbox=\long macro:
\end{verbatim}
\verb|#1->\leavevmode \hbox {#1}.| \quad $\leftarrow$\textsf{此处为定义}\\
\verb+<*> \show\mbox+
\end{dmd}

其中向我们提供了指令\verb|\mbox|的定义。可以注意到,该指令被调用时,将这种调用转化为对\verb|\leavevmode|和\verb|\hbox|的调用。在好奇心的驱使下,我们继续查看指令的定义:

\begin{dmd}
\verb|*\show\hbox|\\
\verb|> \hbox=\hbox.| \quad $\leftarrow$\textsf{这是一个原语}\\
\verb|<*> \show\hbox|
\end{dmd}

可以观察到,\verb|\hbox|不是由其他指令定义而成的。在\TeX 中,这种指令称为原语(primitive)。我们的探索可以继续:

\begin{dmd}
\verb|*\show\leavevmode|\\
\verb|> \leavevmode=macro:|\\
\verb|->\unhbox \voidb@x .|\quad $\leftarrow$\backslash leavevmode\textsf{的定义}\\
\verb|<*> \show\leavevmode|
\end{dmd}

以此类推……

\section{底层工具}

\subsection{百分号图个什么?}

你可能已经注意到,有时\LaTeX 源代码中的行末带有百分号\dm{\%}。基于代码换行时文本间会自动添加空格这样的情况,百分号就有理由出现了。请看如下指令:

\begin{dmd}
\verb|\newcommand{\beurk}{bidule}|
\end{dmd}

为了增强可读性,这条指令可以拆分为多行代码:

\begin{codelist}[9.1]{
\newcommand{\beurk}{
    bidule
}
==(\beurk)==
}\begin{verbatim}
\newcommand{\beurk}{
    bidule
}
==(\beurk)==
\end{verbatim}
\end{codelist}

可以观察到,“bidule”一词的两侧出现了我们不想要的空格。为了避免这种现象,可以使用如下方式改写:


\begin{codelist}[9.2]{
\newcommand{\beurk}{%
    bidule%
}
==(\beurk)==
}\begin{verbatim}
\newcommand{\beurk}{%
    bidule%
}
==(\beurk)==
\end{verbatim}
\end{codelist}

在另一些场景下,空格会为行文带来有害的干预。定义以下环境:

\begin{dmd}
\begin{verbatim}
\newenvironment{hyperimportant}{% 
    \bfseries\itshape}{% 
    \upshape\mdseries}
\end{verbatim}
\end{dmd}

\newenvironment{hyperimportant}{% 
    \bfseries\itshape}{% 
    \upshape\mdseries}

\begin{codelist}[9.3]{
    Il est impératif
\begin{hyperimportant}
  de multiplier les sauvegardes
\end{hyperimportant}
de vos documents personnels
}\begin{verbatim}
Il est impératif
\begin{hyperimportant}
    de multiplier les sauvegardes
\end{hyperimportant}
de vos documents personnels
\end{verbatim}
\end{codelist}

如果仔细观察生成的文本,可以注意到,在粗斜体部分文本“{\bfseries\itshape de ... sauvegardes}”的两侧各有两个空格:

\begin{itemize}
    \item “{\bfseries\itshape de}”前面的两个空格分别由“\dm{impératif}”和begin条目“\verb|\begin{hyperimportant}|”后的换行引入;
    \item “{\bfseries\itshape sauvegardes}”后面的两个空格分别由“\dm{sauvegardes}”和end条目“\verb|\end{hyperimportant}|”后的换行引入。
\end{itemize}

我们可以删除换行,来证明这种观点:

\begin{codelist}[9.4]{
Il est impératif\begin{hyperimportant} de
multiplier  les
sauvegardes\end{hyperimportant} de vos
documents personnels
}\begin{verbatim}
Il est impératif\begin{hyperimportant} de
multiplier  les
sauvegardes\end{hyperimportant} de vos
documents personnels
\end{verbatim}
\end{codelist}

为了防止被这种问题牵扯精力,一般可以求助于两条用于删除双重空格的指令。对于序列之前的双重空格,可以调用指令\verb+\ignorespaces+来消除它;对于序列之后的,可以调用\verb|\unskip|。

\subsubsection{指令\dm{\backslash ignorespaces}}

该指令可以展开后续的指令,并忽略后面的所有空格:

\begin{codelist}[9.5]{
\newcommand{\truc}{ }
\newcommand{\bidule}{ }

a\truc\bidule b\par
a\ignorespaces\truc\bidule b
}\begin{verbatim}
\newcommand{\truc}{ }
\newcommand{\bidule}{ }

a\truc\bidule b\par
a\ignorespaces\truc\bidule b
\end{verbatim}
\end{codelist}

以上示例中,指令\verb|\truc|和\verb|\bidule|的唯一作用都是在被调用时生成空格。例如,以下指令会生成“\verb*|a { }b|”:

\begin{dmd}
\verb|a\truc\bidule b|
\end{dmd}

也就是说,字母a和b之间由两个空格隔开。调用指令\verb|\ignorespaces|——正如其名——可以忽略指令\verb|\truc|和\verb|\bidule|产生的空格。因此,对于前面的示例,可以使用以下指令:

\begin{dmd}
\begin{verbatim}
\newenvironment{hyperimportant}{% 
    \bfseries\itshape\ignorespaces}{\upshape\mdseries}
\end{verbatim}
\end{dmd}

这样就能删除一个空格:

\renewenvironment{hyperimportant}{% 
    \bfseries\itshape\ignorespaces}{\upshape\mdseries}

\begin{codelist}[9.6]{
    Il est impératif
\begin{hyperimportant}
    de multiplier les sauvegardes
\end{hyperimportant}
de vos documents personnels.
}\begin{verbatim}
Il est impératif
\begin{hyperimportant}
    de multiplier les sauvegardes
\end{hyperimportant}
de vos documents personnels.
\end{verbatim}
\end{codelist}

\subsubsection{指令\dm{\backslash unskip}}

如果细心一些,我们可以发现,“{\bfseries\itshape sauvegardes}”和“de”之间仍然有两个空格抵住了我们的攻击。这就到了\TeX 的原语\verb|\unskip|的用武之地:它可以删除后一个被插入的空格:

\begin{codelist}[9.7]{
    \newcommand{\truc}{ }
\newcommand{\bidule}{ }
a\truc\bidule b\par
a\truc\bidule\unskip b
}\begin{verbatim}
\newcommand{\truc}{ }
\newcommand{\bidule}{ }
a\truc\bidule b\par
a\truc\bidule\unskip b
\end{verbatim}
\end{codelist}

最后,我们环境的“正确”定义如下:

\begin{dmd}
\begin{verbatim}
\newenvironment{hyperimportant}{% 
    \bfseries\itshape\ignorespaces}{\unskip\upshape\mdseries}
\end{verbatim}
\end{dmd}

这样,就可以删除所有我们不希望插入的空格:

\renewenvironment{hyperimportant}{% 
    \bfseries\itshape\ignorespaces}{\unskip\upshape\mdseries}

\begin{codelist}[9.8]{
Il est impératif
\begin{hyperimportant}
  de multiplier les sauvegardes
\end{hyperimportant}
de vos documents personnels.
}\begin{verbatim}
Il est impératif
\begin{hyperimportant}
  de multiplier les sauvegardes
\end{hyperimportant}
de vos documents personnels.
\end{verbatim}
\end{codelist}

\subsection{字符\dm{@}}

在开始探索包的源代码时,你会发现,有很大一部分的指令名称定义中都带有字符\dm{@}。然而,在\dm{.tex}文档中,不允许执行名称带有该字符的指令。这样可以保护或限制包指令的能力范围。例如,在包\textsf{changebar}中定义了指令\verb+\cb@defpoint+,它不能被包的使用者调用。若要重定义该内部指令,需要做出以下小操作:

\begin{dmd}
\begin{verbatim}
\makeatletter
% 我们可以在这里胡说八道
\renewcommand{\@ttention}{oulala...}
\makeatother
% 但在这里就不行了
\end{verbatim}
\end{dmd}

这里举例的指令\verb+\@ttention+只有在字符\dm{@}被当作字母的情况下才能被操作。这正是\verb+\makeatletter+的作用:将字符\dm{@}转化为字母,就像其他字母一样。而指令\verb+\makeatother+可以重新赋予该字符区别于其他字母的特殊性。

\begin{exclamation}
这种操作在使用指令\verb+\usepackage+包含的风格文件中不是必需的。对于这些文件,字符\dm{@}可以当作字母使用。
\end{exclamation}

\TeX 得以更改此字符的类别的方法在10.5.1小节会详细解释。

\subsection{\TeX 的\dm{\backslash let}}

有时,修改\LaTeX 内部的指令以在其默认行为中添加功能是很有用的做法。例如,为了修改内部指令\verb+\bidule+\jz{
    好吧,这并不是一条内部指令。它只是作为愚蠢的例子而使用的指令名称……
},可以遵循以下步骤。

\begin{enumerate}
    \item 借助\TeX 的指示\verb+\let+保存该指令:
    
    \begin{dmd}
    \verb|\let\biduleORIG\bidule|
    \end{dmd}

    \item 在初始定义的基础上重新定义指令\verb+\bidule+:
    
    \begin{dmd}
    \begin{verbatim}
\renewcommand{\bidule}{%
    一些新东西\biduleORIG}
    \end{verbatim}
    \end{dmd}

    \item 如果有需要,可以借助如下指令重新回到其原定义:
    
    \begin{dmd}
    \verb+\let\bidule\biduleORIG+
    \end{dmd}
\end{enumerate}

\section{控制结构和测试}

包\textsf{ifthen}引入的结构遵循以下语法:

\begin{dmd}
\verb+\ifthenelse{+\codereplace{布尔表达式}\}\\
\{ ……若真,LaTeX代码……\}\\
\{ ……若假,LaTeX代码……\}
\end{dmd}

以及

\begin{dmd}
\verb+\whiledo{+\codereplace{布尔表达式}\}\\
\{……当真时,LaTeX代码……\}
\end{dmd}

