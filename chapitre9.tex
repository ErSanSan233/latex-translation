\chapter{必要工具}

\begin{quote}
    我所爱的,你何其美好。何其可悦,使人欢畅喜乐。你的身量好像棕树。你的两乳如同其上的果子,累累下垂。

    \hfill《圣经·雅歌》7:7
\end{quote}

在本章,我们会介绍创造比第4章中介绍的指令和环境更复杂的指令和环境所需准备的工具。此外,借着本章的介绍,我们旨在说明,此处提到的第4章需要正确消化,才能继续这一部分的阅读。本章也会介绍一些关于字体的机制,以及挖掘\LaTeX 资源的方法。

\section{赫尔克里·波洛}

\section{在文件中挖掘信息}

首先,为了让使用\LaTeX 写成的文档带有一些个人特色,需要知道组成你使用的\TeX 或\LaTeX 的发行版的文件的组织方法。鄙人使用了UNIX平台的发行版\TeX Live(\wz{http://www.tug.org/texlive})。在这个发行版中,我们可以在第一时间在以下目录中查阅各种包的文档:

\begin{dmd}
/usr/share/texmf-texlive/doc/latex/
\end{dmd}

这个目录中包含其他子目录,通常每个子目录对应一个包,其中就以DVI或PostScript文件的形式提供了文档。在一些情况下,需要去检查这些包的源代码。在发行版te\TeX 中,这些源代码位于:

\begin{dmd}
/usr/share/texmf-texlive/tex/latex
\end{dmd}

在同样的位置,我们通常可以为每个包找到一个目录,包含文本形式且带有扩展名\dm{sty}的源代码,在必要时也会包含相关文件。最后,为了独立于我们可以包含的包而了解\LaTeX 的默认行为,可以借助以下位置的\LaTeX 源代码:

\begin{dmd}
/usr/share/texmf-texlive/tex/latex/base/latex.ltx
\end{dmd}

对于文档类型\dm{book},还可以借助以下位置的文档类型源代码:

\begin{dmd}
/usr/share/texmf-texlive/tex/latex/base/book.cls
\end{dmd}