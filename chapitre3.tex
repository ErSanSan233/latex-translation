\chapter{数学排版}

这十二使徒的名:头一个叫西门,又称彼得……——《圣经·马太福音》10:2

毫无疑问,\LaTeX 最实用和有趣\jz{
    没错,没错!真的有人排公式纯是为了玩!
}的特性就是可以生成数学公式。它生成的公式自然、美观,并且不需要你做任何工作\jz{
    或者只需要你去做两三件小事情。
}。另外,如果你有使用关于某个特定的公式编辑器点来点去的糟糕记忆,现在就偷着乐吧:现在,编写公式不需要鼠标了!使用\LaTeX 生成公式是一个广大的领域,我们这里仅仅会介绍一些用于生成“常用”公式所需的基本知识。因此,本章仅仅包含操作\LaTeX 公式的简短介绍。

\begin{ii}
\LaTeX 的标准指令足以生成大多数常见的数学方程。然而,建议使用美国数学学会(英:American Mathematical Society)发布的扩展amsmath和amssymb。可以在很多情况下,这两个扩展可以简化格式化过程。
\end{ii}

\section{编写数学公式的两种方式}

\LaTeX 可以识别两种数学公式。第一种是在文本中直接插入公式,就像这样:$ax+b=c$;另一种是将若干公式写在环境中,例如:
$$
{\rm d} U = \delta \mathcal{W} +\delta \mathcal{Q} 
$$

这两种模式都遵循一系列原则,涉及不同符号的字号和位置。如下示例使用了两种模式:

\begin{codelist}[3.1]{
  函数$f(x)$定义如下:
\begin{displaymath}
  f(x)=\sqrt{\frac{x-1}{x+1}}
\end{displaymath}
若其导函数存在,求其导函数。
}\begin{verbatim}
函数$f(x)$定义如下:
\begin{displaymath}
  f(x)=\sqrt{\frac{x-1}{x+1}}
\end{displaymath}
若其导函数存在,求其导函数。
\end{verbatim}
\end{codelist}

这个示例告诉我们,我们可以使用\dm{\$}符号来进入“内部”数学模式,并再次使用\dm{\$}符号退出。此外,这里使用了环境\dm{displaymath},这是最简单的生成数学式的方法。使用\verb|\[|和\verb|\]|也可以达成后者的效果(参见3.7.1节。)

3.7节会介绍\LaTeX 的不同环境。

