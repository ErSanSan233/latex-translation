\chapter*{翻译说明}

这是照猫画虎的翻译版本。

之所以这样说,是因为我虽然在字面上忠实翻译了原文,但在装帧上并没有严格套用原文的代码来编写。一方面,每种语言下使用的包不同,而作者使用的一些包搬到中文环境下会产生各种各样的错误,非要实现一模一样的效果实在是很费经历。另一方面,汉字和拉丁字母的排版有着细枝末节的差别,不能照搬照抄。

在翻译过程中,我很少去照着原书的代码翻译,而是打开成品PDF,看到原文用什么样的效果,就尝试按照自己的方式去实现类似的效果。在阅读过程中若是发现了书中给出的代码和效果不太一致的地方,那大概正是画虎画得不像的地方。

这样带来的损失是毋庸置疑的:一些原本联动跳转的地方变成了直接录入的文本,失去了超链接;一些本该随书写进度推进而动态变化的内容也被固定为特定版本。最可惜的是,原作者为本书准备了两个版本的成品PDF——用于显示的“彩色”版、用于印刷的“纸质”版,二者出了插图的颜色不同,对于某些格式的处理也不太一致。这一特性在译文中也没有实现。

实话说来,本书最有意思的地方就在于“自我指涉”,也就是说,在介绍较为复杂的版式实现方法时,往往是以自身的实现效果作为例子。在阅读中,你若发现本书某个地方的版式很新颖,不妨找来原文看看。

本书翻译过程中,对于一些细枝末节的格式有舍弃。这会带来格式不一致或违背标准做法的问题,但它们都是在翻译过程中发现的,修复起来需要大量排查,而对阅读影响不大,在我看来去处理它们有些得不偿失。具体如下:
\begin{itemize}
    \item 对原文引用古代典籍的部分,搜索对应译文直接引用,不进一步处理;
    \item 舍弃了原书不同环境下对左右侧栏宽度的调整;
    \item 舍弃了希腊字母应以非意大利体表示的情况;
    \item 舍弃了部分照抄(verbatim)环境下以特殊字体展示注释的方式;
    \item 放弃了对斜杠开头的代码(如\verb+\tableofcontents+等)两侧与中文间空格的统一处理;
    \item 放弃了对各章首微型目录的调整;
    \item 放弃了为每一种原文字体找到风格相近的中文字体的尝试;
    \item 放弃了所有明确标明页码的跳转提示,改用章节号代替;
    \item 放弃了制作书后索引;
    \item 忽略了格式套用带来的不一致问题(如提示框内脚注的位置问题);
    \item 忽略了对模板中结构性内容(自动生成“Chapter”“Figure”“Table”等文字)的汉化;
    \item 原书附录中展示了一些极特殊数学符号,它们在译文中显示为空白。
\end{itemize}

\section*{更多信息}

本书的GitHub库地址为:

\begin{center}
    \wz{https://github.com/ErSanSan233/latex-translation}
\end{center}

本书译文采用的中文字体如下:

\begin{itemize}
    \item 主体字体——STSongti-SC-Regular
    \begin{itemize}
        \item 粗——FandolSong-Bold
        \item 意大利——FandolKai-Regular
        \item 倾斜——FandolFang-Regular
    \end{itemize}
    \item 非衬线字体——思源黑体,半轻(NotoSansCJKsc-DemiLight)
    \begin{itemize}
        \item 粗——思源黑体,粗(NotoSansCJKsc-Bold)
        \item 意大利——钉钉进步体(DingTalk-JinBuTi)
        \item 倾斜——FandolFang-Regular
    \end{itemize}
    \item 等宽字体——汉仪正圆(HYZhengYuan-EEW)
    \begin{itemize}
        \item 粗——苹方,中黑(PingFangSC-Semibold)
        \item 意大利——手札体,粗(HannotateSC-W7)
        \item 倾斜——FandolFang-Regular
    \end{itemize}
    \item 本书也在一些地方使用了得意黑(SmileySans-Oblique)、Helvetica Neue。
\end{itemize}

\section*{有关原书}

本书原名为\emph{Tout ce que vous avez toujours voulu savoir sur \LaTeX \ sans jamais avoir osé le demander},副书名为\emph{Ou comment utiliser \LaTeX \ quand on n'y connaît goutte}。

本译文基于的原书版本为Ver 1.5。

我是在以下页面获取本书源文件的:

\begin{center}
    \wz{https://archives.framabook.org/tout-sur-latex/index.html}
\end{center}

我在库中的original-book.zip文件里保留了原书的全部信息,方便想要获取原书的人参考。

~\\

\begin{center}
    *

    *\quad *

    ~\\

    \textbf{祝阅读愉快!}
\end{center}