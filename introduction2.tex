\part{关于《关于\LaTeX 的那些你想知道却从不敢问的问题》的那些你想知道却从不敢问的问题}

\fancyhead[LE]{\bfseries\thepage}
\fancyhead[RO]{\bfseries\thepage}
% \fancyhead[LE]{\ongletpaire\bfseries\thepage}
% \fancyhead[RO]{\bfseries\thepage\ongletimpaire}

\chapter*{简介}

\begin{epigraphe}{《圣经·雅歌》7:2}
    王女阿,你的脚在鞋中何其美好。\\你的大腿圆润,好像美玉,是巧匠的手做成的。\\你的肚脐如圆杯,不缺调和的酒。\\你的腰如一堆麦子,周围有百合花\jz{
        本部分的题记都取自《雅歌》,与章标题毫无关联。
    }。
\end{epigraphe}

本部分的名为``关于《关于\LaTeX 的那些你想知道却从不敢问的问题》的那些你想知道却从不敢问的问题'',旨在解释此前的各章节是如何生成的,并借此介绍已定义的用于生成你当前看到的这本书的指令和环境。本部分的目标更宏大一些,因为我们希望为有勇气的读者提供用于创建其自己的风格的坚实知识基础……

在遇到读者询问是否可以复用本文档中这样或那样的风格的问题后,我萌生了编写接下来的章节的想法。对于我来说,继续向下编写这项工作有些艰巨,因为我需要介绍的\LaTeX 知识超出了基础知识的范畴,也因此更难解释。最后,同样重要的是,在这里,我自己的巴扎中的剩货通常必须被``合理化'',才能成为可介绍的内容。这可不是件好搞的事。

在这一部分,我想介绍些生成本文档时使用的手段。我的手段不是能获得你当前看到的版式的唯一方法。例如,文档中的一些部分可以借助一些具有类似功能的包来完成,这些包甚至能比这里开发的工具生成更好的效果。

这一部分隐藏的思想是将好奇的读者领到探索\LaTeX 的小路上,并向他们展示:我们可以借助几个工具,将原指令校准到可以严格适用于他们自己的需求的程度。这些小路具有足够的普适性,可以用于按图索骥,也可以针对类似或不同的情况来调整。这里的关键,一方面是发现``\LaTeX ''内部功能的经典之处,另一方面通过创造自己的工具来获得满足感——但针对这件事,我们不宣扬``重新发明''轮子。

我尝试尽可能只介绍\LaTeX 指令。然而,有时也有必要使用一些\TeX 的功能,这里也是一个介绍这些功能的机会。因此,这一部分由三章组成。

\begin{description}
    \item[必要工具] 介绍需要了解的指令,以为接下来的工作做准备。例如,在这里,我们可以找到\LaTeX 发行版中文件结构的踪迹、切换文档弟子的思路,以及关于基于列表创建新环境的详细介绍;
    \item[装饰] 介绍了我们实现的工具,它们用于修改标题、页眉页脚、侧栏以及其他一些细枝末节的风格。
    \item[新玩具] 这里解释了本书的侧面标签、术语字典、示例、摘要、首字下沉、提示框,以及其他一些细碎的内容的创建过程。
\end{description}

\begin{qquestion}
这些章节中给出的一些解释显得有些云里雾里,即使是作者也无法理解。问题的一些解决方案是片面的。然而,对于鄙人来说,一些内容仍然很神秘。在这些情况下,段落中会插入这种``路面不平''(dos d'âne)\yz{
    原作者已经把这个标志替掉了。在源文档中可以找到名为\dm{dosdane.eps}的文件,指的就是这里说的已经被弃用的标志。
}标志。
\end{qquestion}