\part{关于《关于\LaTeX 的那些你想知道却从不敢问的问题》的那些你想知道却从不敢问的问题}

\chapter*{简介}

\begin{quote}
    王女阿,你的脚在鞋中何其美好。你的大腿圆润,好像美玉,是巧匠的手做成的。你的肚脐如圆杯,不缺调和的酒。你的腰如一堆麦子,周围有百合花\jz{
        本部分的题记都取自《雅歌》,与章标题毫无关联。
    }。

    \hfill《圣经·雅歌》7:2
\end{quote}

本部分的名为“关于《关于\LaTeX 的那些你想知道却从不敢问的问题》的那些你想知道却从不敢问的问题”,旨在解释此前的各章节是如何生成的,并借此介绍已定义的用于生成你当前看到的这本书的指令和环境。本部分的目标更宏大一些,因为我们希望为有勇气的读者提供用于创建其自己的风格的坚实知识基础……

在遇到读者询问是否可以复用本文档中这样或那样的风格的问题后,我萌生了编写接下来的章节的想法。对于我来说,继续向下编写这项工作有些艰巨,因为我需要介绍的\LaTeX 知识超出了基础知识的范畴,也因此更难解释。最后,同样重要的是,在这里,我自己的巴扎中的剩货通常必须被“合理化”,才能成为可介绍的内容。这可不是件好搞的事。

在这一部分,我想介绍些生成本文档时使用的手段。我的手段不是能获得你当前看到的版式的唯一方法。例如,文档中的一些部分可以借助一些具有类似功能的包来完成,这些包甚至能比这里开发的工具生成更好的效果。

这一部分隐藏的思想是将好奇的读者领到探索\LaTeX 的小路上,并向他们展示:我们可以借助几个工具,将原指令校准到可以严格适用于他们自己的需求的程度。这些小路具有足够的普适性,可以用于按图索骥,也可以针对类似或不同的情况来调整。这里的关键,一方面是发现“\LaTeX ”内部功能的经典之处,另一方面通过创造自己的工具来获得满足感——但针对这件事,我们不宣扬“重新发明”轮子。

