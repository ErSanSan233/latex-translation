\chapter{概要手册}

在本章,你可以找到关于\LaTeX 扩展的``散装''信息:一个足够完整的列表。这个列表列举了围绕在你的源文件周围的辅助文件。接下来,本章会包含\textsf{Emacs}的卓越模块——Auc\TeX 的简要解释。最后,针对那些有幸使用UNIX环境工作的人,本章末尾带有与拼写检查器\textsf{Aspell}协同工作的\textsf{Emacs}配置。

\section{扩展}

正如前言所说,\TeX 和\LaTeX 都是\emph{开放}系统。围绕着\LaTeX 内核,有相当数量的标准包建起了该系统的基础。然而,所有的用户都可以为\LaTeX 添加不同的功能,来让它进化。因此,我们搜寻了一批扩展[extension;也可以按英文称为包(\emph{package})]格式或文档类型格式的工具。一些工具已经称为了标准,``所有''工具都可以在\LaTeX 发行版(参见第8章)的服务器或私人页面上下载,其他的则随论文征集或其他作者指南提供。

在这里,我们为你提供一个``标准''包的列表,并且邀请你去参阅通常随包附带的相关文档。注意,CTAN(\wz{http://www.ctan.org})提供了包目录项的参考。
% TODO 缺三角
\begin{packages}
    \item[french]用于将文档``法文化''。避免在词和双标点之间打断句子。同时提供若干围绕法文排版的指令(参见第7章)。
    \item[amsmath]用于制作完美的数学式和方程的包。
    \item[array]改善\dm{tabular}的使用。
    \item[hhline]扩展\LaTeX 基础表格的边线。
    \item[fancyhdr]用于定制页眉和页脚。请看看本书的页眉和页脚。
    \item[varioref]提供指令\verb+\vref+,用于取代\verb+\ref+,根据跳转阅读的提示语和目的位置的相对关系添加``下一页''``第12页''或空白等格式。
    \item[ifthen]提供两种\emph{控制结构},即``if then else''和``do while''。这使得稍微复杂些的指令成为可能。
    \item[chapterbib]用于在各章末插入参考文献。
    \item[overcite]将参考文献标注为上标。
    \item[bibunits]用于生成多个单位组成的参考文献。
    \item[fancybox]提供4种\verb+\fbox+的变体,即\shadowbox{\verb+\shadowbox+}、\doublebox{\verb+\doublebox+}、\ovalbox{\verb+\ovalbox+}、\Ovalbox{\verb+\Ovalbox+}。
    \item[algorithms]以可浮动或不可浮动的环境格式展示代码。
    \item[geometry]用于以一种足够灵活的方式修改边距和页面上大部分相关尺寸。
    \item[url]用于以URL格式展示网址,其中的断字``以最佳方式''处理。
    \item[fancyvrb]提供了环境\dm{verbatim}的改进版本。
\end{packages}

\section{辅助文件}

如下列表展示了你可以在磁盘中源文件旁找到的文件。这些文件的扩展名都由3个字母组成,具体如下\jz{
    一些包(例如\textsf{minitoc}和类\textsf{lettre})可以创建它们自己的辅助文件,本列表不涵盖这些情况
}:

\begin{ficaux}
    \item[tex]\LaTeX 源文件;
    \item[aux]\LaTeX 用于解析引用等内容的辅助文件;
    \item[log]跟踪文件[fichier de trace;英文成为\emph{日志文件(log file)}];
    \item[dvi]\emph{设备无关(device independant)}文件,根据使用场景可以用于显示或打印;
    \item[toc]包含目录的文件(代表\emph{table of contents});
    \item[lof]包含图列表的文件(代表\emph{list of figure});
    \item[lot]包含表格列表的文件;
    \item[bib]包含的参考文献入口的\bib 源文件;
    \item[bbl]包含参考文献的文件,可以由\bib 生成;
    \item[blg]\bib 的跟踪文件;
    \item[idx]未经筛选的索引入口文件;
    \item[ind]包含索引的文件,通常由\textsf{makeindex}生成;
    \item[ilg]\textsf{makeindex}跟踪文件;
    \item[sty]包含更改版式或支持特定工具的指令的定义的文件。
    \item[cls]定义文档类型的文件。 
\end{ficaux}

在将\LaTeX 文档归档时,不同的文件可以有不同的处理方式。

\begin{description}
    \item[可以删除的文件]所有的辅助文件、\emph{log}文件,以及目录文件和图表列表文件都可以删除。
    \item[同样可以删除的文件]\dm{ddl}文件(如果你可以从\dm{bib}文件借助\bib 生成它)。索引文件通常同样可以删除,因为它们原则上是\textsf{makeindex}生成的。\dm{dvi}文件同样不是必需的,因为我们认为你已经有\LaTeX 源文件了。
    \item[需要保留的文件]\LaTeX 源文件和你定义的所有必需的风格文件(\dm{sty}和\dm{cls}文件)都需要保留。但如果你已经知道如何定义文档类型了,那么这个建议可能有点蠢……
\end{description}

\section{Auc\TeX }

Auc\TeX 是\textsf{Emacs}的一个模块,它简化了\LaTeX 文件的录入。我们打开带有扩展名\dm{.tex}、\dm{.sty}或\dm{.cls}的文件时,它会自动加载。Auc\TeX 可以为我们提供3大类功能:

\begin{enumerate}
    \item 源文件的排版(颜色、缩进等);
    \item 用于插入指令和环境的快捷键;
    \item 编译。
\end{enumerate}

\subsection{源文件的排版}

在C和C++的缓冲区,颜色和键\textsf{tab}扮演相同的角色。我们可以注意到,\fbox{\dm{M-q}}可以将段落``排版'',也就是自动将段落打断成长度相近的若干行。

\subsection{快捷键}

\subsubsection{字体}

\fbox{\dm{C-c C-f}}[代表切换字体(\underline{C}hanger \underline{F}ont)]后接以下内容:

\begin{itemize}
    \item \fbox{\dm{C-e}},可以插入\verb+\emph{}+;
    \item \fbox{\dm{C-b}},可以插入\verb+\textbf{}+;
    \item \fbox{\dm{C-t}},可以插入\verb+\texttt{}+;
    \item \fbox{\dm{C-s}},可以插入\verb+\textsl{}+;
    \item \fbox{\dm{C-c}},可以插入\verb+\textsc{}+;
    \item ……
\end{itemize}

\subsubsection{节}

\fbox{\dm{C-c C-s}}可以插入节(\underline{S}ection),并要求你提供其层级、标题,以及在微型缓冲区中的标签。%TODO 啥?以及它这个奇怪的快捷键代表啥

\subsubsection{指令和环境}

\fbox{\dm{M-Tab}}可以补全当前的名称(\emph{英:automatic completion})。
\fbox{\dm{C-c RET}}插入一条指令。
\fbox{\dm{C-c C-e}}插入一个环境(\underline{E}nvironnement)\jz{
    对于语法可被Auc\TeX 识别的某些环境和某些指令,则需要提供一些细节(参数值、图例、表格的形式……)。
}。
\fbox{\dm{C-u C-c C-e}}更改一个环境。
\fbox{\dm{C-c ]}}插入缺失的\verb+\end+来关闭一格环境。

\subsection{编译}

使用\fbox{\dm{C-c C-c}}可以尝试遵循文档编译的流程,根据情况运行\LaTeX 、\bib 或\textsf{xdvi}……同样需要注意,使用Auc\TeX 可以管理主文件的机制(参见6.4节)。为此,则需要在\textsf{Emacs}中打开新文件时输入主文件的文件名。相反,则需要通过以下方式耐心地向Auc\TeX 解释哪个文件是主文件:

\dmhmake{M-x TeX-master-file-ask}\textrm{ 或 }\textsf{C-c\_}%啥?以及这个字体其实要换成Helvetica

你同样需要输入主文件的文件名。这样一来,在你通过\fbox{\dm{C-c C-c}}在``子''文档上执行编译时,将作用于\emph{主}文件。

\subsection{Aspell}

\textsf{Aspell}是一个多语种拼写纠正工具,可以与万能的文本编辑器\textsf{Emacs}对接。若要在\textsf{Emacs}中使用它,需要掌握两条指令:

\dmhmake{M-x ispell-change-dictionary}

以上指令用于选择字典语言(法文\dm{francais}或英文\dm{english});

\dmhmake{M-x ispell-buffer}

以上指令可以在缓冲区开启一次纠错过程。我们也可以借助以下指令检查单独的词的拼写:

\dmhmake{M-xispell-word}\textrm{或 }\textsf{M-\$}

\begin{exclamation}
在我写下这行文字时,发行版Debian配置Emacs默认不是用程序Aspell。因此,需要在你的\dm{.emacs}文件中添加以下代码行:

\begin{dmd}
(setq-default ispell-program-name "aspell")
\end{dmd}
\end{exclamation}

尤其实用的是,我们可以通过配置程序\textsf{Aspell},来明确地要求它忽略或不忽略\LaTeX 指令的参数。例如,我们可以跳过不含有法文的指令中的参数。这样一来,如果我们定义了以下指令:

\begin{dmd}
\begin{verbatim}
\newcommand{\bidule}[2]{% 接受2个参数的指令
    ... }\end{verbatim}
\end{dmd}

只需要在其文件\verb+~/.aspell.conf+中编写如下内容,即可要求\textsf{Aspell}验证第二个参数(\dm{P})而忽略第一个参数的内容(\dm{p}):

\begin{dmd}
add-tex-command bidule pP
\end{dmd}

若要同时忽略两个参数,则需要编写如下内容:

\begin{dmd}
    add-tex-command bidule pp
\end{dmd}

最后,\textsf{Emacs}也提供了单词纠错的``飞打''(à la volée)\jz{
    有些人会称``随词''(\textsl{译注}:原文为à la ouôrde,其中ouôrde即模仿英文word的发音)……
}模式,我们可以通过以下指令来设定该模式是否激活:

\dmhmake{M-x flyspell-mode}

