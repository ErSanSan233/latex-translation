\chapter{概要手册}

在本章,你可以找到关于\LaTeX 扩展的“散装”信息:一个足够完整的列表。这个列表列举了围绕在你的源文件周围的辅助文件。接下来,本章会包含\textsf{Emacs}的卓越模块——Auc\TeX 的简要解释。最后,针对那些有幸使用UNIX环境工作的人,本章末尾带有与拼写检查器\textsf{Aspell}协同工作的\textsf{Emacs}配置。

\section{扩展}

正如前言所说,\TeX 和\LaTeX 都是\emph{开放}系统。围绕着\LaTeX 内核,有相当数量的标准包建起了该系统的基础。然而,所有的用户都可以为\LaTeX 添加不同的功能,来让它进化。因此,我们搜寻了一批扩展[extension;也可以按英文称为包(\emph{package})]格式或文档类型格式的工具。一些工具已经称为了标准,“所有”工具都可以在\LaTeX 发行版(参见第8章)的服务器或私人页面上下载,其他的则随论文征集或其他作者指南提供。

在这里,我们为你提供一个“标准”包的列表,并且邀请你去参阅通常随包附带的相关文档。注意,CTAN(\wz{http://www.ctan.org})提供了包目录项的参考。
% TODO 缺三角
\begin{packages}
    \item[french]用于将文档“法文化”。避免在词和双标点之间打断句子。同时提供若干围绕法文排版的指令(参见第7章)。
    \item[amsmath]用于制作完美的数学式和方程的包。
    \item[array]改善\dm{tabular}的使用。
    \item[hhline]扩展\LaTeX 基础表格的边线。
    \item[fancyhdr]用于定制页眉和页脚。请看看本书的页眉和页脚。
    \item[varioref]提供指令\verb+\vref+,用于取代\verb+\ref+,根据跳转阅读的提示语和目的位置的相对关系添加“下一页”“第12页”或空白等格式。
    \item[ifthen]提供两种\emph{控制结构},即“if then else”和“do while”。这使得稍微复杂些的指令成为可能。
    \item[chapterbib]用于在各章末插入参考文献。
    \item[overcite]将参考文献标注为上标。
    \item[bibunits]用于生成多个单位组成的参考文献。
    \item[fancybox]提供4种\verb+\fbox+的变体,即\shadowbox{\verb+\shadowbox+}、\doublebox{\verb+\doublebox+}、\ovalbox{\verb+\ovalbox+}、\Ovalbox{\verb+\Ovalbox+}。
    \item[algorithms]以可浮动或不可浮动的环境格式展示代码。
    \item[geometry]用于以一种足够灵活的方式修改边距和页面上大部分相关尺寸。
    \item[url]用于以URL格式展示网址,其中的断字“以最佳方式”处理。
    \item[fancyvrb]提供了环境\dm{verbatim}的改进版本。
\end{packages}

\section{辅助文件}

如下列表展示了你可以在磁盘中源文件旁找到的文件。这些文件的扩展名都由3个字母组成,具体如下\jz{
    一些包(例如\textsf{minitoc}和类\textsf{lettre})可以创建它们自己的辅助文件,本列表不涵盖这些情况
}:

\begin{ficaux}
    \item[tex]\LaTeX 源文件;
    \item[aux]\LaTeX 用于解析引用等内容的辅助文件;
    \item[log]跟踪文件[fichier de trace;英文成为\emph{日志文件(log file)}];
    \item[dvi]\emph{设备无关(device independant)}文件,根据使用场景可以用于显示或打印;
    \item[toc]包含目录的文件(代表\emph{table of contents});
    \item[lof]包含图列表的文件(代表\emph{list of figure});
    \item[lot]包含表格列表的文件;
    \item[bib]包含的参考文献入口的\bib 源文件;
    \item[bbl]包含参考文献的文件,可以由\bib 生成;
    \item[blg]\bib 的跟踪文件;
    \item[idx]未经筛选的索引入口文件;
    \item[ind]包含索引的文件,通常由\textsf{makeindex}生成;
    \item[ilg]\textsf{makeindex}跟踪文件;
    \item[sty]包含更改版式或支持特定工具的指令的定义的文件。
    \item[cls]定义文档类型的文件。 
\end{ficaux}

在将\LaTeX 文档归档时,不同的文件可以有不同的处理方式。

\begin{description}
    \item[可以删除的文件]所有的辅助文件、\emph{log}文件,以及目录文件和图表列表文件都可以删除。
    \item[同样可以删除的文件]\dm{ddl}文件(如果你可以从\dm{bib}文件借助\bib 生成它)。索引文件通常同样可以删除,因为它们原则上是\textsf{makeindex}生成的。\dm{dvi}文件同样不是必需的,因为我们认为你已经有\LaTeX 源文件了。
    \item[需要保留的文件]\LaTeX 源文件和你定义的所有必需的风格文件(\dm{sty}和\dm{cls}文件)都需要保留。但如果你已经知道如何定义文档类型了,那么这个建议可能有点蠢……
\end{description}

\section{Auc\TeX }