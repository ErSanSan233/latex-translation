\chapter{需要了解的知识}

\begin{quote}
    亵慢的人受刑罚,愚蒙的人就得智慧。智慧人受训诲,便得知识。——《圣经·箴言》12:11
\end{quote}


本章要研究使用\LaTeX 生成文档时的基本排版指令。我们将零散地处理用于突出显示、\LaTeX 标准环境、标题、页面下方的注释、页眉和页脚,以及浮动的环境。接下来,我们会介绍参考系统和\LaTeX 生成的辅助文件。最后,阅读到本章末尾的人将有机会读到一些关于断字的思考。

所有这些指令都将以其默认行为模式使用。也就是说,我们这里不介绍重新定义它们的方法。对应地,你将能够以传统的版式来生成文档。若要打出一篇更进阶的文章,你需要了解如何输入数学式(第3章)、一些关于科技文档的知识(第6章),以及包含图像的方法(第5章)。

\section{突出显示}

要了解\LaTeX 选用字体的机制,需要知道:我们通常通过4个参数来区别字体。

\paragraph*{族(famille)}指字体的整体形状。默认情况下,\LaTeX 使用三种字体族:罗马体、\textsf{无衬线体}、\texttt{打字机体}。\LaTeX 中,以英文单词\textit{family}来指代字体族。

\paragraph*{风格(style)}指字体体现出的体态(英文以\textit{shape}指代),分为:\textit{意大利体}、\textsl{倾斜}和\textsc{小型大写字母风格(petites capitales)}\yz{
    本书翻译时,以楷体对应意大利体、以仿宋体对应倾斜排版,按原文翻译。中文字体的族和风格往往是并列的,很少有交叉叠加的情况,因此在展示叠加效果时酌情保留原文。中文中很少见到类似小型大写字母的突出显示方式。因此,除了特意展示西文的部分外,本书忽略小型大写字母格式,必要时以其他风格取代。
}。

\paragraph*{字重(graisse)}指字体笔画的粗细(\LaTeX 中以\textit{serie}指代)。默认情况下有两种粗细:中等和\textbf{加粗}。

\paragraph*{字号}字{\large 体}{\Large 的}{\LARGE 大}{\small 小}。

\subsection{族--风格--字重}

有两种不同的宏可以设置族、风格、字重这三个变量:\textit{指令}和\textit{声明}(如表\ref{tab:2.1}所示)。指令以花括号的形式将其参数括住。声明可以打断行文,同时修改三个变量之一,直到新的命令出现。总体上的规则是,我们使用指令来突出显示一个词或一组词:

\begin{table}
    \centering
    \begin{tabular}{|c|l|c|}
\hline
指令 & 声明 & 输出\\
\hline
\verb+\textrm{...}+ & \verb+{\rmfamily ...}+ & 罗马(roman) \\
\verb+\textsf{...}+ & \verb+{\sffamily ...}+ & \textsf{非衬线(sans sérif)} \\
\verb+\texttt{...}+ & \verb+{\ttfamily ...}+ & \texttt{打字机(machine à écrire)} \\
\hline
\verb+\textup{...}+ & \verb+{\upshape ...}+ & 正(droit) \\
\verb+\textit{...}+ & \verb+{\itshape ...}+ & \textit{意大利(italique)} \\
\verb+\textsl{...}+ & \verb+{\slshape ...}+ & \textsl{倾斜(penché)} \\
\verb+\textsc{...}+ & \verb+{\scshape ...}+ & \textsc{小型大写(Petites Capitales)} \\
\hline
\verb+\textmd{...}+ & \verb+{\mdseries ...}+ & 中等(médium) \\
\verb+\textbf{...}+ & \verb+{\bfseries ...}+ & \textbf{加粗(gras)} \\
\hline
    \end{tabular}
    \caption{更改字体的声明}
    \label{tab:2.1}
\end{table}

\begin{codelist}[2.1]{
    \texttt{char}类型的\emph{变量}\textsl{总是}被编码为\textbf{8 位}。
}
\begin{verbatim}
\texttt{char}类型的\emph{变量}\textsl{总是}被编码为\textbf{8 位}。
\end{verbatim}
\end{codelist}

注意,在上面的命令中,指令\verb|\emph|(对应的声明是\verb|\em|,可以以更优雅的方式突出显示一组词)。相较声明,强烈建议使用\emph{指令}。当要修改文本的一部分时,使用指令是更明智的选择\jz{定义指令也是。}:

\begin{codelist}[2.2]{
    {\em \bfseries 马格马(Magma) \mdseries 的音乐就像一面镜子,每个人都能看到他自己的倒影。}
}
    \begin{verbatim}
{\em \bfseries 马格马(Magma) \mdseries 的音乐
就像一面镜子,每个人都能看到他自己的倒影。}
    \end{verbatim}
\end{codelist}

接下来的例子展示了如何使用组。声明\verb|\slshape|出现在一个组中,因此它只在组内发挥作用。此外,组会继承它外层的组的参数。这样一来,“silence”一词会使用\textsf{非衬线}体(根据外层的组),并且\textsl{倾斜}展示(根据内层的声明):

\begin{codelist}[2.3]{
    \sffamily 在爵士乐中,{\slshape 沉默(silence)\/}永远是正确的。因此,这是一种充满万千可能的音乐。
}\begin{verbatim}\sffamily 在爵士乐中,
{\slshape 沉默(silence)\/}永远是正确的。因此,
这是一种充满万千可能的音乐。
\end{verbatim}
\end{codelist}

\subsection{意大利体校正}

另一个推荐使用指令而不是声明的理由是,与声明不同,指令可以实现\emph{意大利体校正}。所谓意大利体校正,是指在以意大利体显示的字符组后,有必要增加一个间距,使得这组字符不会“碰到”后面的词。这个间距与所涉及的字符有关\yz{此例展示不同情况下\emph{f}和a间距的细微差距,宜保留原文。例句意为:首长永远是对的。}:

\begin{codelist}[2.4]{
    le {\em chef} a toujours raison.\par
    le {\em chef\/} a toujours raison.\par
    le \emph{chef} a toujours raison.\par
}\begin{verbatim}le {\em chef} a toujours raison.\par
le {\em chef\/} a toujours raison.\par
le \emph{chef} a toujours raison.\par
\end{verbatim}
\end{codelist}

我们可以看到,指令\verb|\emph|实现了校正,然而,若要使用声明,则需要明确地借助宏\verb|\/|来实现相同的效果。

\subsection{字号}

表\ref{tab:2.2}中展示的宏可以修改行文的字号。这些宏都是\emph{声明}。同时,对于每一个宏都存在同名的环境。

\begin{table}[h]
    \centering
    \begin{tabular}{|c|c||l|c|}
\hline
\verb|\Huge| & {\Huge 宏大} & \verb|\normalsize| & {正常} \\
\verb|\huge| & {\huge 庞大} & \verb|\small| & {\small 小} \\
\verb|\LARGE| & {\LARGE 特大} & \verb|\footnotesize| & {\footnotesize 加小} \\
\verb|\Large| & {\Large 加大} & \verb|\scriptsize| & {\scriptsize 特小} \\
\verb|\large| & {\large 大} & \verb|\tiny| & {\tiny 微小} \\
\hline
    \end{tabular}
    \caption{修改字号}
    \label{tab:2.2}
\end{table}

\subsection{几个建议}

习惯上,我们应当尽量减少字体变化。实际上,如果字体变化得不合适宜或没有实际用处,尤其是喧宾夺主,影响了文档的可读性,看起来就会很低级。这里是有关使用字体变化的几条建议(仍然是习惯上的建议!):
\begin{itemize}
    \item 相较于使用其余指令,更多地使用\verb+\emph+(默认会将字体改为\emph{意大利}体)。
    \item 将\textbf{加粗}的机会留给特别重要的提示。
    \item 在法文文档中,几乎只在人名中使用小型大写字母(如Donald \textsc{Knuth})。
    \item \dm{打字机}字体族经常被用于生成编程语言的代码或类似的内容。
\end{itemize}

以下内容适用于能读懂其中道理的人……

除以上内容外,我们给出以下两个用于突出显示的情况,读者可以字形斟酌:改变字号、添加下划线(使用指令\verb|\underline|)。

\begin{quote}
    “也许那些希望{\tiny 小声说话}的诗人会让图书的字体频繁变化,但目前,只有一些字体狂人{\tiny (比如本手册的作者)}才喜欢这样做。”——克努特,\TeX Book[9]
\end{quote}

\begin{quote}
    “注意,出版行业认为,以添加下划线的形式来强调内容是不好的习惯。下划线只应用于一种情况——输出设备无法以其他方式突出显示内容,比如使用打字机。”——迈克尔·古森斯(Michel Goossens)等,《\LaTeX 伴侣》(\textit{\LaTeX{} Companion})[6]
\end{quote}

\section{环境}

\LaTeX 以\emph{环境}格式提供了一系列工具,具体结构是一个代码块,语法如下:

\begin{dmd}
    \verb+\begin+\textsl{\<环境名\>}\verb|}|\\
    \verb|...|\\
    \verb+\end+\textsl{\<环境名\>}\verb|}|\\
\end{dmd}

其中\textsl{\<环境名\>}需要替换为具体的环境名称。我们目前遇到的第一个环境是\dm{document}。\verb|\begin| 和\verb|end|间的文字会以特殊版式展示。

\begin{exclamation}
    我们立刻注意到,环境中所有声明都是局部的。另外,当然可以在我们自己定义的环境中套用已经存在的环境。
\end{exclamation}

\subsection{居中和对齐}

想要居中显示几行文字,我们使用环境\dm{center}:

\begin{codelist}[2.5]{
    ……上一段的结尾。
    \begin{center}
        完美居中的\\
        几行文字\\
        且前后带有间距
    \end{center}
    然后继续下一段……
}\begin{verbatim}……上一段的结尾。
\begin{center}
  完美居中的\\
  几行文字\\
  且前后带有间距
\end{center}
然后继续下一段……
\end{verbatim}
\end{codelist}

同样,我们可以轻易地借助环境\dm{flushright},让一个段落右对齐排列:

\begin{codelist}[2.6]{
    ……上一段的结尾。
    \begin{flushright}
      两行右对齐\\
      的文字
    \end{flushright}
    然后继续下一段……
    }{
\begin{verbatim}……上一段的结尾。
\begin{flushright}
  两行右对齐\\
  的文字
\end{flushright}
然后继续下一段……
\end{verbatim}
    }
\end{codelist}

注意,以上两个示例中,命令\verb|\\|起到了换行的功能。除一些特殊场景(表格、文档的标题和作者、特意的居中或对齐处理)外,不要使用这个命令——如果想要换行,需要使用空行或命令\verb|\par|。

\begin{flushleft}
    一般情况下,我们使用环境\dm{flushleft}时需要搭配命令\verb|\\|。但我们可以使用这个环境来生成右侧不对齐的文档,将换行的麻烦工作留给\LaTeX ,就像这段文字一样。(译注:此处给出本段原文。请注意右侧的换行的不对齐处理。En général, on emploie l'environnement \dm{flushleft} avec des commandes \verb|\\|. Mais on peut l'utiliser pour produire un paragraphe comme celui-ci, non justifié à droite, en laissant à \LaTeX{} le soin d’insérer les sauts de lignes.)
\end{flushleft}

%TODO 排查代码下正文内容
\begin{exclamation}
    绝大部分环境都会重启一行来插入内容。然而,重要的是:环境只是在插入内容的位置\emph{中断}当前段落,而不是结束当前段落。你可以在前两个实例中看到,“然后继续下一段……”这句话前没有缩进。另外,\LaTeX 贴心地在每个环境前后都留了一段空白。
\end{exclamation}

我们可以注意到,前面的三个环境分别代表以下三个声明:

\begin{itemize}
    \item \verb|\centering|;
    \item \verb|\raggedleft|;
    \item \verb|\raggedright|。
\end{itemize}

例如,我们可以这样写\yz{
    实际上,Emacs代表Editor MACroS。本例是对Emacs的调侃。
}:

\begin{codelist}[2.7]{
    Emacs代表:

    {\centering Emacs\\Makes\\
      A\\Computer\\Slow\\}
}{\begin{verbatim}Emacs代表:

{\centering Emacs\\Makes\\
  A\\Computer\\Slow\\}
\end{verbatim}
}
\end{codelist}

\subsection{列表}

\LaTeX 提供了三种呈现\emph{列表}的基本环境:\dm{itemize}、\dm{enumerate}、\dm{description}。如果它们都不能满足你,也可以定义属于你自己的\celan{9.5节}列表。但目前,先来看看标准的列表。

首先,\dm{itemize}可以生成不编号的项目列表。在法文版本中,一级列表会使用连接号(---)标记;在其他版本中,会使用点($\bullet$)标记:

\begin{codelist}[2.8]{
……一句话的结尾。
\begin{itemize}
\item 在复杂的计算中,分子的系数需要传递给分母;
\item 不应当写逗号。
\end{itemize}
然后行文继续。
}\begin{verbatim}
……一句话的结尾。
\begin{itemize}
\item 在复杂的计算中,
  分子的系数需要
  传递给分母;
\item 不应当写逗号。
\end{itemize}
然后行文继续。
\end{verbatim}
\end{codelist}

环境\dm{enumerate}遵循类似的规则,只不过项目会被编号。一个环境中可以套用另一个环境。下面的例子中,我们同时展示了\dm{ enumerate}和\dm{description}环境:

\begin{codelist}[2.9]{
    ……还是一句话的结尾。
    \begin{description}
    \item[\TeX] \TeX{}book;
    \item[\LaTeX] 两本重要的书:
      \begin{enumerate}
      \item 《\LaTeX{}:一个文档准备系统》(\emph{\LaTeX{}: A Document Preparation System})。
      \item 《\LaTeX{}伴侣》。
      \end{enumerate}
    \end{description}
    跟之前一样,接下来段落继续……
}\begin{verbatim}……还是一句话的结尾。
\begin{description}
\item[\TeX] \TeX{}book;
\item[\LaTeX] 两本重要的书:
  \begin{enumerate}
  \item 《\LaTeX{}:一个文档
    准备系统》(\emph{\LaTeX{}: 
    A Document Preparation System})。
  \item 《\LaTeX{}伴侣》。
  \end{enumerate}
\end{description}
跟之前一样,接下来段落继续……
\end{verbatim}
\end{codelist}

至于环境\dm{description}的列表,在我们习惯的文档处理中没有对应内容。不幸的是,对于\LaTeX 初学者,它最好的结果是被误用,最差的结果是被无视。

\subsection{制表}

环境\dm{tabbing}可以用于打字机上使用的那种老式制表过程%TODO查证
。我们可以使用指令\verb|\=|来放置定位标记,并使用指令\verb|\>|来在定位标记之间移动。此外,\verb|\\|可以用来换行。

\begin{codelist}[2.10]{
  \begin{tabbing}
    左倾 \= 中间立场 \= 右倾 \\
    \> 中立派 \\
    \> \> 保守派 \\
    xxxxxxxxx \= \kill
    \> 无观点
    \end{tabbing}
}\begin{verbatim}
\begin{tabbing}
  左倾 \= 中间立场 \= 右倾 \\
  \> 中立派 \\
  \> \> 保守派 \\
  字字字字字 \= \kill
  \> 无观点
\end{tabbing}
\end{verbatim}
\end{codelist}

我们可以从这个例子中看到两个规则:

\begin{itemize}
    \item 可以在制表时插入一行“模板”,并且使用指令\verb|\kill|来隐藏这一行;
    \item 如果已经存在定位标记,新的指令\verb|\=|可以在逻辑上移除它们。
\end{itemize}

\subsection{表格}

\LaTeX 中用于生成\emph{表格}的环境称为\dm{tabular}。表线的处理可能不太精细,但对于线条简单的表格,结果可以接受\jz{
    附录B提供了一些提示,可以用来找到能够生成更复杂表格的包。
}:

\begin{codelist}[2.11]{
    嚯:
    \begin{tabular}{|r|c|}
        \hline
        俩 & 仨 \\
        五个 & 六个  \\ \hline
      \end{tabular}
}\begin{verbatim}嚯:
\begin{tabular}{|r|c|}
  \hline
  俩 & 仨 \\
  五个 & 六个  \\ \hline
\end{tabular}
\end{verbatim}
\end{codelist}
通过这个例子,我们可以得到以下结论。

\begin{itemize}
    \item 环境\dm{tabular}会等待输入一个参数,应用指示表格的格式。每列都应该以一个定位字符表示。
    \begin{itemize}
        \item \dm{r}:右对齐。
        \item \dm{c}:居中。
        \item \dm{l}:左对齐。
    \end{itemize}
    \item 字符\verb|&|用于分隔不同列。
    \item 指令\verb|\\|用于跳转至下一行。
    \item 布局字符串中的\dm{|}表示插入纵向表线。
    \item 横向表线由指令\verb|\hline|插入。
\end{itemize}

因此,我们可以自由地调整\verb|\hline|和\dm{|}的数量,以隐藏或显示表线。包\textsf{array}可以满足一些关于表格的幻想。

\begin{exclamation}
大多数环境都会另起一行,但\dm{tabular}不会。\dm{tabular}会紧接当前的文本生成表格。
\end{exclamation}

此外,我们还可以使用参数来精确地指定表格竖直方向上的位置:

\begin{codelist}[2.12]{
一个表:\begin{tabular}[b]{|c|} 
甲\\乙
\end{tabular}
,另一个表:\begin{tabular}[t]{|c|} 
丙\\丁\end{tabular}
}\begin{verbatim}一个表:\begin{tabular}[b]{|c|} 
  甲\\乙
\end{tabular}
,另一个表:\begin{tabular}[t]{|c|} 
  丙\\丁
\end{tabular}
\end{verbatim}
\end{codelist}

我们可以看出,参数\dm{b}将表格“放置”在当前行上,参数\dm{t}将表格“悬挂”在当前行下。如果没有参数,表格会在竖直方向上居中,就像前面的例子一样。

显然,表格可以不插入句子中,而可以单独成段,比如配合环境\dm{center}被单独居中。

\subsection{模拟终端}

环境\dm{verbatim}可以将其内容逐字插入文档。因此,无论什么字符,甚至是特殊字符,都可以使用它来插入,如插入一个\textsf{C++}代码片段:

\begin{codelist}[2.13]{\ttfamily
  \begin{tabbing}
    cl\=ass pixel \{\\
      \>int x, y;\\
    public:\\
      \>pixel(int i=0, int j=0);\};
  \end{tabbing}
}
\ttfamily
\backslash begin\{verbatim\}
\begin{verbatim}
class pixel{
  int x,y;
public:
  pixel(int i=0, int j=0);};
\end{verbatim}
\backslash end\{verbatim\}
\end{codelist}

\begin{exclamation}
我们可以在环境\dm{verbatim}中写任何内容,\emph{除了}以下字符串——\verb|\end{verbatim}|!
\end{exclamation}

