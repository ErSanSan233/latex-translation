\chapter{需要了解的知识}

\begin{quote}
    亵慢的人受刑罚,愚蒙的人就得智慧。智慧人受训诲,便得知识。——《圣经·箴言》12:11
\end{quote}


本章要研究使用\LaTeX 生成文档时的基本排版指令。我们将零散地处理用于突出显示、\LaTeX 标准环境、标题、页面下方的注释、页眉和页脚,以及浮动的环境。接下来,我们会介绍参考系统和\LaTeX 生成的辅助文件。最后,阅读到本章末尾的人将有机会读到一些关于断字的思考。

所有这些指令都将以其默认行为模式使用。也就是说,我们这里不介绍重新定义它们的方法。对应地,你将能够以传统的版式来生成文档。若要打出一篇更进阶的文章,你需要了解如何输入数学式(第3章)、一些关于科技文档的知识(第6章),以及包含图像的方法(第5章)。

\section{突出显示}

要了解\LaTeX 选用字体的机制,需要知道:我们通常通过4个参数来区别字体。

\paragraph*{族(famille)}指字体的整体形状。默认情况下,\LaTeX 使用三种字体族:罗马体、\textsf{无衬线体}、\texttt{打字机体}。\LaTeX 中,以英文单词\textit{family}来指代字体族。

\paragraph*{风格(style)}指字体体现出的体态(英文以\textit{shape}指代),分为:\textit{意大利体}、\textsl{倾斜}和\textsc{小型大写字母风格(petites capitales)}\yz{
    本书翻译时,以楷体对应意大利体、以仿宋体对应倾斜排版,按原文翻译。中文字体的族和风格往往是并列的,很少有交叉叠加的情况,因此在展示叠加效果时酌情保留原文。中文中很少见到类似小型大写字母的突出显示方式。因此,除了特意展示西文的部分外,本书忽略小型大写字母格式,必要时以其他风格取代。
}。

\paragraph*{字重(graisse)}指字体笔画的粗细(\LaTeX 中以\textit{serie}指代)。默认情况下有两种粗细:中等和\textbf{加粗}。

\paragraph*{字号}字{\large 体}{\Large 的}{\LARGE 大}{\small 小}。

\subsection{族--风格--字重}

有两种不同的宏可以设置族、风格、字重这三个变量:\textit{指令}和\textit{声明}(如表\ref{tab:2.1}所示)。指令以花括号的形式将其参数括住。声明可以打断行文,同时修改三个变量之一,直到新的命令出现。总体上的规则是,我们使用指令来突出显示一个词或一组词:

\begin{table}
    \centering
    \begin{tabular}{|c|l|c|}
\hline
指令 & 声明 & 输出\\
\hline
\verb+\textrm{...}+ & \verb+{\rmfamily ...}+ & 罗马(roman) \\
\verb+\textsf{...}+ & \verb+{\sffamily ...}+ & \textsf{非衬线(sans sérif)} \\
\verb+\texttt{...}+ & \verb+{\ttfamily ...}+ & \texttt{打字机(machine à écrire)} \\
\hline
\verb+\textup{...}+ & \verb+{\upshape ...}+ & 正(droit) \\
\verb+\textit{...}+ & \verb+{\itshape ...}+ & \textit{意大利(italique)} \\
\verb+\textsl{...}+ & \verb+{\slshape ...}+ & \textsl{倾斜(penché)} \\
\verb+\textsc{...}+ & \verb+{\scshape ...}+ & \textsc{小型大写(Petites Capitales)} \\
\hline
\verb+\textmd{...}+ & \verb+{\mdseries ...}+ & 中等(médium) \\
\verb+\textbf{...}+ & \verb+{\bfseries ...}+ & \textbf{加粗(gras)} \\
\hline
    \end{tabular}
    \caption{更改字体的声明}
    \label{tab:2.1}
\end{table}

\begin{codelist}[2.1]{
    \texttt{char}类型的\emph{变量}\textsl{总是}被编码为\textbf{8 位}。
}
\begin{verbatim}
\texttt{char}类型的\emph{变量}\textsl{总是}被编码为\textbf{8 位}。
\end{verbatim}
\end{codelist}

注意,在上面的命令中,指令\verb|\emph|(对应的声明是\verb|\em|,可以以更优雅的方式突出显示一组词)。相较声明,强烈建议使用\emph{指令}。当要修改文本的一部分时,使用指令是更明智的选择\jz{定义指令也是。}:

\begin{codelist}[2.2]{
    {\em \bfseries 马格马(Magma) \mdseries 的音乐就像一面镜子,每个人都能看到他自己的倒影。}
}
    \begin{verbatim}
{\em \bfseries 马格马(Magma) \mdseries 的音乐
就像一面镜子,每个人都能看到他自己的倒影。}
    \end{verbatim}
\end{codelist}

接下来的例子展示了如何使用组。声明\verb|\slshape|出现在一个组中,因此它只在组内发挥作用。此外,组会继承它外层的组的参数。这样一来,“silence”一词会使用\textsf{非衬线}体(根据外层的组),并且\textsl{倾斜}展示(根据内层的声明):

\begin{codelist}[2.3]{
    \sffamily 在爵士乐中,{\slshape 沉默(silence)\/}永远是正确的。因此,这是一种充满万千可能的音乐。
}\begin{verbatim}\sffamily 在爵士乐中,
{\slshape 沉默(silence)\/}永远是正确的。因此,
这是一种充满万千可能的音乐。
\end{verbatim}
\end{codelist}

\subsection{意大利体校正}

另一个推荐使用指令而不是声明的理由是,与声明不同,指令可以实现\emph{意大利体校正}。所谓意大利体校正,是指在以意大利体显示的字符组后,有必要增加一个间距,使得这组字符不会“碰到”后面的词。这个间距与所涉及的字符有关\yz{此例展示不同情况下\emph{f}和a间距的细微差距,宜保留原文。例句意为:首长永远是对的。}:

\begin{codelist}[2.4]{
    le {\em chef} a toujours raison.\par
    le {\em chef\/} a toujours raison.\par
    le \emph{chef} a toujours raison.\par
}\begin{verbatim}le {\em chef} a toujours raison.\par
le {\em chef\/} a toujours raison.\par
le \emph{chef} a toujours raison.\par
\end{verbatim}
\end{codelist}

我们可以看到,指令\verb|\emph|实现了校正,然而,若要使用声明,则需要明确地借助宏\verb|\/|来实现相同的效果。