\chapter*{序}

男人的邪恶胜过女人的善良\footnote{
    本书的章首引言来自《旧约》与《新约》,将它们引用在这里纯粹是我一手挑动的——有时,这些句子中带有一些与章标题相关的内容(译注:宗教相关内容按原文直译,不代表译者对任何宗教文献中任何语句的认可或否认。本书未标注“译注”字样的脚注均为原书脚注)。
}。——《便西拉智训\footnote{
    译注:原文如此,但作为《诗歌智慧书》一部分的《便西拉智训》(天主教译为《德勋篇》)似乎属于次经,即在一些教派中不被承认作为《圣经》的一部分出现。 
}》42:14

\section*{从前……}

一切始于1990年年初。我当时正在PC 286计算机上使用称为\textit{WordPerfect}的软件,以此入门人们所谓的“文字处理”。这款软件现在仍然存在,并且由Corel公司维护,运行在日后拥有响当当名头的MS-DOS中。MS-DOS集成了用以粗略预览文档的接口,尤其允许用户“看到代码”,也就是借助一种标记语言将文档可视化,以灵活地控制。

稍晚些时间,随着Windows 3.1迅速风靡,人们突如其来地追求图形界面,我虽然仍情有不甘,却逐渐说服了自己去使用那款在今天很出名的文字处理软件——的2.0版(后面还带个小小的字母,在当时那可真是重大的升级)……但我日后才知道,这个版本有个很有趣的“特性”:文件体积过大,超过了某个特定的值时,会出现保存失败的情况!这时,你既不能保存,也不能恢复文档。有些头铁的朋友尝试先删除几行再保存,但这种撞大运的解决方案并没能成功……

当时,大家毫不掩饰地嘲讽这些“你懂的”公司制作的软件\footnote{
    这些被嘲讽的对象中,我们可以看到一些名场面:通用汽车公司老板对比尔·盖茨挑衅性言论的回应(译注:可能是指比尔·盖茨的观点,即如果汽车工业能够像计算机领域一样发展,那么一辆汽车只需要25美元就能买到,并且消耗1加仑汽油就能跑1000英里。作为回应,通用汽车方面罗列了一系列言论来嘲讽,例如“如果那样,那么想要汽车熄火,需要点击开始菜单”),以及罗伯托·迪·科斯莫(Roberto Di Cosmo)的“赛博空间中的陷阱”(piège dans le cyberespace)。
}——这里就不点名了。我周围的大多数人躺平地选择了接受,认为使用这些堂而皇之不给出警告的可悲的跟风之流是正常现象。软件的这种“特性”坚定了我的信念:\textit{我绝不使用这种软件}。当时还在攻读工程师学位的我意识到,我今后的部分工作将会集中在起草文档和使用通用的信息系统上。为此,我需要足够健壮的工具。

我是在让·莫奈大学(Université Jean Monnet)和圣-埃蒂安高等矿业学校(École des Mines de Saint-Étienne)攻读DEA(现在叫master recherche)\footnote{
    译注:DEA即diplôme d'études approfondies,法国教育体系下的一种学位。
}时相继接触UNIX和Linux的。那时(1993~1994年),在我刚写论文的开头时,“拉泰克”(latèque)这个词就开始围着我转。这里问题似乎是要找到一款能排出数学公式的软件,而说到撰写理科文档,\LaTeX 似乎显然是避不开的\textit{唯一}答案。说实话,找软件这种问题甚至都根本没出现过!

于是,我着手把这个叫做\LaTeX 的“玩意儿”装在Mac系统(安装的发行版叫Oz\TeX )和另一个由古登堡(Gutenberg)协会支持的发行版系统——Solaris上。为此,我还得去收买一个系统管理员,让他同意创建一个特权用户\texttt{texadm},用来管理那个发行版……

1994年年初,我带着坚定的意志使用\LaTeX 开始写论文。在1995年,在被我发现的种种技巧激起的兴趣的巨大感召下,% 此句翻译不好:enthou- siasmé par ce que je découvrais, TODO
我着手为同事和实验室起草用于入门\LaTeX 的指导手册。这个手册就是本书的原型。在1997年,在练习了两年并一只脚踏入了排版领域后,我更坚定了自己的看法:\LaTeX 绝对是写严肃文件的首选软件:它有对版面(mise en page)的全面控制,有对参考文献的管理,支持索引(通用名称和作者名),能轻松操作文件。最重要的是,\textit{排版的结果很好看}。从那时起,这就是支撑我使用\LaTeX 的最强大而无可争辩的理由。

今天,作为国立圣-埃蒂安工程师学院(École Nationale d’ingénieurs de Saint-Étienne)的计算机高级讲师,我用\LaTeX 来起草理科文档和教学材料。几年使用下来,我仍然在学习和发现,也仍然会对项目贡献者提出的各种扩展啧啧称奇。这些扩展使\LaTeX 成为了充满宝藏的巴扎(bazar),成为了一款名副其实地朝着更高工效发展的\footnote{
    并不是指那些诸如在菜单中添加一个功能入口、在弹出对话框时添加个提示音的“提效”。
}、始终以“产出优美的工作成果”为目标的卓越而独特的工具。

\section*{本书结构}

本书是针对“使用\LaTeX 进行文字处理”的介绍。它不是一本参考手册,但本书的写作目标是传授读者使用\LaTeX 的基本知识,并在可能情况下,让读者对它感兴趣。读者可以在本书中找到\textit{开始}使用\LaTeX 的必要信息和起草文档的建议。为了提升阅读体验,我们“高明地”将本书