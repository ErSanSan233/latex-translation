\chapter*{序}

\begin{quote}
    男人的邪恶胜过女人的善良\jz{
    本书的章首引言来自《旧约》与《新约》,将它们引用在这里纯粹是我一手挑动的——有时,这些句子中带有一些与章标题相关的内容(译注:宗教相关内容按原文直译,不代表译者对任何宗教文献中任何语句的认可或否认。本书未标注“译注”字样的脚注均为原书脚注)。
}。——《便西拉智训\yz{原文如此,但作为《诗歌智慧书》一部分的《便西拉智训》(天主教译为《德训篇》)似乎属于次经,即在一些教派中不被承认作为《圣经》的一部分出现。 
}》42:14
\end{quote}

\section*{从前……}

一切始于1990年年初。我当时正在PC 286计算机上使用称为\emph{WordPerfect}的软件,以此入门人们所谓的“文字处理”。这款软件现在仍然存在,并且由Corel公司维护,运行在日后拥有响当当名头的MS-DOS中。MS-DOS集成了用以粗略预览文档的接口,尤其允许用户“看到代码”,也就是借助一种标记语言将文档可视化,以灵活地控制。

稍晚些时间,随着Windows 3.1迅速风靡,人们突如其来地追求图形界面,我虽然仍情有不甘,却逐渐说服了自己去使用那款在今天很出名的文字处理软件——的2.0版(后面还带个小小的字母,在当时那可真是重大的升级)……但我日后才知道,这个版本有个很有趣的“特性”:文件体积过大,超过了某个特定的值时,会出现保存失败的情况!这时,你既不能保存,也不能恢复文档。有些头铁的朋友尝试先删除几行再保存,但这种撞大运的解决方案并没能成功……

当时,大家毫不掩饰地嘲讽这些“你懂的”公司制作的软件\jz{
    这些被嘲讽的对象中,我们可以看到一些名场面:通用汽车公司老板对比尔·盖茨挑衅性言论的回应(译注:可能是指比尔·盖茨的观点,即如果汽车工业能够像计算机领域一样发展,那么一辆汽车只需要25美元就能买到,并且消耗1加仑汽油就能跑1000英里。作为回应,通用汽车方面罗列了一系列言论来嘲讽,例如“如果那样,那么想要汽车熄火,需要点击开始菜单”),以及罗伯托·迪·科斯莫(Roberto Di Cosmo)的“赛博空间中的陷阱”(piège dans le cyberespace)。
}——这里就不点名了。我周围的大多数人躺平地选择了接受,认为使用这些堂而皇之不给出警告的可悲的跟风之流是正常现象。软件的这种“特性”坚定了我的信念:\emph{我绝不使用这种软件}。当时还在攻读工程师学位的我意识到,我今后的部分工作将会集中在起草文档和使用通用的信息系统上。为此,我需要足够健壮的工具。

我是在让·莫奈大学(Université Jean Monnet)和圣-埃蒂安高等矿业学校(École des Mines de Saint-Étienne)攻读DEA(现在叫master recherche)\yz{
    DEA即diplôme d'études approfondies,法国教育体系下的一种学位。
}时相继接触UNIX和Linux的。那时(1993~1994年),在我刚写论文的开头时,“拉泰克”(latèque)这个词就开始围着我转。这里问题似乎是要找到一款能排出数学公式的软件,而说到撰写理科文档,\LaTeX 似乎显然是避不开的\emph{唯一}答案。说实话,找软件这种问题甚至都根本没出现过!

于是,我着手把这个叫做\LaTeX 的“玩意儿”装在Mac系统(安装的发行版叫Oz\TeX )和另一个由古登堡(Gutenberg)协会支持的发行版系统——Solaris上。为此,我还得去收买一个系统管理员,让他同意创建一个特权用户\texttt{texadm},用来管理那个发行版……

1994年年初,我带着坚定的意志使用\LaTeX 开始写论文。在1995年,在被我发现的种种技巧激起的兴趣的巨大感召下,% 此句翻译不好:enthou- siasmé par ce que je découvrais, TODO
我着手为同事和实验室起草用于入门\LaTeX 的指导手册。这个手册就是本书的原型。在1997年,在练习了两年并一只脚踏入了排版领域后,我更坚定了自己的看法:\LaTeX 绝对是写严肃文件的首选软件:它有对版面(mise en page)的全面控制,有对参考文献的管理,支持索引(通用名称和作者名),能轻松操作文件。最重要的是,\emph{排版的结果很好看}。从那时起,这就是支撑我使用\LaTeX 的最强大而无可争辩的理由。

今天,作为国立圣-埃蒂安工程师学院(École Nationale d’ingénieurs de Saint-Étienne)的计算机高级讲师,我用\LaTeX 来起草理科文档和教学材料。几年使用下来,我仍然在学习和发现,也仍然会对项目贡献者提出的各种扩展啧啧称奇。这些扩展使\LaTeX 成为了充满宝藏的巴扎(bazar),成为了一款名副其实地朝着更高工效发展的\jz{
    并不是指那些诸如在菜单中添加一个功能入口、在弹出对话框时添加个提示音的“提效”。
}、始终以“产出优美的工作成果”为目标的卓越而独特的工具。

\section*{本书结构}

本书是针对“使用\LaTeX 进行文字处理”的介绍。它不是一本参考手册,但本书的写作目标是传授读者使用\LaTeX 的基本知识,并在可能情况下,让读者对它感兴趣。读者可以在本书中找到\emph{开始}使用\LaTeX 的必要信息和起草文档的建议。为了提升阅读体验,我们“高明地”将本书分为了若干章节,并配有附录。
本书首先介绍\LaTeX 的基础知识:

\paragraph*{基本原则}展示\LaTeX 的基本原理。为了读懂本书的剩余部分,需要阅读本章。

\paragraph*{需要知道的知识}展示标准工具。为了起草一篇简单的文档,需要了解这些知识。

% todo

然后有如下附录:

% todo

我们建议您先从第1章一路读到数学部分。其余的章节相对独立,可以根据需要阅读。再强调一遍,我们建议在熟练掌握了基础概念之后再去阅读本书的第II部分。文档最后的索引提供了查询所需内容的快捷入口。最后,正如同其他关于\LaTeX 的答疑解惑的法文资料,我没有费神地将所有\LaTeX 术语和计算机术语逐一翻译。

\section*{你需要知道的知识}

本书适用于初学者阅读,不要求读者有关于\LaTeX 的任何知识。然而,本书读者应当具有基本的、有关操作系统和计算机用户的知识。本书读者最好懂得如何从使用绘图软件或图片处理软件开始,创建一个封装在其计算机系统中的PostScript文件。

\section*{你不会通过本书学习到的知识}

你正阅读的这本图书在令人称赞的同时也有以下知识面漏洞。

\begin{itemize}
    \item 本书不含有关于\TeX 或\LaTeX 生成字体原理的清晰解释。你不会找到关于“元字体”\linebreak (METAFONT)一词的知识。
    \item 你不会找到关于在UNIX系统下安装\LaTeX 发布版的知识。
    \item 你不会找到任何现有扩展包的“目录”或清单,无论扩展包是否实用、是否兼容。
    \item 本书回避了“先有鸡还是先有蛋”之类的问题,也避免讨论关于上帝和科学的问题。
    \item ……
\end{itemize}

\begin{exclamation}
不要对本书的内容抱有不切实际的幻想:本书书名着实是个不要脸的谎言。
\end{exclamation}

\section*{\TeX 是什么?}

唐纳德·欧文·克努特(Donald Ervin Knuth)——就是那个有着众多关于数学和算法的著作[包括《计算机程序设计的艺术》(英:\emph{The Art of Computer Programing})]%todo ref
的数学家——对20世纪70年代的技术条件下打印出来的文章的样子深感失望,产生了开发称为\TeX 的文字处理系统的初步想法。20世纪80年代初次公布的\TeX 是由一个宏处理器(processeur de macro ;英:macro processsor)和几个基元(primitive)组成的复杂系统。第一组预编译的宏很快以\emph{“普通格式”(format plain)}的名义出现。

注意,\TeX 既不是文字处理器[克努特将其称为“typesetting system”,可以翻译成“排字系统”(système de composition)]也不是一种编译后的编程语言。这是克努特关于\TeX 的一些说明\jz{出自\TeX book的“The Name of the Game”一章。}:

\begin{quote}
    “英文的‘technology’一词由希腊文词根‘$ \tau\epsilon\chi...$’演变而来,这个词根有时也指艺术和科学技术。\TeX 由此而来,正是$ \tau\epsilon\chi$的大写形式。”
\end{quote}

关于\TeX 中“X”的发音:

\begin{quote}
    “……它的发音像德语单词ach中的‘ch’,或西班牙语中的‘j’……如果你对着电脑正确地发音,屏幕上会出现哈气。”
\end{quote}

你谦逊的仆人可能会更想让你读成“TeK”,来避开那种有气无力的感觉和没过几天就要给你擦一次电脑的麻烦工作。

最后,对于\TeX 的标识设计,克努特强调字母E需要稍微错位一些,以提示人们这是关于排版的工具。对于确实会遇到的一些无法使字母E稍微错位的情况,他坚持道,需要将\TeX 写成“\texttt{TeX}”。

目前,\TeX 的最新版本号是3.1415926(没错,它收敛于$\pi$)。在\emph{\TeX : the program}一书的前言中,克努特估测上一个程序漏洞已于1985年11月27日发现并改正,并出价20.48美元来悬赏下一个漏洞。今天,这个十六进制的金额停留在327.68美元,如果有人喜欢2的幂,这个数字应该会让他满意……

\section*{\LaTeX 是什么?}

1985年,\TeX 已经传播了一段时间,莱斯利·兰波特(Leslie Lamport)将宏组合起来,创造了一个视野更广的格式,称为\LaTeX ,版本号为2.09。今天,\LaTeX 已经成为了事实标准,只有一些玍古的情况才会只支持\TeX 而不支持\LaTeX 。然而,\LaTeX 有点像\TeX 的“镀层”,提供\TeX 的宏的调用。有时,掌握\TeX 中的部分概念有助于从困难的处境中脱身。兰波特在他的书中这样说[10]:%TODO cf.

\begin{quote}
    “可以将\LaTeX 想象成一幢房子,它的构架和钉子就是由\TeX 提供的。如果你只是在房子中生活,那么你不需要准备钉子、搭建构架,但如果想要为房子新增一个房间,那么你就会需要它们。”
\end{quote}

他还说道:

\begin{quote}
    “\LaTeX 的出名是因为它允许作者从排版工作中抽离,并且专注在写作上。如果你在形式上花费了太多时间,那么你并没有很好地使用\LaTeX 。”
\end{quote}

从1994年至今,一个由欧美成员组成的团队[以弗朗克·米特尔巴赫(Frank Mittelbach)为核心]着手\LaTeX 的开发。1994年发布的\LaTeX 版本称作\LaTeXe。团队的长期目标是孵化一个名为\LaTeX 3的系统。

\section*{使用许可}

画重点:\TeX 和\LaTeX 属于自由软件——也因此是免费的。同时,自由软件(logiciel libre;英:free software)的标志是其\emph{开放性}。因此,\TeX 也可以有其Web源码\jz{克努特孕育的Web语言被形容为一种“文学性的编程语言”。使用Web源码,可以生成程序的Pascal或C代码,也可以为代码生成\TeX 文档。}。\LaTeX 的宏是以\TeX 源码的格式发布的。对于大部分用户来说,获取程序的源码可能不是首要考虑的,但需要知道,正是这种\emph{不隐藏任何内容}的性质,使得人们可以改进现有的扩展、创造新的扩展。

一款软件是自由软件,并不意味着我们可以使用它做任何想做的事情。自由软件属于其\linebreak 作者,所有的改动都需要被记录。同样,每次改动都需要以与具有与改动前不同的\linebreak 文件名体现。这样可以保证系统的严密和便携(关于 \LaTeXe  的使用许可,请参阅\linebreak \wz{ftp://ftp.lip6.fr/pub/TeX/CTAN/macros/latex/base/lppl.txt})。

\section*{不使用\LaTeX 的五个理由}

在一些情况下,强烈建议不使用\LaTeX 。具体来说,这些不使用\LaTeX 的理由如下。

\begin{enumerate}
    \item 你只将文字处理器用于制作贺卡、写邮件、记录几个想法等用途。
    \item 你十分喜欢鼠标(可能具有1~3个按键),并且认为输入方程的唯一方式就是频繁地使用鼠标点来点去。
    \item 你觉得UNIX是一个“让人头痛”且“不易使用”的系统,或者你对所有的编程语言都有着强烈的反感。
    \item 你认为以下情况是正常的:
        \begin{enumerate}
            \item 新版软件不能读取其旧版本创建的文档;
            \item 要使用新版软件,必须换一个操作系统;
            \item 要使用新版操作系统,必须换一台计算机;
            \item 要使用新计算机,必须……
        \end{enumerate}
    \item 你不知道键盘上的“\backslash”键在哪里。
\end{enumerate}

如果你的情况满足以上任何一条,最好在你现在的系统上知足常乐。

\section*{使用\LaTeX 的若干理由}

说服本书读者使用\TeX 和\LaTeX 而不是其他系统似乎不成问题——毕竟,你都读这本书了,也就已经不知不觉被说服了。让我们看看\TeX 的设计者是怎么说的:

\begin{quote}
    “在使用\TeX 起草文档时,你就是在指挥计算机如何准确地把你的稿件转化为几个页面,以媲美世界上最好的打印机能够实现的排版样式。”——D.E. 克努特,\TeX book[9]
\end{quote}

\TeX 和\LaTeX 可以生成无与伦比的文档(并可以极细微地调整\jz{
    作为参考,\LaTeX 内置的衡量单位是\emph{比例点(英:scaled point)},在\TeX book中记作\texttt{sp},合$1/65536$点;1点合约$1/72$英寸;1英寸合2.54厘米。比例点可以在大约50埃米的尺度上调整文档。目前打印机的分辨率对于这个尺度来说,实在是太充裕了。
}),这显然归功于它们的以下能力:

\begin{itemize}
    \item 仔细地绘制字体;
    \item 处理排版上的细节,包括连接号(tiret)和合字,比如你可能没有观察过的以下情况:
    \begin{itemize}
        \item “avez-vous --- bien --- regardé ces tirets (page 19--23) ”这句文字中的各种连接号;
        \item fin一词中的“f\/i”、souffle一词中的“f\/f\/l”,以及trèfle一词中的“f\/l”;
    \end{itemize}
    \item 性能良好的断字算法;
    \item 专门针对数学公式的呈现。
\end{itemize}

此外,\LaTeX 是少数瞄准\emph{科技}文档的文字处理软件。这是因为,除了处理方程和公式之外外,\LaTeX 还有大量围绕起草\emph{文章}、生成\emph{参考文献}和\emph{索引}的功能。

最后,\LaTeX 尤其针对大文件的生成做了适配。这不仅是由于处理\LaTeX 文档本身占用的内存空间极小,也是因为\emph{宏}和\emph{交叉引用(référence croisée;英:cross reference)}可以让我们对文件有着全面而灵活的控制。

\paragraph*{交叉引用}\LaTeX 允许\emph{以符号的形式}于文档的任何位置引用有编号的对象。此外,标题、图片、表格、方程、参考文献、列表、定理等的序号都可以在文章的多个位置以简单的方式引用,不需要我们去关心具体的号码本身是多少。

\paragraph*{宏}宏无疑是\LaTeX 最强大的功能。要知道,生成文档的\textbf{所有}过程对视都是一系列指令或宏。因此,每个用户都可以文件中的宏来改变文件的生成情况。显然,我们可以很好地定义我们自己的宏,使得文档的一部分呈现出特殊的效果。围绕宏的一个很强烈的观点是,我们\emph{原则上}可以将设置格式的部分从起草文档的过程中分离出去。

\section*{所见即所得的缺陷}

\LaTeX 处于“所见即所得”\jz{
    英:What you see is what you get,简称为Wysiwyg。指软件允许用户在屏幕上看到即将在纸上获得的结果完全相同的内容。第一款“所见即所得”的文本处理器大概是\textsf{Bravo},它于1974年出现在施乐帕罗奥多研究中心(英:Xerox Palo Alto Research Center)的机器Alto上。
}的对立面,因为\LaTeX 源码是包含文本内容本身和页面布局命令的文本文档。兰波特将这种方法成为\emph{逻辑}页面布局而不是\emph{视觉}页面布局\jz{
    说点不好听的,根据兰伯特在其关于\LaTeX 的书中的描述,“所见即所得”的软件被柯尼汉[译注:Kernighan,可能指UNIX开发者布莱恩·W.克尼汉(Brian W. Kernighan)]描述为“只能看到已经有了的东西”。
}。

然而,我们也可以说\LaTeX 是“所见即所得”的,因为在编译后,我们就可以在屏幕上呈现出文档未来出现在纸面上的\textbf{精确}形态。

以下是一个能够说明“所见即所得”缺陷和逻辑页面布局优势的案例\jz{
    兰波特在它的手册中提到了一个类似的案例。
}:假设在一个文档中,某个具有两个参数的函数出现了一定次数。科学文档的一个精巧之处就是可以使用符号,我们可以定义一个宏\texttt{\backslash mafct}来生成这个函数。如此一来,\texttt{\backslash mafct\{1\}\{2.5\}} 和 \texttt{\backslash mafct\{x\}\{t\}}会分别生成$\mathcal{F}_{\alpha, \beta}(1, 2.5)$和$\mathcal{F}_{\alpha, \beta}(x, t)$。一旦我们想要改变符号,只需重新定义\texttt{\backslash mafct}这个宏,以在需要的位置重新生成对应的符号(比如分别生成$\mathsf{F}^{\alpha, \beta}[1, 2.5]$和$\mathsf{F}^{\alpha, \beta}[x, t]$),就可以了!

另一个例子是:假设你的文件中有很多科技词汇,你想要用一种特殊的形式展示她们。\linebreak 因此你事先定义了宏\texttt{\backslash jargon}\yz{
    jargon意为“术语”。
},以将科技词汇设置为意大利体,并在在文档中写下\linebreak \texttt{\backslash jargon\{implémentation\}}之类的实现。你的文档中以这种方式提到了235个术语词汇。你如果改变了主意,想把它们从意大利体改成其他的格式,那么只需要重新定义宏\texttt{\backslash jargon},而不需要逐个排查那235处术语词汇。经过一些练习以后,你甚至能使这个宏自动将术语插入文档的索引中……

以下是一个略微变形的例子:在稍前的名为“不……的五个理由”的章节标题中,我在源码中完全没有写“五\jz{
    即使在这里也没有写。
}”这个字。标题是使用“……的\texttt{\backslash ref\{nbraisons\}}个理由”这样的句子写成的,它会自动将文中提到的不使用\LaTeX 的理由的数量替换为对应的法文词汇。这样,如果还想在列表中再插入一条理由,我就不用重新统计一遍数量了\yz{
    此例特指原版书。翻译时舍弃了原书源码中类似的自动化部分。
}。

\begin{exclamation}
    阅读本书的过程中,会有其他案例会为你逐一呈现“所见即所得”的缺陷。这个“说明”(nota)段落为你指出了一些重要的知识点,同时也是另一个案例。理由是:在作者输入这行文字的时候,展示“警示牌”般的版式是细枝末节的问题,但它只是一个nota,作者是这样写的:

    \begin{dmd}
\verb+\begin{nota}+\\
  在阅读本书的过程中……\\
\verb+\end{nota}+
    \end{dmd}
\end{exclamation}

作为对宏的总结,我们可以说,这是微软公司的著名软件——Word中样式的推广。阅读本文档,尤其是它的第II部分%TODO合适
,足以使你信服:宏可以比这些脍炙人口的样式走得更远……

对于那些对“所见即所得”模式上瘾的人, 有团队发布了一个\emph{“所见即所表”(原文如此;What you see is what you Mean (sic))}版本的\LaTeX ,命名为LyX。你可以访问\wz{http://www.lyx.org}来了解它。

\section*{如何打印本书}

准备打印机\jz{
    啊——啊——[就像弗兰克·扎帕(Frank Zappa)说的那样](译注:有人知道这是那首歌吗?)
},使用源码文档生成的“papier”版本文件,可以获得适用于A5型号纸张的打印预览。

\section*{你可以用本书做什么?}

\paragraph*{作者}Vincent Lozano
\paragraph*{原书名}Tout ce que vous avez toujours voulu savoir sur \LaTeX \ sans jamais avoir osé le demander
\paragraph*{日期}2013年11月22日
\paragraph*{版权开放(Copyleft)}本书是开放图书,遵循开放作品许可(Licence Art Libre,LAL):

\wz{\centerline{http://www.artlibre.org}
}

总之,LAL规定你可以复制本书,你同样可以在遵循以下条款的前提下传播本书:

\begin{itemize}
    \item 注明其遵循LAL;
    \item 注明原作者名Vincent Lozano及对本书做出修改的人名;
    \item 注明其源码可以通过\wz{http://cours.enise.fr/info/latex}下载。
\end{itemize}

最后,你可以在满足以下条件的情况下修改本书:

\begin{itemize}
    \item 满足以上传播协议;
    \item 注明你的作品是修改过的版本,以及如果可能,注明修改的内容;
    \item 以相同的使用协议或与本书的使用协议不冲突的协议规定下传播。
\end{itemize}

\section*{开始之前}

就像很多非常强大的软件一样,\LaTeX 的使用从来就不简单。实际上,当我们在自己的方向上前进时,\LaTeX 经常让人感觉很舒适,我们可以借助它来避免过于纠结版式问题,就像兰波特所说的那样。如果我们想要改变行为时的解决方案仍然是选择另一条指令,那么一切都会顺利进行。然而,尽管\LaTeX 给出的选择代表了优秀出版人采用的现行通用标准,我们仍然有一天会想要排出某种特殊的版式,而\LaTeX 表面上做不到这一点。这种情况下,有几种解决方案供你选择。
\begin{itemize}
    \item 包含一个可以解决你的问题的\emph{包}(\LaTeX 是一个开放系统,有大量或多或少标准化的包可供我们去实现多种多样、稀奇古怪的操作)。
    \item 找一个\TeX 学家(\TeX nician\jz{也有人喜欢称作\TeX pert,但很少见。})来帮你排除问题。
    \item 如果前两个方案对你来说没有用,就不要在代码中埋头\jz{对于写起代码来“文思如尿崩”(pisser du code)的人来说,这是最让人开心的解决方案了。}探寻蛛丝马迹、查找出错的指令并修改了。此刻你需要的是去了解这个系统的的第一层,去了解\TeX 。这里,我们遇到了\LaTeX 的一个缺点:如果说其他软件不能做到的都是很复杂的事情,那么有时让\LaTeX 做一些简单的事情也是很困难的(在阅读过本书第II部分后%TODO
    ,你可能会同意这一点)。
\end{itemize}

\section*{排版上的约定}

为了让呈现效果更清晰,本书遵循了一些排版约定。文档中散布的\LaTeX 代码的片段看起来是这样的格式:

\begin{dmd}
\textsl{\%注意看}\\
\verb+这样的\emph{就}是\LaTeX 代码了。+
\end{dmd}

相关内容会使用\LaTeX 的“\texttt{打字机}”(\texttt{machine à écrire})字体显示。代码也会使用如下的形式展示,中间竖条上的数字偶尔会被引用\yz{这个格式我不一定能调出来,暂时用清单编号代替了}:%TODO

\begin{codelist}[0.1]{
    这样的\emph{就}是\LaTeX 代码了。
}\begin{verbatim}
%注意看
这样的\emph{就}是\LaTeX 代码了。
\end{verbatim}
\end{codelist}

\begin{ii}
一些内容会以“补充说明”的形式给出,这是为了强调一个知识点。读者不需要第一时间阅读。
\end{ii}

\begin{exclamation}
对于请读者务必阅读的内容,我们会使用这种形式来引起注意……
\end{exclamation}

\textsf{软件名}或\LaTeX 中的\textsf{包名}会以本句所展示的形式展示。英文词汇会以\emph{这种形式(英:like this)}展示。为了展示指令中通用的部分,我们会使用\codereplace{这种形式}。例如:

\begin{dmd}
\verb|这是\LaTeX 源码中的\emph{|\codereplace{强调文字}\verb|}。|
\end{dmd}

偶尔出现的UNIX指令会以如下形式展示:

\dmh{grep -wi bidule /tmp/truc.dat | sort -n}

在其中一个附录中,\textsf{emac}指令会以如下形式展示:

\dmh{\textsf{M-x doctor}}

最后,作为让人反感的精华,\dm{Makefile}片段会以如下形式展示:

\begin{mdframed}
    \verb|bidule : bidule.o truc.o|\\
    $\longmapsto$ \verb|gcc -o $@ $^|
\end{mdframed}

\section*{致谢}