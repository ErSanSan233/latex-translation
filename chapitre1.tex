\part{关于\LaTeX 的那些你想知道却从不敢问的问题}

\chapter{基本原则}

人若身患漏症,他因这漏症就不洁净了。——《圣经·利未记》15:2

本章介绍\LaTeX 的基本原理。你将会看到关于\LaTeX 安装的简介、使用\LaTeX 的基本“流程”(session)介绍、文章格式的结构、使用变音符号的注意事项,认识几个工具,以及了解面对编译错误消息时的态度。

\section{安装}

你想安装\LaTeX 吗?你将要安装的是\LaTeX 的其中一个\textit{发行版},具体的版本取决于你的操作系统\jz{
    如果你不知道操作系统是什么东西,那么你使用的是macOS;如果你不知道你的计算机用的\textit{具体是哪个}操作系统,那么你在用Windows;否则,你在用UNIX……
}。发行版中带有可以自动安装和配置\LaTeX 、\TeX 和其他相关内容的程序。

\paragraph*{对于UNIX:}我们可以找到称为te\TeX 的发行版,虽然它的开发早在2006年就停止了。今天,我们一般安装\TeX Live(\wz{http://www.tug.org/texlive})。

\paragraph*{对于macOS:}建议安装的发行版是Mac\TeX(\wz{http://www.tug.org/mactex})。

\paragraph*{对于Windows:}最简单的方式无疑是选择pro\TeX t(\wz{http://www.tug.org/protext})。它会安装称为MiK\TeX 的发行版(\wz{http://www.miktex.org})和几个开发工具,其中包含一个查看PostScript文件的程序(\textsf{gsview})。

偶尔,需要在为发行版中搭配一款文字编辑器(如果其中没有包含),因为你很快就能看到,使用\LaTeX 就是在文件中输入文字和命令。

\begin{itemize}
    \item UNIX中,推荐使用\textsf{emacs}或\textsf{vi},即使前者明显比后者更高级,但二者用户之间无结果的恶意争吵仍在继续。
    \item \textsf{kile}和\textsf{texmaker}是已集成的开发环境。依靠它们,初学的用户在入门时会觉得更轻松。它们的特点是将编辑、编译和可视化集成在一个界面。这两个环境也使通过菜单、对话框或其他标签来探索\LaTeX 指令称为可能(如图\ref{fig:1.1}a所示)。
    \item Windows中的对应产品是\textsf{\TeX nicCenter}(如图1.1b所示)。
    \item macOS中的对应山品是\textsf{\TeX shop}和\textsf{i\TeX max}。
\end{itemize}

\begin{figure}[H]
    %TODO 图
    \centering
    Kile

    \TeX nicCenter
    \caption{集成的两个开发环境:Linux中的Kile和Windows中的\TeX nicCenter。它们将编辑、编译和可视化集成在一个界面中}
    \label{fig:1.1}
\end{figure}

你很快就会学到,用\LaTeX 制作文档是一个翻译(也称作\textit{编译})的过程——将编辑者创建的源文件转换为用于显示或印刷的格式\jz{本章会略微多介绍一些这个格式。}。因此,发行版中内置了或多或少的著名工具,可以将编译后的不同格式的文件显示出来。

\paragraph*{对于PDF格式:}除了著名的\textsf{acrobat reader},UNIX中还有一些可以显示PDF文件,如\textsf{xpdf}、\textsf{evince}等。

\paragraph*{对于DVI格式:}UNIX中的\textsf{xdvi}、\textsf{kdvi}和Windows中的\textsf{yap}都是可以显示这种\LaTeX 编译文件的程序。

\paragraph*{对于PostScript格式:}\textsf{ghostscript}套件(在各平台下的名称可能有差异)可以显示PostScript文件。

\begin{exclamation}
    需要注意,为了使你选用的发行版包含\LaTeX 的“法文”模式,以确保能够正确处理断字(césure;英:hyphenation),我们需要在编译文档是需要更改其“日志”(见1.6节)%带有引用
    以使法文模式加载:\\
    \texttt{LaTeX2e <2005/12/01>\\
    Babel <v3.8h> and hyphenation patterns for english, [...] dumylang, \fbox{french}, loaded.}
\end{exclamation}

\section{“生产”周期}

即使\LaTeX 并不是通常意义上说的编译型语言,但我们仍然可以将制作一个\LaTeX 文档的周期与使用一款经典的编程语言开发软件的\textit{编辑—编译—执行}周期进行类比。

\subsection{编辑}

一个\LaTeX \textit{源}文件是一个文本文件\jz{即文件仅由组成其中符号的代码构成。}。因此,对\LaTeX 文件的操作并不依赖于某个特定的软件,只需要一个经典的文本编辑器即可。因此,若要操作\LaTeX 文档:

\texttt{emacs }<文件名>\texttt{.tex &}

或

\texttt{vi }<文件名>\texttt{.tex}