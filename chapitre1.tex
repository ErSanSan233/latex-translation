\chapter{基本原则}

人若身患漏症,他因这漏症就不洁净了。——《圣经·利未记》15:2

本章介绍\LaTeX 的基本原理。你将会看到关于\LaTeX 安装的简介、使用\LaTeX 的基本“流程”(session)介绍、文章格式的结构、使用变音符号的注意事项,认识几个工具,以及了解面对编译错误消息时的态度。

\section{安装}

你想安装\LaTeX 吗?你将要安装的是\LaTeX 的其中一个\textit{发行版},具体的版本取决于你的操作系统\jz{
    如果你不知道操作系统是什么东西,那么你使用的是macOS;如果你不知道你的计算机用的\textit{具体是哪个}操作系统,那么你在用Windows;否则,你在用UNIX……
}。发行版中带有可以自动安装和配置\LaTeX 、\TeX 和其他相关内容的程序。

\paragraph*{对于UNIX:}我们可以找到称为te\TeX 的发行版,虽然它的开发早在2006年就停止了。今天,我们一般安装\TeX Live(\wz{http://www.tug.org/texlive})。

\paragraph*{对于macOS:}建议安装的发行版是Mac\TeX(\wz{http://www.tug.org/mactex})。

\paragraph*{对于Windows:}最简单的方式无疑是选择pro\TeX t(\wz{http://www.tug.org/protext})。它会安装称为MiK\TeX 的发行版(\wz{http://www.miktex.org})和几个开发工具,其中包含一个查看PostScript文件的程序(\textsf{gsview})。

偶尔,需要在为发行版中搭配一款文字编辑器(如果其中没有包含),因为你很快就能看到,使用\LaTeX 就是在文件中输入文字和命令。

\begin{itemize}
    \item UNIX中,推荐使用\textsf{emacs}或\textsf{vi},即使前者明显比后者更高级,但二者用户之间无结果的恶意争吵仍在继续。
    \item 
\end{itemize}