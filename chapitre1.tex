\part{关于\LaTeX 的那些你想知道却从不敢问的问题}

\chapter{基本原则}

人若身患漏症,他因这漏症就不洁净了。——《圣经·利未记》15:2

本章介绍\LaTeX 的基本原理。你将会看到关于\LaTeX 安装的简介、使用\LaTeX 的基本“流程”(session)介绍、文章格式的结构、使用变音符号的注意事项,认识几个工具,以及了解面对编译错误消息时的态度。

\section{安装}

你想安装\LaTeX 吗?你将要安装的是\LaTeX 的其中一个\textit{发行版},具体的版本取决于你的操作系统\jz{
    如果你不知道操作系统是什么东西,那么你使用的是macOS;如果你不知道你的计算机用的\textit{具体是哪个}操作系统,那么你在用Windows;否则,你在用UNIX……
}。发行版中带有可以自动安装和配置\LaTeX 、\TeX 和其他相关内容的程序。

\paragraph*{对于UNIX}我们可以找到称为te\TeX 的发行版,虽然它的开发早在2006年就停止了。今天,我们一般安装\TeX Live(\wz{http://www.tug.org/texlive})。

\paragraph*{对于macOS}建议安装的发行版是Mac\TeX(\wz{http://www.tug.org/mactex})。

\paragraph*{对于Windows}最简单的方式无疑是选择pro\TeX t(\wz{http://www.tug.org/protext})。它会安装称为MiK\TeX 的发行版(\wz{http://www.miktex.org})和几个开发工具,其中包含一个查看PostScript文件的程序(\textsf{gsview})。

偶尔,需要在为发行版中搭配一款文字编辑器(如果其中没有包含),因为你很快就能看到,使用\LaTeX 就是在文件中输入文字和命令。

\begin{itemize}
    \item UNIX中,推荐使用\textsf{emacs}或\textsf{vi},即使前者明显比后者更高级,但二者用户之间无结果的恶意争吵仍在继续。
    \item \textsf{kile}和\textsf{texmaker}是已集成的开发环境。依靠它们,初学的用户在入门时会觉得更轻松。它们的特点是将编辑、编译和可视化集成在一个界面。这两个环境也使通过菜单、对话框或其他标签来探索\LaTeX 指令称为可能(如图\ref{fig:1.1}a所示)。
    \item Windows中的对应产品是\textsf{\TeX nicCenter}(如图1.1b所示)。
    \item macOS中的对应山品是\textsf{\TeX shop}和\textsf{i\TeX max}。
\end{itemize}

\begin{figure}[H]
    %TODO 图
    \centering
    Kile

    \TeX nicCenter
    \caption{集成的两个开发环境:Linux中的Kile和Windows中的\TeX nicCenter。它们将编辑、编译和可视化集成在一个界面中}
    \label{fig:1.1}
\end{figure}

你很快就会学到,用\LaTeX 制作文档是一个翻译(也称作\textit{编译})的过程——将编辑者创建的源文件转换为用于显示或印刷的格式\jz{本章会略微多介绍一些这个格式。}。因此,发行版中内置了或多或少的著名工具,可以将编译后的不同格式的文件显示出来。

\paragraph*{对于PDF格式}除了著名的\textsf{acrobat reader},UNIX中还有一些可以显示PDF文件,如\textsf{xpdf}、\textsf{evince}等。

\paragraph*{对于DVI格式}UNIX中的\textsf{xdvi}、\textsf{kdvi}和Windows中的\textsf{yap}都是可以显示这种\LaTeX 编译文件的程序。

\paragraph*{对于PostScript格式}\textsf{ghostscript}套件(在各平台下的名称可能有差异)可以显示PostScript文件。

\begin{exclamation}
    需要注意,为了使你选用的发行版包含\LaTeX 的“法文”模式,以确保能够正确处理断字(césure;英:hyphenation),我们需要在编译文档是需要更改其“日志”(见1.6节)%带有引用
    以使法文模式加载:

    \begin{dmd}
    LaTeX2e <2005/12/01>\\
    Babel <v3.8h> and hyphenation patterns for english, [...] dumylang, \fbox{french}, loaded.
    \end{dmd}
\end{exclamation}

\section{“生产”周期}

即使\LaTeX 并不是通常意义上说的编译型语言,但我们仍然可以将制作一个\LaTeX 文档的周期与使用一款经典的编程语言开发软件的\textit{编辑—编译—执行}周期进行类比。

\subsection{编辑}

一个\LaTeX \textit{源}文件是一个文本文件\jz{即文件仅由组成其中符号的代码构成。}。因此,对\LaTeX 文件的操作并不依赖于某个特定的软件,只需要一个经典的文本编辑器即可。因此,若要操作\LaTeX 文档,指令

\dmh{emacs \textsl{\<文件名\>}.tex \&} %TODO <>

或

\dmh{vi \textsl{\<文件名\>}.tex}

足以让你进入\LaTeX 文档这个充满野性和未知的世界。在Windows中,根据自己的喜好,我们可以选用一款文字编辑器。注意,对于\LaTeX 源文件,推荐使用\texttt{.tex}扩展名名。

\subsection{编译}

我们用如下指令开始编译:

\dmh{pdflatex \textsl{\<文件名\>}.tex}

早晚有一天,你会看到编译会产出错误。这将是1.6节会处理的问题。总之,解决了编译问题后,我们会得到一个带有\texttt{.pdf}扩展名的文件,它代表\textit{便携文件格式(英:portable document format)},这是一种由Adobe公司创造的著名格式。

\begin{exclamation}
    历史上,编译\LaTeX 源文件会生成\texttt{dvi}文件,代表\textit{设备无关(英:device independant)}。此类文件独不受输出环境(如屏幕、打印机等)的影响。这是一种包含了“图像”的\LaTeX 便携二进制文件,可以用于各种操作系统。随后,出现了一批用途各异的程序:
    \begin{itemize}
        \item 用于显示文档,即\texttt{.dvi}\rightarrow 点阵屏幕;
        \item 用于打印,即\texttt{.dvi}\rightarrow 打印机语言;
        \item 用于转换格式,即\texttt{.dvi}\rightarrow PostScript文件。
    \end{itemize}
\end{exclamation}

图\ref{fig:1.2}表明了UNIX生成最终文件过程中参与流程的多种程序。

\begin{exclamation}
    除了使用pdflatex外,也可以使用其他“编译器”来生成PDF文件。例如,xelatex和lualatex可以能正确地处理以UTF-8编码的文件,是常用的替代选项。
\end{exclamation}

\begin{figure}[H]
    %TODO 图
    \centering

    \caption{UNIX中参与生成过程的工具}
    \label{fig:1.2}
\end{figure}

\subsection{显示}%visualisation有时翻译成可视化,有时翻译成显示,这个可以后期再统一一下。

在编译后,可以简单地使用\textsf{evince}程序来完成显示步骤。输入以下指令:

\dmh{evince \textsl{\<文件名\>}.pdf \&}

这是一个\textsf{linux}下运行的十分直观的程序,能够给出一个方便阅读的文件预览。

\begin{exclamation}
    注意,不必在每次编译后都重新运行evince,它显示的内容会自动刷新。
\end{exclamation}

\subsection{打印}

对于\texttt{pdf}格式,如何打印它这一问题就丢给了你的操作系统。关于这一点,没有特殊的注意事项。你有了一个文件,可以自由地处置它,无论是直接打印,还是根据你所处的环境来发挥才艺。

\begin{exclamation}
    从\texttt{dvi}到\texttt{ps}格式的转换需要调用dvips程序:

    \dmh{dvips \textsl{\<文件名\>}.dvi}

    这可以生成一个PostScrpt格式的文件。这个格式也由Adobe创造,是一种打印机语言,可以看作\texttt{pdf}的祖先。目前的打印机出厂即可识别这种打印机语言。我们可以说,文件发送到打印机时,十有八九传送的是PostScrpt格式的参数。对于PostScript格式的文件,有大量可以显示、修改这种文件的工具。
\end{exclamation}

\section{源文件的结构}

本节将介绍一种文档类型。实际上,所有\LaTeX 文档都具有相同的结构,形式如下:

\begin{dmd}
\backslash documentclass[\textsl{\<类选项1\>},\textsl{\<类选项2\>},...]\{\textsl{\<类\>}\}\\
\backslash usepackage[\textsl{\<包选项1\>},\textsl{\<包选项2\>},...]\{\textsl{\<包\>}\}\\
...\\
\textsl{\<文前部分\>}\\
...\\
\backslash begin\{document\}\\
...\\
\textsl{\<文本\>}\\
...\\
\backslash end\{document\}
\end{dmd}

如此一来,所有的\LaTeX 文档都可以按以下方式拆解。

\begin{itemize}
    \item 说明文档的\textsl{\<类\>};
    \item 文前部分,包含以下内容:
        \begin{itemize}
            \item 使用特定的\textsl{\<包\>};
            \item 多样的初始化和声明;
        \end{itemize}
    \item 文档主体,即我们将要亲手输入的全部内容,出现在\texttt{\backslash begin\{document\}}和\texttt{\backslash end\{document\}}之间。
\end{itemize}

以下介绍各部分的细节。

\subsection{文档的类}

