\part{关于\LaTeX 的那些你想知道却从不敢问的问题}

\chapter{基本原则}

人若身患漏症,他因这漏症就不洁净了。——《圣经·利未记》15:2

本章介绍\LaTeX 的基本原理。你将会看到关于\LaTeX 安装的简介、使用\LaTeX 的基本“流程”(session)介绍、文章格式的结构、使用变音符号的注意事项,认识几个工具,以及了解面对编译错误消息时的态度。

\section{安装}

你想安装\LaTeX 吗?你将要安装的是\LaTeX 的其中一个\textit{发行版},具体的版本取决于你的操作系统\jz{
    如果你不知道操作系统是什么东西,那么你使用的是macOS;如果你不知道你的计算机用的\textit{具体是哪个}操作系统,那么你在用Windows;否则,你在用UNIX……
}。发行版中带有可以自动安装和配置\LaTeX 、\TeX 和其他相关内容的程序。

\paragraph*{对于UNIX}我们可以找到称为te\TeX 的发行版,虽然它的开发早在2006年就停止了。今天,我们一般安装\TeX Live(\wz{http://www.tug.org/texlive})。

\paragraph*{对于macOS}建议安装的发行版是Mac\TeX(\wz{http://www.tug.org/mactex})。

\paragraph*{对于Windows}最简单的方式无疑是选择pro\TeX t(\wz{http://www.tug.org/protext})。它会安装称为MiK\TeX 的发行版(\wz{http://www.miktex.org})和几个开发工具,其中包含一个查看PostScript文件的程序(\textsf{gsview})。

偶尔,需要在为发行版中搭配一款文字编辑器(如果其中没有包含),因为你很快就能看到,使用\LaTeX 就是在文件中输入文字和命令。

\begin{itemize}
    \item UNIX中,推荐使用\textsf{emacs}或\textsf{vi},即使前者明显比后者更高级,但二者用户之间无结果的恶意争吵仍在继续。
    \item \textsf{kile}和\textsf{texmaker}是已集成的开发环境。依靠它们,初学的用户在入门时会觉得更轻松。它们的特点是将编辑、编译和可视化集成在一个界面。这两个环境也使通过菜单、对话框或其他标签来探索\LaTeX 指令称为可能(如图\ref{fig:1.1}a所示)。
    \item Windows中的对应产品是\textsf{\TeX nicCenter}(如图1.1b所示)。
    \item macOS中的对应山品是\textsf{\TeX shop}和\textsf{i\TeX max}。
\end{itemize}

\begin{figure}[H]
    %TODO 图
    \centering
    Kile

    \TeX nicCenter
    \caption{集成的两个开发环境:Linux中的Kile和Windows中的\TeX nicCenter。它们将编辑、编译和可视化集成在一个界面中}
    \label{fig:1.1}
\end{figure}

你很快就会学到,用\LaTeX 制作文档是一个翻译(也称作\textit{编译})的过程——将编辑者创建的源文件转换为用于显示或印刷的格式\jz{本章会略微多介绍一些这个格式。}。因此,发行版中内置了或多或少的著名工具,可以将编译后的不同格式的文件显示出来。

\paragraph*{对于PDF格式}除了著名的\textsf{acrobat reader},UNIX中还有一些可以显示PDF文件,如\textsf{xpdf}、\textsf{evince}等。

\paragraph*{对于DVI格式}UNIX中的\textsf{xdvi}、\textsf{kdvi}和Windows中的\textsf{yap}都是可以显示这种\LaTeX 编译文件的程序。

\paragraph*{对于PostScript格式}\textsf{ghostscript}套件(在各平台下的名称可能有差异)可以显示PostScript文件。

\begin{exclamation}
    需要注意,为了使你选用的发行版包含\LaTeX 的“法文”模式,以确保能够正确处理断字(césure;英:hyphenation),我们需要在编译文档是需要更改其“日志”(见1.6节)%带有引用
    以使法文模式加载:

    \begin{dmd}
    LaTeX2e <2005/12/01>\\
    Babel <v3.8h> and hyphenation patterns for english, [...] dumylang, \fbox{french}, loaded.
    \end{dmd}
\end{exclamation}

\section{“生产”周期}

即使\LaTeX 并不是通常意义上说的编译型语言,但我们仍然可以将制作一个\LaTeX 文档的周期与使用一款经典的编程语言开发软件的\textit{编辑—编译—执行}周期进行类比。

\subsection{编辑}

一个\LaTeX \textit{源}文件是一个文本文件\jz{即文件仅由组成其中符号的代码构成。}。因此,对\LaTeX 文件的操作并不依赖于某个特定的软件,只需要一个经典的文本编辑器即可。因此,若要操作\LaTeX 文档,指令

\dmh{emacs \textsl{\<文件名\>}.tex \&} %TODO <>

或

\dmh{vi \textsl{\<文件名\>}.tex}

足以让你进入\LaTeX 文档这个充满野性和未知的世界。在Windows中,根据自己的喜好,我们可以选用一款文字编辑器。注意,对于\LaTeX 源文件,推荐使用\dm{.tex}扩展名名。

\subsection{编译}

我们用如下指令开始编译:

\dmh{pdflatex \textsl{\<文件名\>}.tex}

早晚有一天,你会看到编译会产出错误。这将是1.6节会处理的问题。总之,解决了编译问题后,我们会得到一个带有\dm{.pdf}扩展名的文件,它代表\textit{便携文件格式(英:portable document format)},这是一种由Adobe公司创造的著名格式。

\begin{exclamation}
    历史上,编译\LaTeX 源文件会生成\dm{dvi}文件,代表\textit{设备无关(英:device independant)}。此类文件独不受输出环境(如屏幕、打印机等)的影响。这是一种包含了“图像”的\LaTeX 便携二进制文件,可以用于各种操作系统。随后,出现了一批用途各异的程序:
    \begin{itemize}
        \item 用于显示文档,即\dm{.dvi}\rightarrow 点阵屏幕;
        \item 用于打印,即\dm{.dvi}\rightarrow 打印机语言;
        \item 用于转换格式,即\dm{.dvi}\rightarrow PostScript文件。
    \end{itemize}
\end{exclamation}

图\ref{fig:1.2}表明了UNIX生成最终文件过程中参与流程的多种程序。

\begin{exclamation}
    除了使用pdflatex外,也可以使用其他“编译器”来生成PDF文件。例如,xelatex和lualatex可以能正确地处理以UTF-8编码的文件,是常用的替代选项。
\end{exclamation}

\begin{figure}[H]
    %TODO 图
    \centering

    \caption{UNIX中参与生成过程的工具}
    \label{fig:1.2}
\end{figure}

\subsection{显示}%visualisation有时翻译成可视化,有时翻译成显示,这个可以后期再统一一下。

在编译后,可以简单地使用\textsf{evince}程序来完成显示步骤。输入以下指令:

\dmh{evince \textsl{\<文件名\>}.pdf \&}

这是一个\textsf{linux}下运行的十分直观的程序,能够给出一个方便阅读的文件预览。

\begin{exclamation}
    注意,不必在每次编译后都重新运行evince,它显示的内容会自动刷新。
\end{exclamation}

\subsection{打印}

对于\dm{pdf}格式,如何打印它这一问题就丢给了你的操作系统。关于这一点,没有特殊的注意事项。你有了一个文件,可以自由地处置它,无论是直接打印,还是根据你所处的环境来发挥才艺。

\begin{exclamation}
    从\dm{dvi}到\dm{ps}格式的转换需要调用dvips程序:

    \dmh{dvips \textsl{\<文件名\>}.dvi}

    这可以生成一个PostScrpt格式的文件。这个格式也由Adobe创造,是一种打印机语言,可以看作\dm{pdf}的祖先。目前的打印机出厂即可识别这种打印机语言。我们可以说,文件发送到打印机时,十有八九传送的是PostScrpt格式的参数。对于PostScript格式的文件,有大量可以显示、修改这种文件的工具。
\end{exclamation}

\section{源文件的结构}

本节将介绍一种文档类型。实际上,所有\LaTeX 文档都具有相同的结构,形式如下:

\begin{dmd}
\backslash documentclass[\textsl{\<类选项$_1$\>},\textsl{\<类选项$_2$\>},...]\{\textsl{\<类\>}\}\\
\backslash usepackage[\textsl{\<包选项$_1$\>},\textsl{\<包选项$_2$\>},...]\{\textsl{\<包\>}\}\\
...\\
\textsl{\<文前部分\>}\\
...\\
\backslash begin\{document\}\\
...\\
\textsl{\<文本\>}\\
...\\
\backslash end\{document\}
\end{dmd}

如此一来,所有的\LaTeX 文档都可以按以下方式拆解。

\begin{itemize}
    \item 说明文档的\textsl{\<类\>};
    \item 文前部分,包含以下内容:
        \begin{itemize}
            \item 使用特定的\textsl{\<包\>};
            \item 多样的初始化和声明;
        \end{itemize}
    \item 文档主体,即我们将要亲手输入的全部内容,出现在\dm{\backslash begin\{document\}}和\dm{\backslash end\{document\}}之间。
\end{itemize}

以下介绍各部分的细节。

\subsection{文档的类}

所谓类,就是提供给\LaTeX 的一个指示,可以帮助\LaTeX 决定如何为文档的特定部分排版。根据具体使用的类不同,允许使用与否的指令可能不同(如\dm{\backslash chapter}在\dm{book}类中允许使用,在\dm{article}类中不允许使用)。另一方面,根据所选择的类,给出的命令会具有特定的含义(标题、材料表……)。在入门时\jz{
    实际上,我们可以在\dm{\backslash documentclass}前添加更多神奇的“咒语”……
},所有的\LaTeX 文档都必须以的指令开始——\dm{\backslash documentclass}接由花括号括住的类,包含以下几种:

\begin{itemize}
    \item \dm{article},用于文章;
    \item \dm{proc},用于电气与电子工程师协会(英:Institute of Electrical and Electronics Engineers,IEEE)会刊(英:proceeding)风格的文章;
    \item \dm{report},用于几十页篇幅的报告;
    \item \dm{book},用于图书或论文;
    \item \dm{letter},用于信件;
    \item \dm{slides},用于演示文档。
\end{itemize}

我们当然也可以为文档定义自己的类。类的配置项用方括号括住,可以是以下内容之一:

\begin{itemize}
    \item \dm{11pt, 12pt},用于全局地更改文字字号;
    \item \dm{twoside},用于生成适合双面打印的文档;
    \item \dm{draft},用于以草稿模式生成文档。
\end{itemize}

例如,输入:

\begin{dmd}
    \backslash documentclass{article}
\end{dmd}

以上命令可以将全部配置项配置为默认值(字号为10 pt,单列,单面……)。

\begin{dmd}
    \backslash documentclass[12pt]{article}
\end{dmd}

以上命令将字号设置为12 pt(默认为10 pt)。再如:

\begin{dmd}
    \backslash documentclass[twoside, draft]{report}
\end{dmd}

以上命令可以以草稿模式生成适合双面打印的报告。

\subsection{文前部分}

文前部分是指位于子句\dm{\backslash documentclass}和子句\dm{\backslash begin\{documennt\}}间的区域。在这个区域中,我们可以明确想要包含的扩展(请看下一小节)%TODO
、初始化全局参数(如页边距等)、定义风格(如标题样式、序号等)、定义特殊的宏,等等。

\subsection{添加扩展}

\LaTeX 命令\dm{\backslash usepackage}可以与C语言的指令\dm{\#include}类比。这一命令允许添加\LaTeX 中满足宏或环境形式的功能\jz{
    相关内容将在下一章讲解。%TODO
}。目前,只需记住,我们可以在一行之内包含多个包:

\begin{dmd}
    \backslash usepackage\{\textsl{\<包$_1$\>},\textsl{\<包$_2$\>},\textsl{\<包$_3$\>},...\}
\end{dmd}

如果\textsl{\<包$_1$\>}、\textsl{\<包$_2$\>}、\textsl{\<包$_3$\>}拥有共同的配置项\textsl{\<opt1\>},我们可以输入:

\begin{dmd}
    \backslash usepackage[\textsl{\<opt1\>}]\{\textsl{\<包$_1$\>},\textsl{\<包$_2$\>},\textsl{\<包$_3$\>}\}
\end{dmd}

相反,如果\textsl{\<opt1\>}只涉及\textsl{\<包$_2$\>},那么我们只能像这样写成两行:

\begin{dmd}
    \backslash usepackage\{\textsl{\<包$_1$\>},\textsl{\<包$_3$\>}\}\\
    \backslash usepackage[\textsl{\<opt1\>}]\{\textsl{\<包$_2$\>}\}
\end{dmd}

下面是两个例子:

\begin{dmd}
    \% 包graphicx带有配置项draft和xdvi\\
    \backslash usepackage[xdvi, draft]\{graphicx\}\\
    \% 包array和包subfig\\
    \backslash usepackage\{array, subfig\}
\end{dmd}

\begin{exclamation}
    根据定义,所有(类、包、命令的)的配置项参数都是\textit{可选的}。因此我们可以这样记:\LaTeX 中所有由方括号括住的参数\dm{[...]}都是非强制的。
\end{exclamation}

\section{开始!}

在本节,我们将尝试从一个只含几个排版命令的文档开始,介绍\LaTeX 的基本原理。

\begin{codelist}[1.1]{
    从你手中掉落的工具总是掉到最难够到的地方,或脆弱的物品上。\\
    这是\emph{墨菲}定律(loi de Murphy)的一个体现。
}  \begin{dmd}
    \backslash documentclass\{article\}\\
    \backslash begin\{document\}\\
    从你手中掉落的工具\\总是掉到最难够到的地方,\\或脆弱的物品上。\\
    ~\\
    这是\backslash emph\{墨菲\}定律(loi\ de\ \ \ \ \  Murphy)的\\一个体现。\\
    \backslash end{document}
\end{dmd}
\end{codelist}

这个示例体现了\LaTeX 中的几个重要的原理,具体如下。

\paragraph*{空行代表跳转至下一段} \LaTeX 中的空行代表一段文字的结尾,因此在以上实例中,第一段从“\dm{从你}”开始,直到“\dm{物品。}”结束。指令\dm{\backslash par}与空行等价,可以用来表示一段文字的起始。

\paragraph*{\LaTeX 会忽略换行}最终的文档中,换行并不由源文件中的换行决定。\LaTeX 会自动为各段文本\textit{打断、压缩、调节}文字,除非你有特殊的要求。

\paragraph*{\LaTeX 会忽略重复的空格}输入1个或18784个空格是等价的,比如源码中\dm{de}和\dm{Murphy}前插入的空格那样。此规则也适用于跳转段落:输入一行或多行空格是等价的。

\paragraph*{“\backslash ”是转义字符(caractère d’échappement;英:escape char)}“\backslash ”可以告诉\LaTeX 它后面的一系列字符是控制序列,也就是说,是最一般意义上的指令(或宏)。这里,它对“墨菲”一词生效,具体的效果由指令\dm{\backslash emph}控制。

\paragraph*{“\{”和“\}”}它们是\textit{组}的定界符,稍后会进一步解释它们。

\section{几个特殊字符}

就像符号“\backslash”的出现所暗示的那样,\LaTeX 中还有10个有特殊含义的符号,在此将其列出:

\begin{dmd}
    \verb+\ $ & % # ^ _ { } ~+
\end{dmd}

%todo 此前代码可以重新简化为verb

以下是一个使用部分特殊字符的案例:

\begin{codelist}[1.2]{
\textbf{是}下标:$x_{i+1}$,还是上标:$e^{i\pi}$;这是问题~1!
}
\begin{verbatim}
%毫无意义的段落
\textbf{是}下标:$x_{i+1}$,
还是上标:$e^{i\pi}$;
这是问题~1!%还是问题2?
\end{verbatim}
\end{codelist}

目前,你需要知道:

\begin{itemize}
    \item \dm{\%}会使得\LaTeX 忽略当前行的剩余部分,因此,它是表示注释的符号(与C中的\dm{//}等价);
    \item \verb+~+代表不可拆分的空格\jz{
        见2.10节。
    },可以防止\LaTeX 在指定的位置断字。尽管有大量的情况需要插入这个符号来表示不可拆分(如所有形如“\verb+图~1+”的情况),然而,对于此类符号的使用,并没有系统化的规则。
    \item \dm{\$}用于标记公式的开始和结束。\LaTeX 遇到一个\dm{\$}符号时,它会切换到\celan{第3章}数学模式,直到遇到下一个\dm{\$}符号。
    \item \dm{\_}和\dm{\^{}}分别代表将文本转化为下标和上标。\textbf{注意},这两个符号只能在数学模式下使用。
    \item \dm{\{}和\dm{\}}分别表示组的开始和结束。本例中出现了两种组:一种出现在数学模式中,用于把将要放到下标或上标的“子公式”组合起来;另一种把将要设置成粗体的文字组合起来。
\end{itemize}

我们可以使用如下的指令来在让文档生成部分特殊字符:

\begin{dmd}
    \verb+\$ \& \% \# \{ \} \_+
\end{dmd}

这串指令可以输出“\$ \& \% \# \{ \} \_”。2.2.5小节%TODO
会解释如何使文档生成其余特殊字符(即\verb+\ ~ ^+)。

\subsection{调用指令}

你已经知道了,要想调用指令或宏,需要输入转义字符,并紧接着输入你想使用的宏名。%TODO 二对一
但是,\LaTeX 如何知道宏名的末尾在哪里呢?此处以用于生成\TeX 标识的\dm{\backslash TeX}为例来解释\yz{
    此例涉及对西文行文中空格的处理,不宜翻译。
}。

\begin{codelist}{
    \TeX book is for \TeX hackers.

    \TeX\  has some powerful macros.

    \LaTeX{} is a document preparation system
}
    \begin{verbatim}
\TeX book is for \TeX hackers.

\TeX\  has some powerful macros.

\LaTeX{} is a document preparation system
    \end{verbatim}
\end{codelist}

\begin{exclamation}
    \verb*|\ |(其中\verb*| |代表空格)称作控制空格(espace de contrôle)。这个空格不会被\LaTeX 忽略。因此,指令“\verb*|et\ \ \ hop !|”会生成“et\ \ \ hop !”。实际上,以\dm{\backslash}\textsl{\<函数\>}\dm{\{}\textsl{\<参数\>}\dm{\}}的形式来调用宏是很好的习惯。因此,使用上例中的的第三种方式比第二种方式更佳。这种形式可以避免空格被忽略的情况发生\jz{
        所以他为什么要跟我们说这些?!
    }。因此,我们将使用\verb|the \teX{}book|来生成“the \TeX{}book”,使用“\verb*|\LaTeX{} is a ...|”来生成“|\LaTeX{} is a ...”。
\end{exclamation}

\subsection{变音符号}