\chapter{你的回合!}

\begin{epigraphe}{《圣经·利未记》18:22}
    不可与男人苟合,\\像与女人一样,\\这本是可憎恶的。
\end{epigraphe}

如果说\LaTeX 真的可以让制作几乎所有我们能想到的文档,那么难处往往在于如何向它发号施令。这里,我们提供几个方便你围绕这个功能怪兽寻找更多文档的要点。

\section{图书和其他手册}

以下作品包含了\TeX 和\LaTeX 的标准文档:

\begin{itemize}
    \item L. 兰波特(L. Lamport)的《\LaTeX :一个文档准备系统》(\emph{\LaTeX : A Document Preparation System})[10];
    \item M. 古森斯(M. Goossens)、F. 米特尔巴赫(F. Mittelbach)和A. 萨马兰(A. Samarin)的《\LaTeX 伴侣》(\emph{The \LaTeX \ Companion})[6];
    \item 《\LaTeX 图形学伴侣》(\emph{The \LaTeX  Graphics Companion})[5],作者同上;
    \item 高德纳的\TeX book(\emph{The \TeX book})[9];
    \item (日渐消弭的)\LaTeX 法文问答网站,网址为
    
    \begin{center}
        \wz{http://www.grappa.univ-lille3.fr/FAQ-LaTeX}
    \end{center}

    目前,网站记录了大约70个问题,基本能够回答关于版式的经典问题。
\end{itemize}

最近,出现了两本法文的入门书:

\begin{itemize}
    \item D. 比图泽(D. Bitouzé)和J.C.沙尔庞捷(J.C. Charpentier)的\emph{\LaTeX }[4];
    \item C. 舍瓦利耶(C. Chevalier)等人的《\LaTeX 速成》(\emph{\LaTeX  pour l'impatient})[3]。
\end{itemize}

以下列出了针对\LaTeXe 的法文书:

\begin{itemize}
    \item M. 博杜安(M. Baudoin)的《去学LaTeX!》(\emph{Apprends LaTeX !});
    \item B. 巴亚尔(B. Bayart)的《好看的\LaTeXe 手册》(\emph{Joli manuel pour \LaTeXe});
    \item C. 威廉斯(C. Willems)和F. 杰拉尔(F. Geraerds)的《\LaTeX 备忘录》(\emph{Aide mémoire pour \LaTeX});
    \item F. 杰拉尔的《\LaTeX 文本处理指南》(\emph{Guide d'introduction français au traitement de texte});
    \item T. 厄伊提克(T. Oetiker)、H. 帕特尔(H. Partl)、I. 希纳(I. Hyna)、E. 薛格尔(E. Schlegl)的《\LaTeXe 简短(吗?)教程》(\emph{Une courte (?) introduction à \LaTeXe}),原书为德文。
\end{itemize}

这些文档通常可以以\dm{pdf}格式获取。

\section{本地使用}

在你扎到可怜的国际网络之前,需要知道,你如果有幸使用了发行版\TeX Live,那么就有了一个可以在全部文档中进行搜索的工具。例如:

\dmh{texdoc graphicx}

该指令可以为你自动查找扩展\textsf{graphicx}的手册。以下位置可以提供一个你的发行版\TeX Live中已安装的包的索引页面:

\begin{dmd}
file:///usr/local/texlive/2013/doc.html
\end{dmd}

当然,前提是该发行版安装在了\dm{/usr/local/}文件夹中。

最后,请善用你系统中搜索可用文件的指令,以尝试借助名称查询关于包或字体的信息。例如,在UNIX中,可以尝试使用:

\dmh{locate babel.pdf}

\section{FTP、网络和新闻订阅}

正如很多开源软件一样,关于\LaTeX ,我们也可以在网上找到大量信息。

\subsection{CTAN}

该网站为\TeX 及周边团队(\TeX et Cie)创建了\emph{标准}档案。CTAN这个悦耳的名字代表“综合\TeX 档案网络”(英:Comprehensive TeX Archive Network)。该档案被复制为多个镜像,其中就支持法文。参考链接为:

\begin{dmd}
\wz{http://www.ctan.org}
\end{dmd}

该网站提供了大多数\LaTeX 包,并配有文档和源代码,便于你将其安装到自己的系统中。如果你有需要,该网站也有FTP镜像,例如:

\begin{dmd}
\wz{ftp://ftp.lip6.fr/pub/TeX/CTAN}
\end{dmd}

\subsection{其他网站}

以下网站介绍围绕\LaTeX 的计划:

\begin{dmd}
\wz{http://www.latex-project.org}
\end{dmd}

可以在这里找到关于\LaTeXe 格式修改的信息和\LaTeX 3的进展。需要注意,该网站为英文。另一个英文网站\TeX FAQ也很有启发性:

\begin{dmd}
\wz{http://tex.stackexchange.com}
\end{dmd}

\subsection{新闻订阅}

有两个关于\TeX 和\LaTeX 的讨论组:

\begin{itemize}
    \item \wz{fr.comp.text.tex},为法文版;
    \item \wz{comp.text.tex},功能相同,但为英文版,有更多流量。
\end{itemize}

如果你愿意在面对偶尔特别多的信息时去花些心思整理,这些讨论组可以构成卓越的信息源。你可以将在小组中提问作为最后的手段——最后手段的意思是,你已经掌握了大量资料。如果你的问题简短而明确,你通常不同等待太久就能得到解答。

~\\

\begin{center}
    *

    *\quad *

    ~\\

    \textbf{轮到你了!}
\end{center}