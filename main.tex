% press command+option+B to build, or change "Auto Build: Run" in Settings.

\documentclass{book}

\usepackage{xeCJK}
\usepackage{geometry}
\usepackage{titlesec}
\usepackage{bm}
\usepackage{underlin}
\usepackage{wrapfig}
\usepackage{caption}
\usepackage{graphicx}
\usepackage{pdfpages}
\usepackage{multirow}%表格合并行列单元格

% 分栏并浮动图片
\usepackage{float}
\usepackage{multicol}
\usepackage[none]{hyphenat}
\usepackage{appendix}

% 数学
\usepackage{unicode-math}%正体希腊字母
\usepackage{mathrsfs}
\usepackage{amsfonts}

% 设置图书版式
\usepackage{marginnote}%侧栏
\usepackage{fancyhdr}%页眉页脚页码
\usepackage{indentfirst}%首段空两格
\usepackage{mdframed}%加阴影
\usepackage{color}%特殊字体颜色
\usepackage{setspace}%行距

\usepackage{hyperref}%所有网页用超链接同步挂靠
\usepackage{xpatch}%设置verbatim前后间距

%行号
\usepackage{lineno}

\def\codereplace#1{%代码中的可替换部分
    \textsl{⟨#1⟩}}
    % \textsl{#1}}

\def\wz#1{%网址
    {\ttfamily #1}}

\def\jz#1{%原书脚注
    \footnote{#1}
}

\def\yz#1{%译注
    \footnote{\textsl{译注:}#1}
}

\def\dm#1{%代码
    \texttt{#1}}

\def\dmh#1{%代码行
    {\xeCJKsetup{CJKecglue={\hskip 0pt}}
    \char"258E \texttt{#1}}
}

\def\celan#1{%侧栏TODO三角形方向
$\blacktriangleright$
\marginpar{
        $\blacktriangleleft$\textsf{#1}
    }
}

\definecolor{LightPink}{RGB}{230,173,173}
\newenvironment{exclamation}[1][!]%原书的圆圈感叹号
{
    \begin{mdframed}[backgroundcolor=LightPink,hidealllines=true]
        {\textcircled{\scriptsize #1}} \quad
        \sffamily
}{      
    \rmfamily
    \end{mdframed}
}

\definecolor{LightBlue}{RGB}{173,216,230}
\newenvironment{ii}[1][i]%原书的圆圈i
{
    \begin{mdframed}[backgroundcolor=LightBlue,hidealllines=true]
        {\textcircled{\scriptsize #1}} \quad
        \sffamily
}{      
    \rmfamily
    \end{mdframed}
}

\newenvironment{dmd}{%代码段
    \xeCJKsetup{CJKecglue={\hskip 0pt}}
    \vspace{0.5em}
    \ttfamily
    \setlength{\parindent}{0pt}
}{
    \rmfamily
    \vspace{0.5em}
}

%代码清单
\newenvironment{codelist}[2][TODO代码清单序号]{
    \setlength{\parindent}{0pt}
    \begin{mdframed}
        \textbf{清单 #1}
        \begin{mdframed}[backgroundcolor=lightgray,hidealllines=true]
            #2 
    \end{mdframed}
}
{
    \end{mdframed}
}

%字体
\setCJKmainfont[
    BoldFont={NotoSerifCJKsc-Bold},
    ItalicFont={Kai},
    SlantedFont={STFangsong},
    ]
    {STSongti-SC-Regular}
\setCJKsansfont[
    ItalicFont={SmileySans-Oblique},
    SlantedFont={STFangsong}]
    {NotoSansCJKsc-DemiLight}
\setCJKmonofont[
    % ItalicFont={SmileySans-Oblique},
    SlantedFont={STFangsong}]
    {STYuanti-SC-Regular}
\setmonofont[
    ItalicFont={LMMono10-Italic},
    SlantedFont={SFMono-RegularItalic}
    ]
    {LMMono10-Regular}

%特殊字符的字体显示
\xeCJKsetcharclass{"2580}{"259F}{1}
\xeCJKsetcharclass{"27E8}{"27E9}{1}
\xeCJKsetup{AutoFallBack=true}
\setCJKfallbackfamilyfont{\CJKrmdefault}{STIXTwoMath-Regular}

\definecolor{MediumBlue}{RGB}{0,0,205}
%引用颜色
\hypersetup{
    colorlinks = true,
    linkcolor = MediumBlue
}

\title{关于\LaTeX 的那些你想知道却从不敢问的问题\\{\small 或者说,如何在不会使用\LaTeX 的情况下使用\LaTeX }\\~\\Tout ce que vous avez toujours voulu savoir sur \LaTeX \ sans jamais oser le demander\\{\small Ou comment utiliser \LaTeX \ quand on n'y connaît goutte}\\~\\ ver. 1.5}
\author{Vincent Lozano 著}

\begin{document}
\maketitle
\tableofcontents
\mainmatter

%正文版式
\newgeometry{
    % inner = 2cm, outer = 4cm,
    % top = 2.5cm, bottom=2.5cm, 
    marginparwidth = 2cm, marginparsep = 0.5cm,
    % includeheadfoot
}
\pagestyle{fancy}
\fancyhead{} 
\fancyhead[RO,LE]{\thepage}
\fancyfoot{}

\setstretch{1.2}
\xpretocmd\verbatim{
    \setlength{\topsep}{0pt}
    \setlength{\partopsep}{0pt}}{}{\fail}

\chapter*{序}

男人的邪恶胜过女人的善良\jz{
    本书的章首引言来自《旧约》与《新约》,将它们引用在这里纯粹是我一手挑动的——有时,这些句子中带有一些与章标题相关的内容(译注:宗教相关内容按原文直译,不代表译者对任何宗教文献中任何语句的认可或否认。本书未标注“译注”字样的脚注均为原书脚注)。
}。——《便西拉智训\yz{原文如此,但作为《诗歌智慧书》一部分的《便西拉智训》(天主教译为《德勋篇》)似乎属于次经,即在一些教派中不被承认作为《圣经》的一部分出现。 
}》42:14

\section*{从前……}

一切始于1990年年初。我当时正在PC 286计算机上使用称为\textit{WordPerfect}的软件,以此入门人们所谓的“文字处理”。这款软件现在仍然存在,并且由Corel公司维护,运行在日后拥有响当当名头的MS-DOS中。MS-DOS集成了用以粗略预览文档的接口,尤其允许用户“看到代码”,也就是借助一种标记语言将文档可视化,以灵活地控制。

稍晚些时间,随着Windows 3.1迅速风靡,人们突如其来地追求图形界面,我虽然仍情有不甘,却逐渐说服了自己去使用那款在今天很出名的文字处理软件——的2.0版(后面还带个小小的字母,在当时那可真是重大的升级)……但我日后才知道,这个版本有个很有趣的“特性”:文件体积过大,超过了某个特定的值时,会出现保存失败的情况!这时,你既不能保存,也不能恢复文档。有些头铁的朋友尝试先删除几行再保存,但这种撞大运的解决方案并没能成功……

当时,大家毫不掩饰地嘲讽这些“你懂的”公司制作的软件\jz{
    这些被嘲讽的对象中,我们可以看到一些名场面:通用汽车公司老板对比尔·盖茨挑衅性言论的回应(译注:可能是指比尔·盖茨的观点,即如果汽车工业能够像计算机领域一样发展,那么一辆汽车只需要25美元就能买到,并且消耗1加仑汽油就能跑1000英里。作为回应,通用汽车方面罗列了一系列言论来嘲讽,例如“如果那样,那么想要汽车熄火,需要点击开始菜单”),以及罗伯托·迪·科斯莫(Roberto Di Cosmo)的“赛博空间中的陷阱”(piège dans le cyberespace)。
}——这里就不点名了。我周围的大多数人躺平地选择了接受,认为使用这些堂而皇之不给出警告的可悲的跟风之流是正常现象。软件的这种“特性”坚定了我的信念:\textit{我绝不使用这种软件}。当时还在攻读工程师学位的我意识到,我今后的部分工作将会集中在起草文档和使用通用的信息系统上。为此,我需要足够健壮的工具。

我是在让·莫奈大学(Université Jean Monnet)和圣-埃蒂安高等矿业学校(École des Mines de Saint-Étienne)攻读DEA(现在叫master recherche)\jz{
    译注:DEA即diplôme d'études approfondies,法国教育体系下的一种学位。
}时相继接触UNIX和Linux的。那时(1993~1994年),在我刚写论文的开头时,“拉泰克”(latèque)这个词就开始围着我转。这里问题似乎是要找到一款能排出数学公式的软件,而说到撰写理科文档,\LaTeX 似乎显然是避不开的\textit{唯一}答案。说实话,找软件这种问题甚至都根本没出现过!

于是,我着手把这个叫做\LaTeX 的“玩意儿”装在Mac系统(安装的发行版叫Oz\TeX )和另一个由古登堡(Gutenberg)协会支持的发行版系统——Solaris上。为此,我还得去收买一个系统管理员,让他同意创建一个特权用户\texttt{texadm},用来管理那个发行版……

1994年年初,我带着坚定的意志使用\LaTeX 开始写论文。在1995年,在被我发现的种种技巧激起的兴趣的巨大感召下,% 此句翻译不好:enthou- siasmé par ce que je découvrais, TODO
我着手为同事和实验室起草用于入门\LaTeX 的指导手册。这个手册就是本书的原型。在1997年,在练习了两年并一只脚踏入了排版领域后,我更坚定了自己的看法:\LaTeX 绝对是写严肃文件的首选软件:它有对版面(mise en page)的全面控制,有对参考文献的管理,支持索引(通用名称和作者名),能轻松操作文件。最重要的是,\textit{排版的结果很好看}。从那时起,这就是支撑我使用\LaTeX 的最强大而无可争辩的理由。

今天,作为国立圣-埃蒂安工程师学院(École Nationale d’ingénieurs de Saint-Étienne)的计算机高级讲师,我用\LaTeX 来起草理科文档和教学材料。几年使用下来,我仍然在学习和发现,也仍然会对项目贡献者提出的各种扩展啧啧称奇。这些扩展使\LaTeX 成为了充满宝藏的巴扎(bazar),成为了一款名副其实地朝着更高工效发展的\jz{
    并不是指那些诸如在菜单中添加一个功能入口、在弹出对话框时添加个提示音的“提效”。
}、始终以“产出优美的工作成果”为目标的卓越而独特的工具。

\section*{本书结构}

本书是针对“使用\LaTeX 进行文字处理”的介绍。它不是一本参考手册,但本书的写作目标是传授读者使用\LaTeX 的基本知识,并在可能情况下,让读者对它感兴趣。读者可以在本书中找到\textit{开始}使用\LaTeX 的必要信息和起草文档的建议。为了提升阅读体验,我们“高明地”将本书分为了若干章节,并配有附录。
本书首先介绍\LaTeX 的基础知识:

% todo

然后有如下附录:

% todo

我们建议您先从第1章一路读到数学部分。其余的章节相对独立,可以根据需要阅读。再强调一遍,我们建议在熟练掌握了基础概念之后再去阅读本书的第II部分。文档最后的索引提供了查询所需内容的快捷入口。最后,正如同其他关于\LaTeX 的答疑解惑的法文资料,我没有费神地将所有\LaTeX 术语和计算机术语逐一翻译。

\section*{你需要知道的知识}

本书适用于初学者阅读,不要求读者有关于\LaTeX 的任何知识。然而,本书读者应当具有基本的、有关操作系统和计算机用户的知识。本书读者最好懂得如何从使用绘图获图片处理软件开始,创建一个封装在其计算机系统中的PostScript文件。

\section*{你不会通过本书学习到的知识}

你正阅读的这本图书在令人称赞的同时也有以下知识面漏洞。

\begin{itemize}
    \item 本书不含有关于\TeX 或\LaTeX 生成字体原理的清晰解释。你不会找到关于“元字体”(METAFONT)一词的知识。
    \item 你不会找到关于在UNIX系统下安装\LaTeX 发布版的知识。
    \item 你不会找到任何现有扩展包的“目录”或清单,无论扩展包是否实用、是否兼容。
    \item 本书回避了“先有鸡还是先有蛋”之类的问题,也避免讨论关于上帝和科学的问题。
    \item ……
\end{itemize}

\begin{exclamation}

不要对本书的内容抱有不切实际的幻想:本书书名着实是个不要脸的谎言。

\end{exclamation}

\section*{\TeX 是什么?}

唐纳德·欧文·克努特(Donald Ervin Knuth)——就是那个有着众多关于数学和算法的著作[包括《计算机程序设计的艺术》(英:\textit{The Art of Computer Programing})]%todo ref
的数学家——对20世纪70年代的技术条件下打印出来的文章的样子深感失望,产生了开发称为\TeX 的文字处理系统的初步想法。20世纪80年代初次公布的\TeX 是由一个宏处理器(processeur de macro ;英:macro processsor)和几个基元(primitive)组成的复杂系统。第一组预编译的宏很快以\textit{“普通格式”(format plain)}的名义出现。

注意,\TeX 既不是文字处理器[克努特将其称为“typesetting system”,可以翻译成“排字系统”(système de composition)]也不是一种编译后的编程语言。这是克努特关于\TeX 的一些说明\jz{出自\TeX Book的“The Name of the Game”一章。}:

\begin{quote}
    “英文的‘technology’一词由希腊文词根‘$ \tau\epsilon\chi...$’演变而来,这个词根有时也指艺术和科学技术。\TeX 由此而来,正是$ \tau\epsilon\chi$的大写形式。”
\end{quote}

关于\TeX 中“X”的发音:

\begin{quote}
    “……它的发音像德语单词ach中的‘ch’,或西班牙语中的‘j’……如果你对着电脑正确地发音,屏幕上会出现哈气。”
\end{quote}

你家里的家政阿姨可能会更想让你读成“TeK”,从而避免读得像橡胶一样,或者没两天就要擦电脑。\yz{此句原文是“Votre humble serviteur se contente lui de le prononcer « TeK » pour contrecarrer l’aspect caoutchouteux et éviter d’avoir à nettoyer son écran régulièrement.”,实在没看懂是什么意思(尤其是中间出现的lui看不太懂是哪个词的宾语),先大致猜着翻了,请大家赐教。}

最后,对于\TeX 的标识设计,克努特强调字母E需要稍微错位一些,以提示人们这是关于排版的工具。对于确实会遇到的一些无法使字母E稍微错位的情况,他坚持道,需要将\TeX 写成“\texttt{TeX}”。

目前,\TeX 的最新版本号是3.1415926(没错,它收敛于$\pi$)。在\textit{\TeX : the program}一书的前言中,克努特估测上一个程序漏洞已于1985年11月27日发现并改正,并出价20.48美元来悬赏下一个漏洞。今天,这个十六进制的金额停留在327.68美元,如果有人喜欢2的幂,这个数字应该会让他满意……

\section*{\LaTeX 是什么?}

1985年,\TeX 已经传播了一段时间,莱斯利·兰波特(Leslie Lamport)将宏组合起来,创造了一个视野更广的格式,称为\LaTeX ,版本号为2.09。今天,\LaTeX 已经成为了事实标准,只有一些玍古的情况才会只支持\TeX 而不支持\LaTeX 。然而,\LaTeX 有点像\TeX 的“镀层”,提供\TeX 的宏的调用。有时,掌握\TeX 中的部分概念有助于从困难的处境中脱身。兰波特在他的书中这样说[10]:%TODO cf.

\begin{quote}
    “可以将\LaTeX 想象成一幢房子,它的构架和钉子就是由\TeX 提供的。如果你只是在房子中生活,那么你不需要准备钉子、搭建构架,但如果想要为房子新增一个房间,那么你就会需要它们。”
\end{quote}

他还说道:

\begin{quote}
    “\LaTeX 的出名是因为它允许作者从排版工作中抽离,并且专注在写作上。如果你在形式上花费了太多时间,那么你并没有很好地使用\LaTeX 。”
\end{quote}

从1994年至今,一个由欧美成员组成的团队[以弗朗克·米特尔巴赫(Frank Mittelbach)为核心]着手\LaTeX 的开发。1994年发布的\LaTeX 版本称作\LaTeXe。团队的长期目标是孵化一个名为\LaTeX 3的系统。

\section*{使用许可}

画重点:\TeX 和\LaTeX 属于自由软件——也因此是免费的。同时,自由软件(logiciels libres;英:free software)的标志是其\textit{开放性}。因此,\TeX 也可以有其Web源码\jz{克努特孕育的Web语言被形容为一种“文学性的编程语言”。使用Web源码,可以生成程序的Pascal或C代码,也可以为代码生成\TeX 文档。}。\LaTeX 的宏是以\TeX 源码的格式发布的\yz{此句原文:Les macros de \LaTeX \ sont quant à elles distribuées sous forme de code source \TeX .,翻译时没看懂elles指谁,翻译可能有误。我查到macro可以写成macro-instruction,为阴性,故用elles指代这些宏,但不能确定。请读者赐教。}。对于大部分用户来说,获取程序的源码可能不是首要考虑的,但需要知道,正是这种\textit{不隐藏任何内容}的性质,使得人们可以改进现有的扩展、创造新的扩展。

一款软件是自由软件,并不意味着我们可以使用它做任何想做的事情。自由软件属于其作者,所有的改动都需要被记录。同样,每次改动都需要以与具有与改动前不同的文件名体现。这样可以保证系统的严密和便携(关于\LaTeXe 的使用许可,请参阅\wz{ftp://ftp.lip6.fr/pub/TeX/CTAN/macros/latex/base/lppl.txt})。

\section*{不使用\LaTeX 的5个理由}

在一些情况下,强烈建议不使用\LaTeX 。具体来说,这些不使用\LaTeX 的理由如下。

\begin{enumerate}
    \item 你只将文字处理器用于制作贺卡、写邮件、记录几个想法等用途。
    \item 你十分喜欢鼠标(可能具有1~3个按键),并且认为输入方程的唯一方式就是频繁地使用鼠标点来点去。
    \item 你觉得UNIX是一个“让人头痛”且“不易使用”的系统,或者你对所有的编程语言都有着强烈的反感。
    \item 你认为以下情况是正常的:
        \begin{enumerate}
            \item 新版软件不能读取其旧版本创建的文档;
            \item 要使用新版软件,必须换一个操作系统;
            \item 要使用新版操作系统,必须换一台计算机;
            \item 要使用新计算机,必须……
        \end{enumerate}
    \item 你不知道键盘上的“\backslash”键在哪里。
\end{enumerate}

如果你的情况满足以上任何一条,最好在你现在的系统上知足常乐。

\section*{使用\LaTeX 的若干理由}

说服本书读者使用\TeX 和\LaTeX 而不是其他系统似乎不成问题——毕竟,你都读这本书了,也就已经不知不觉被说服了。让我们看看\TeX 的设计者是怎么说的:

\begin{quote}
    “在使用\TeX 起草文档时,你就是在指挥计算机如何准确地把你的稿件转化为几个页面,以媲美世界上最好的打印机能够实现的排版样式。”——D.E. 克努特,\TeX Book[9]
\end{quote}

\TeX 和\LaTeX 可以生成无与伦比的文档(并可以极细微地调整\jz{
    作为参考,\LaTeX 内置的衡量单位是\textit{比例点(英:scaled point)},在\TeX Book中记作\texttt{sp},合$1/65536$点;1点合约$1/72$英寸;1英寸合2.54厘米。比例点可以在大约50埃米的尺度上调整文档。目前打印机的分辨率对于这个尺度来说,实在是太充裕了。
}),这显然归功于以下元素:

\begin{itemize}
    \item 仔细地绘制字体;
    \item 处理排版上的细节,如连接号(tiret)和合字:
    \begin{itemize}
        \item “avez-vous --- bien --- regardé ces tirets (page 19--23) ”这句文字中的各种连接号;
        \item fin一词中的“f\/i”、souffle一词中的“f\/f\/l”,以及trèfle一词中的“f\/l”;
    \end{itemize}
    \item 性能良好的断字算法;
    \item 专门针对数学公式的呈现。
\end{itemize}

此外,\LaTeX 是少数瞄准\textit{科技}文档的文字处理软件。这是因为,除了处理方程和公式之外外,\LaTeX 还有大量围绕起草\textit{文章}、生成\textit{参考文献}和\textit{索引}的功能。

最后,\LaTeX 尤其针对大文件的生成做了适配。这不仅是由于处理\LaTeX 文档本身占用的内存空间极小,也是因为\textit{宏}和\textit{交叉引用(référence croisée;英:cross reference)}可以让我们对文件有着全面而灵活的控制。

\paragraph*{交叉引用}\LaTeX 允许\textit{以符号的形式}于文档的任何位置引用有编号的对象。此外,标题、图片、表格、方程、参考文献、列表、定理等的序号都可以在文章的多个位置以简单的方式引用,不需要我们去关心具体的号码本身是多少。

\paragraph*{宏}宏无疑是\LaTeX 最强大的功能。
\part{关于\LaTeX 的那些你想知道却从不敢问的问题}

\chapter{基本原则}

人若身患漏症,他因这漏症就不洁净了。——《圣经·利未记》15:2

本章介绍\LaTeX 的基本原理。你将会看到关于\LaTeX 安装的简介、使用\LaTeX 的基本“流程”(session)介绍、文章格式的结构、使用变音符号的注意事项,认识几个工具,以及了解面对编译错误消息时的态度。

\section{安装}

你想安装\LaTeX 吗?你将要安装的是\LaTeX 的其中一个\textit{发行版},具体的版本取决于你的操作系统\jz{
    如果你不知道操作系统是什么东西,那么你使用的是macOS;如果你不知道你的计算机用的\textit{具体是哪个}操作系统,那么你在用Windows;否则,你在用UNIX……
}。发行版中带有可以自动安装和配置\LaTeX 、\TeX 和其他相关内容的程序。

\paragraph*{对于UNIX}我们可以找到称为te\TeX 的发行版,虽然它的开发早在2006年就停止了。今天,我们一般安装\TeX Live(\wz{http://www.tug.org/texlive})。

\paragraph*{对于macOS}建议安装的发行版是Mac\TeX(\wz{http://www.tug.org/mactex})。

\paragraph*{对于Windows}最简单的方式无疑是选择pro\TeX t(\wz{http://www.tug.org/protext})。它会安装称为MiK\TeX 的发行版(\wz{http://www.miktex.org})和几个开发工具,其中包含一个查看PostScript文件的程序(\textsf{gsview})。

偶尔,需要在为发行版中搭配一款文字编辑器(如果其中没有包含),因为你很快就能看到,使用\LaTeX 就是在文件中输入文字和命令。

\begin{itemize}
    \item UNIX中,推荐使用\textsf{emacs}或\textsf{vi},即使前者明显比后者更高级,但二者用户之间无结果的恶意争吵仍在继续。
    \item \textsf{kile}和\textsf{texmaker}是已集成的开发环境。依靠它们,初学的用户在入门时会觉得更轻松。它们的特点是将编辑、编译和可视化集成在一个界面。这两个环境也使通过菜单、对话框或其他标签来探索\LaTeX 指令称为可能(如图\ref{fig:1.1}a所示)。
    \item Windows中的对应产品是\textsf{\TeX nicCenter}(如图1.1b所示)。
    \item macOS中的对应山品是\textsf{\TeX shop}和\textsf{i\TeX max}。
\end{itemize}

\begin{figure}[H]
    %TODO 图
    \centering
    Kile

    \TeX nicCenter
    \caption{集成的两个开发环境:Linux中的Kile和Windows中的\TeX nicCenter。它们将编辑、编译和可视化集成在一个界面中}
    \label{fig:1.1}
\end{figure}

你很快就会学到,用\LaTeX 制作文档是一个翻译(也称作\textit{编译})的过程——将编辑者创建的源文件转换为用于显示或印刷的格式\jz{本章会略微多介绍一些这个格式。}。因此,发行版中内置了或多或少的著名工具,可以将编译后的不同格式的文件显示出来。

\paragraph*{对于PDF格式}除了著名的\textsf{acrobat reader},UNIX中还有一些可以显示PDF文件,如\textsf{xpdf}、\textsf{evince}等。

\paragraph*{对于DVI格式}UNIX中的\textsf{xdvi}、\textsf{kdvi}和Windows中的\textsf{yap}都是可以显示这种\LaTeX 编译文件的程序。

\paragraph*{对于PostScript格式}\textsf{ghostscript}套件(在各平台下的名称可能有差异)可以显示PostScript文件。

\begin{exclamation}
    需要注意,为了使你选用的发行版包含\LaTeX 的“法文”模式,以确保能够正确处理断字(césure;英:hyphenation),我们需要在编译文档是需要更改其“日志”(见1.6节)%带有引用
    以使法文模式加载:

    \begin{dmd}
    LaTeX2e <2005/12/01>\\
    Babel <v3.8h> and hyphenation patterns for english, [...] dumylang, \fbox{french}, loaded.
    \end{dmd}
\end{exclamation}

\section{“生产”周期}

即使\LaTeX 并不是通常意义上说的编译型语言,但我们仍然可以将制作一个\LaTeX 文档的周期与使用一款经典的编程语言开发软件的\textit{编辑—编译—执行}周期进行类比。

\subsection{编辑}

一个\LaTeX \textit{源}文件是一个文本文件\jz{即文件仅由组成其中符号的代码构成。}。因此,对\LaTeX 文件的操作并不依赖于某个特定的软件,只需要一个经典的文本编辑器即可。因此,若要操作\LaTeX 文档,指令

\dmh{emacs \textsl{\<文件名\>}.tex \&} %TODO <>

或

\dmh{vi \textsl{\<文件名\>}.tex}

足以让你进入\LaTeX 文档这个充满野性和未知的世界。在Windows中,根据自己的喜好,我们可以选用一款文字编辑器。注意,对于\LaTeX 源文件,推荐使用\dm{.tex}扩展名名。

\subsection{编译}

我们用如下指令开始编译:

\dmh{pdflatex \textsl{\<文件名\>}.tex}

早晚有一天,你会看到编译会产出错误。这将是1.6节会处理的问题。总之,解决了编译问题后,我们会得到一个带有\dm{.pdf}扩展名的文件,它代表\textit{便携文件格式(英:portable document format)},这是一种由Adobe公司创造的著名格式。

\begin{exclamation}
    历史上,编译\LaTeX 源文件会生成\dm{dvi}文件,代表\textit{设备无关(英:device independant)}。此类文件独不受输出环境(如屏幕、打印机等)的影响。这是一种包含了“图像”的\LaTeX 便携二进制文件,可以用于各种操作系统。随后,出现了一批用途各异的程序:
    \begin{itemize}
        \item 用于显示文档,即\dm{.dvi}\rightarrow 点阵屏幕;
        \item 用于打印,即\dm{.dvi}\rightarrow 打印机语言;
        \item 用于转换格式,即\dm{.dvi}\rightarrow PostScript文件。
    \end{itemize}
\end{exclamation}

图\ref{fig:1.2}表明了UNIX生成最终文件过程中参与流程的多种程序。

\begin{exclamation}
    除了使用pdflatex外,也可以使用其他“编译器”来生成PDF文件。例如,xelatex和lualatex可以能正确地处理以UTF-8编码的文件,是常用的替代选项。
\end{exclamation}

\begin{figure}[H]
    %TODO 图
    \centering

    \caption{UNIX中参与生成过程的工具}
    \label{fig:1.2}
\end{figure}

\subsection{显示}%visualisation有时翻译成可视化,有时翻译成显示,这个可以后期再统一一下。

在编译后,可以简单地使用\textsf{evince}程序来完成显示步骤。输入以下指令:

\dmh{evince \textsl{\<文件名\>}.pdf \&}

这是一个\textsf{linux}下运行的十分直观的程序,能够给出一个方便阅读的文件预览。

\begin{exclamation}
    注意,不必在每次编译后都重新运行evince,它显示的内容会自动刷新。
\end{exclamation}

\subsection{打印}

对于\dm{pdf}格式,如何打印它这一问题就丢给了你的操作系统。关于这一点,没有特殊的注意事项。你有了一个文件,可以自由地处置它,无论是直接打印,还是根据你所处的环境来发挥才艺。

\begin{exclamation}
    从\dm{dvi}到\dm{ps}格式的转换需要调用dvips程序:

    \dmh{dvips \textsl{\<文件名\>}.dvi}

    这可以生成一个PostScrpt格式的文件。这个格式也由Adobe创造,是一种打印机语言,可以看作\dm{pdf}的祖先。目前的打印机出厂即可识别这种打印机语言。我们可以说,文件发送到打印机时,十有八九传送的是PostScrpt格式的参数。对于PostScript格式的文件,有大量可以显示、修改这种文件的工具。
\end{exclamation}

\section{源文件的结构}

本节将介绍一种文档类型。实际上,所有\LaTeX 文档都具有相同的结构,形式如下:

\begin{dmd}
\backslash documentclass[\textsl{\<类选项$_1$\>},\textsl{\<类选项$_2$\>},...]\{\textsl{\<类\>}\}\\
\backslash usepackage[\textsl{\<包选项$_1$\>},\textsl{\<包选项$_2$\>},...]\{\textsl{\<包\>}\}\\
...\\
\textsl{\<文前部分\>}\\
...\\
\backslash begin\{document\}\\
...\\
\textsl{\<文本\>}\\
...\\
\backslash end\{document\}
\end{dmd}

如此一来,所有的\LaTeX 文档都可以按以下方式拆解。

\begin{itemize}
    \item 说明文档的\textsl{\<类\>};
    \item 文前部分,包含以下内容:
        \begin{itemize}
            \item 使用特定的\textsl{\<包\>};
            \item 多样的初始化和声明;
        \end{itemize}
    \item 文档主体,即我们将要亲手输入的全部内容,出现在\dm{\backslash begin\{document\}}和\dm{\backslash end\{document\}}之间。
\end{itemize}

以下介绍各部分的细节。

\subsection{文档的类}

所谓类,就是提供给\LaTeX 的一个指示,可以帮助\LaTeX 决定如何为文档的特定部分排版。根据具体使用的类不同,允许使用与否的指令可能不同(如\dm{\backslash chapter}在\dm{book}类中允许使用,在\dm{article}类中不允许使用)。另一方面,根据所选择的类,给出的命令会具有特定的含义(标题、材料表……)。在入门时\jz{
    实际上,我们可以在\dm{\backslash documentclass}前添加更多神奇的“咒语”……
},所有的\LaTeX 文档都必须以的指令开始——\dm{\backslash documentclass}接由花括号括住的类,包含以下几种:

\begin{itemize}
    \item \dm{article},用于文章;
    \item \dm{proc},用于电气与电子工程师协会(英:Institute of Electrical and Electronics Engineers,IEEE)会刊(英:proceeding)风格的文章;
    \item \dm{report},用于几十页篇幅的报告;
    \item \dm{book},用于图书或论文;
    \item \dm{letter},用于信件;
    \item \dm{slides},用于演示文档。
\end{itemize}

我们当然也可以为文档定义自己的类。类的配置项用方括号括住,可以是以下内容之一:

\begin{itemize}
    \item \dm{11pt, 12pt},用于全局地更改文字字号;
    \item \dm{twoside},用于生成适合双面打印的文档;
    \item \dm{draft},用于以草稿模式生成文档。
\end{itemize}

例如,输入:

\begin{dmd}
    \backslash documentclass{article}
\end{dmd}

以上命令可以将全部配置项配置为默认值(字号为10 pt,单列,单面……)。

\begin{dmd}
    \backslash documentclass[12pt]{article}
\end{dmd}

以上命令将字号设置为12 pt(默认为10 pt)。再如:

\begin{dmd}
    \backslash documentclass[twoside, draft]{report}
\end{dmd}

以上命令可以以草稿模式生成适合双面打印的报告。

\subsection{文前部分}

文前部分是指位于子句\dm{\backslash documentclass}和子句\dm{\backslash begin\{documennt\}}间的区域。在这个区域中,我们可以明确想要包含的扩展(请看下一小节)%TODO
、初始化全局参数(如页边距等)、定义风格(如标题样式、序号等)、定义特殊的宏,等等。

\subsection{添加扩展}

\LaTeX 命令\dm{\backslash usepackage}可以与C语言的指令\dm{\#include}类比。这一命令允许添加\LaTeX 中满足宏或环境形式的功能\jz{
    相关内容将在下一章讲解。%TODO
}。目前,只需记住,我们可以在一行之内包含多个包:

\begin{dmd}
    \backslash usepackage\{\textsl{\<包$_1$\>},\textsl{\<包$_2$\>},\textsl{\<包$_3$\>},...\}
\end{dmd}

如果\textsl{\<包$_1$\>}、\textsl{\<包$_2$\>}、\textsl{\<包$_3$\>}拥有共同的配置项\textsl{\<opt1\>},我们可以输入:

\begin{dmd}
    \backslash usepackage[\textsl{\<opt1\>}]\{\textsl{\<包$_1$\>},\textsl{\<包$_2$\>},\textsl{\<包$_3$\>}\}
\end{dmd}

相反,如果\textsl{\<opt1\>}只涉及\textsl{\<包$_2$\>},那么我们只能像这样写成两行:

\begin{dmd}
    \backslash usepackage\{\textsl{\<包$_1$\>},\textsl{\<包$_3$\>}\}\\
    \backslash usepackage[\textsl{\<opt1\>}]\{\textsl{\<包$_2$\>}\}
\end{dmd}

下面是两个例子:

\begin{dmd}
    \% 包graphicx带有配置项draft和xdvi\\
    \backslash usepackage[xdvi, draft]\{graphicx\}\\
    \% 包array和包subfig\\
    \backslash usepackage\{array, subfig\}
\end{dmd}

\begin{exclamation}
    根据定义,所有(类、包、命令的)的配置项参数都是\textit{可选的}。因此我们可以这样记:\LaTeX 中所有由方括号括住的参数\dm{[...]}都是非强制的。
\end{exclamation}

\section{开始!}

在本节,我们将尝试从一个只含几个排版命令的文档开始,介绍\LaTeX 的基本原理。

\begin{codelist}[1.1]{
    从你手中掉落的工具总是掉到最难够到的地方,或脆弱的物品上。\\
    这是\emph{墨菲}定律(loi de Murphy)的一个体现。
}  \begin{dmd}
    \backslash documentclass\{article\}\\
    \backslash begin\{document\}\\
    从你手中掉落的工具\\总是掉到最难够到的地方,\\或脆弱的物品上。\\
    ~\\
    这是\backslash emph\{墨菲\}定律(loi\ de\ \ \ \ \  Murphy)的\\一个体现。\\
    \backslash end{document}
\end{dmd}
\end{codelist}

这个示例体现了\LaTeX 中的几个重要的原理,具体如下。

\paragraph*{空行代表跳转至下一段} \LaTeX 中的空行代表一段文字的结尾,因此在以上实例中,第一段从“\dm{从你}”开始,直到“\dm{物品。}”结束。指令\dm{\backslash par}与空行等价,可以用来表示一段文字的起始。

\paragraph*{\LaTeX 会忽略换行}最终的文档中,换行并不由源文件中的换行决定。\LaTeX 会自动为各段文本\textit{打断、压缩、调节}文字,除非你有特殊的要求。

\paragraph*{\LaTeX 会忽略重复的空格}输入1个或18784个空格是等价的,比如源码中\dm{de}和\dm{Murphy}前插入的空格那样。此规则也适用于跳转段落:输入一行或多行空格是等价的。

\paragraph*{“\backslash ”是转义字符(caractère d’échappement;英:escape char)}“\backslash ”可以告诉\LaTeX 它后面的一系列字符是控制序列,也就是说,是最一般意义上的指令(或宏)。这里,它对“墨菲”一词生效,具体的效果由指令\dm{\backslash emph}控制。

\paragraph*{“\{”和“\}”}它们是\textit{组}的定界符,稍后会进一步解释它们。

\section{几个特殊字符}

就像符号“\backslash”的出现所暗示的那样,\LaTeX 中还有10个有特殊含义的符号,在此将其列出:

\begin{dmd}
    \verb+\ $ & % # ^ _ { } ~+
\end{dmd}

%todo 此前代码可以重新简化为verb

以下是一个使用部分特殊字符的案例:

\begin{codelist}[1.2]{
\textbf{是}下标:$x_{i+1}$,还是上标:$e^{i\pi}$;这是问题~1!
}
\begin{verbatim}
%毫无意义的段落
\textbf{是}下标:$x_{i+1}$,
还是上标:$e^{i\pi}$;
这是问题~1!%还是问题2?
\end{verbatim}
\end{codelist}

目前,你需要知道:

\begin{itemize}
    \item \dm{\%}会使得\LaTeX 忽略当前行的剩余部分,因此,它是表示注释的符号(与C中的\dm{//}等价);
    \item \verb+~+代表不可拆分的空格\jz{
        见2.10节。
    },可以防止\LaTeX 在指定的位置断字。尽管有大量的情况需要插入这个符号来表示不可拆分(如所有形如“\verb+图~1+”的情况),然而,对于此类符号的使用,并没有系统化的规则。
    \item \dm{\$}用于标记公式的开始和结束。\LaTeX 遇到一个\dm{\$}符号时,它会切换到\celan{第3章}数学模式,直到遇到下一个\dm{\$}符号。
    \item \dm{\_}和\dm{\^{}}分别代表将文本转化为下标和上标。\textbf{注意},这两个符号只能在数学模式下使用。
    \item \dm{\{}和\dm{\}}分别表示组的开始和结束。本例中出现了两种组:一种出现在数学模式中,用于把将要放到下标或上标的“子公式”组合起来;另一种把将要设置成粗体的文字组合起来。
\end{itemize}

我们可以使用如下的指令来在让文档生成部分特殊字符:

\begin{dmd}
    \verb+\$ \& \% \# \{ \} \_+
\end{dmd}

这串指令可以输出“\$ \& \% \# \{ \} \_”。2.2.5小节%TODO
会解释如何使文档生成其余特殊字符(即\verb+\ ~ ^+)。

\subsection{调用指令}

你已经知道了,要想调用指令或宏,需要输入转义字符,并紧接着输入你想使用的宏名。%TODO 二对一
但是,\LaTeX 如何知道宏名的末尾在哪里呢?此处以用于生成\TeX 标识的\dm{\backslash TeX}为例来解释\yz{
    此例涉及对西文行文中空格的处理,不宜翻译。
}。

\begin{codelist}{
    \TeX book is for \TeX hackers.

    \TeX\  has some powerful macros.

    \LaTeX{} is a document preparation system
}
    \begin{verbatim}
\TeX book is for \TeX hackers.

\TeX\  has some powerful macros.

\LaTeX{} is a document preparation system
    \end{verbatim}
\end{codelist}

\begin{exclamation}
    \verb*|\ |(其中\verb*| |代表空格)称作控制空格(espace de contrôle)。这个空格不会被\LaTeX 忽略。因此,指令“\verb*|et\ \ \ hop !|”会生成“et\ \ \ hop !”。实际上,以\dm{\backslash}\textsl{\<函数\>}\dm{\{}\textsl{\<参数\>}\dm{\}}的形式来调用宏是很好的习惯。因此,使用上例中的的第三种方式比第二种方式更佳。这种形式可以避免空格被忽略的情况发生\jz{
        所以他为什么要跟我们说这些?!
    }。因此,我们将使用\verb|the \teX{}book|来生成“the \TeX{}book”,使用“\verb*|\LaTeX{} is a ...|”来生成“|\LaTeX{} is a ...”。
\end{exclamation}

\subsection{变音符号}
\chapter{需要了解的知识}

\begin{quote}
    亵慢的人受刑罚,愚蒙的人就得智慧。智慧人受训诲,便得知识。——《圣经·箴言》12:11
\end{quote}


本章要研究使用\LaTeX 生成文档时的基本排版指令。我们将零散地处理用于突出显示、\LaTeX 标准环境、标题、页面下方的注释、页眉和页脚,以及浮动的环境。接下来,我们会介绍参考系统和\LaTeX 生成的辅助文件。最后,阅读到本章末尾的人将有机会读到一些关于断字的思考。

所有这些指令都将以其默认行为模式使用。也就是说,我们这里不介绍重新定义它们的方法。对应地,你将能够以传统的版式来生成文档。若要打出一篇更进阶的文章,你需要了解如何输入数学式(第3章)、一些关于科技文档的知识(第6章),以及包含图像的方法(第5章)。

\section{突出显示}

要了解\LaTeX 选用字体的机制,需要知道:我们通常通过4个参数来区别字体。

\paragraph*{族(famille)}指字体的整体形状。默认情况下,\LaTeX 使用三种字体族:罗马体、\textsf{无衬线体}、\texttt{打字机体}。\LaTeX 中,以英文单词\textit{family}来指代字体族。

\paragraph*{风格(style)}指字体体现出的体态(英文以\textit{shape}指代),分为:\textit{意大利体}、\textsl{倾斜}和\textsc{小型大写字母风格(petites capitales)}\yz{
    本书翻译时,以楷体对应意大利体、以仿宋体对应倾斜排版,按原文翻译。中文字体的族和风格往往是并列的,很少有交叉叠加的情况,因此在展示叠加效果时酌情保留原文。中文中很少见到类似小型大写字母的突出显示方式。因此,除了特意展示西文的部分外,本书忽略小型大写字母格式,必要时以其他风格取代。
}。

\paragraph*{字重(graisse)}指字体笔画的粗细(\LaTeX 中以\textit{serie}指代)。默认情况下有两种粗细:中等和\textbf{加粗}。

\paragraph*{字号}字{\large 体}{\Large 的}{\LARGE 大}{\small 小}。

\subsection{族--风格--字重}

有两种不同的宏可以设置族、风格、字重这三个变量:\textit{指令}和\textit{声明}(如表\ref{tab:2.1}所示)。指令以花括号的形式将其参数括住。声明可以打断行文,同时修改三个变量之一,直到新的命令出现。总体上的规则是,我们使用指令来突出显示一个词或一组词:

\begin{table}
    \centering
    \begin{tabular}{|c|l|c|}
\hline
指令 & 声明 & 输出\\
\hline
\verb+\textrm{...}+ & \verb+{\rmfamily ...}+ & 罗马(roman) \\
\verb+\textsf{...}+ & \verb+{\sffamily ...}+ & \textsf{非衬线(sans sérif)} \\
\verb+\texttt{...}+ & \verb+{\ttfamily ...}+ & \texttt{打字机(machine à écrire)} \\
\hline
\verb+\textup{...}+ & \verb+{\upshape ...}+ & 正(droit) \\
\verb+\textit{...}+ & \verb+{\itshape ...}+ & \textit{意大利(italique)} \\
\verb+\textsl{...}+ & \verb+{\slshape ...}+ & \textsl{倾斜(penché)} \\
\verb+\textsc{...}+ & \verb+{\scshape ...}+ & \textsc{小型大写(Petites Capitales)} \\
\hline
\verb+\textmd{...}+ & \verb+{\mdseries ...}+ & 中等(médium) \\
\verb+\textbf{...}+ & \verb+{\bfseries ...}+ & \textbf{加粗(gras)} \\
\hline
    \end{tabular}
    \caption{更改字体的声明}
    \label{tab:2.1}
\end{table}

\begin{codelist}[2.1]{
    \texttt{char}类型的\emph{变量}\textsl{总是}被编码为\textbf{8 位}。
}
\begin{verbatim}
\texttt{char}类型的\emph{变量}\textsl{总是}被编码为\textbf{8 位}。
\end{verbatim}
\end{codelist}

注意,在上面的命令中,指令\verb|\emph|(对应的声明是\verb|\em|,可以以更优雅的方式突出显示一组词)。相较声明,强烈建议使用\emph{指令}。当要修改文本的一部分时,使用指令是更明智的选择\jz{定义指令也是。}:

\begin{codelist}[2.2]{
    {\em \bfseries 马格马(Magma) \mdseries 的音乐就像一面镜子,每个人都能看到他自己的倒影。}
}
    \begin{verbatim}
{\em \bfseries 马格马(Magma) \mdseries 的音乐
就像一面镜子,每个人都能看到他自己的倒影。}
    \end{verbatim}
\end{codelist}

接下来的例子展示了如何使用组。声明\verb|\slshape|出现在一个组中,因此它只在组内发挥作用。此外,组会继承它外层的组的参数。这样一来,“silence”一词会使用\textsf{非衬线}体(根据外层的组),并且\textsl{倾斜}展示(根据内层的声明):

\begin{codelist}[2.3]{
    \sffamily 在爵士乐中,{\slshape 沉默(silence)\/}永远是正确的。因此,这是一种充满万千可能的音乐。
}\begin{verbatim}\sffamily 在爵士乐中,
{\slshape 沉默(silence)\/}永远是正确的。因此,
这是一种充满万千可能的音乐。
\end{verbatim}
\end{codelist}

\subsection{意大利体校正}

另一个推荐使用指令而不是声明的理由是,与声明不同,指令可以实现\emph{意大利体校正}。所谓意大利体校正,是指在以意大利体显示的字符组后,有必要增加一个间距,使得这组字符不会“碰到”后面的词。这个间距与所涉及的字符有关\yz{此例展示不同情况下\emph{f}和a间距的细微差距,宜保留原文。例句意为:首长永远是对的。}:

\begin{codelist}[2.4]{
    le {\em chef} a toujours raison.\par
    le {\em chef\/} a toujours raison.\par
    le \emph{chef} a toujours raison.\par
}\begin{verbatim}le {\em chef} a toujours raison.\par
le {\em chef\/} a toujours raison.\par
le \emph{chef} a toujours raison.\par
\end{verbatim}
\end{codelist}

我们可以看到,指令\verb|\emph|实现了校正,然而,若要使用声明,则需要明确地借助宏\verb|\/|来实现相同的效果。
\chapter{数学排版}

\begin{epigraphe}{《圣经·马太福音》10:2}
  这十二使徒的名:\\头一个叫西门,又称彼得……
\end{epigraphe}

毫无疑问,\LaTeX 最实用和有趣\jz{
    没错,没错!真的有人排公式纯是为了玩!
}的特性就是可以生成数学公式。它生成的公式自然、美观,并且不需要你做任何工作\jz{
    或者只需要你去做两三件小事情。
}。另外,如果你有使用关于某个特定的公式编辑器点来点去的糟糕记忆,现在就偷着乐吧:现在,编写公式不需要鼠标了!使用\LaTeX 生成公式是一个广大的领域,我们这里仅仅会介绍一些用于生成“常用”公式所需的基本知识。因此,本章仅仅包含操作\LaTeX 公式的简短介绍。

\begin{ii}
\LaTeX 的标准指令足以生成大多数常见的数学方程。然而,建议使用美国数学学会(英:American Mathematical Society)发布的扩展amsmath和amssymb。可以在很多情况下,这两个扩展可以简化格式化过程。
\end{ii}

\section{编写数学公式的两种方式}

\LaTeX 可以识别两种数学公式。第一种是在文本中直接插入公式,就像这样:$ax+b=c$;另一种是将若干公式写在环境中,例如:
$$
{\rm d} U = \delta \mathcal{W} +\delta \mathcal{Q} 
$$

这两种模式都遵循一系列原则,涉及不同符号的字号和位置。如下示例使用了两种模式:

\begin{codelist}[3.1]{
  函数$f(x)$定义如下:
\begin{displaymath}
  f(x)=\sqrt{\frac{x-1}{x+1}}
\end{displaymath}
若其导函数存在,求其导函数。
}
\begin{verbatim}
函数$f(x)$定义如下:
\begin{displaymath}
  f(x)=\sqrt{\frac{x-1}{x+1}}
\end{displaymath}
若其导函数存在,求其导函数。\end{verbatim}
\end{codelist}

这个示例告诉我们,我们可以使用\dm{\$}符号来进入“内部”数学模式,并再次使用\dm{\$}符号退出。此外,这里使用了环境\dm{displaymath},这是最简单的生成数学式的方法。使用\verb|\[|和\verb|\]|也可以达成后者的效果(参见3.7.1节。)

3.7节会介绍\LaTeX 的不同环境。

\section{常用指令}

\subsection{上标和下标}

正如1.4.1小节提到的,指令\verb|_|和\verb|^|分别可以生成\emph{下标}和\emph{上标}。若需要这两条指令处理多个字符,需要将这些参数“打包”到一组花括号中。

\begin{center}
  \begin{tabular}{lc|lc|lc}
    \verb|x_2| & $x_2$ & \verb|x_{2y}| & $x_{2y}$ & \verb|x_{t_0}| & $x_{t_0}$ \\
    \hline
    \verb|x^2| & $x^2$ & \verb|x^{2y}| & $x^{2y}$ & \verb|x_{t^0}| & $x_{t^0}$ \\
    \hline
    & &  \verb|x^{2y}_{t_0}| & $x^{2y}_{t_0}$ & \verb|x_{t^1}^{2y}| & $x_{t^1}^{2y}$
  \end{tabular}
\end{center}

\subsection{分式和根式}

生成\emph{分式}和\emph{根式}的指令如下:

\begin{itemize}
  \item 指令\verb|\frac{|\codereplace{分子}\verb|}{|\codereplace{分母}\verb|}|可以生成分式,\codereplace{分子}会排在分数线上方,\codereplace{分母}会排在分数线下方;
  \item 指令\verb|\sqrt[|\codereplace{n}\verb|]{|\codereplace{arg}\verb|}|可以生成分式,表示变量\codereplace{arg}的\codereplace{n}次方根。
\end{itemize}

注意,这两种指令在字间模式和方程模式下生成的效果不同。对于分式$\frac{1}{\sin x + 1}$和根式$\sqrt{3x^2-1}$,他们在方程模式下的显示效果为:
\begin{displaymath}
  \frac{1}{\sin x + 1}\quad \sqrt{3x^2-1}
\end{displaymath}

作为介绍这两条指令的结尾,我们来看看它们是如何套用的:

\begin{codelist}[3.2]{
  \begin{displaymath}
    \sqrt{\frac{1+\sqrt[3]{3x+1}}
              {3x+\frac{1-x}{1+x}}}
  \end{displaymath}
}
\begin{verbatim}
\begin{displaymath}
  \sqrt{\frac{1+\sqrt[3]{3x+1}}
             {3x+\frac{1-x}{1+x}}}
\end{displaymath}\end{verbatim}
\end{codelist}

\subsection{符号}

\subsubsection{常用符号}

表\ref{tab:3.1}展示了部分生成你可能需要的符号的宏。

\begin{table}[hbt]
  \centering
  \begin{tabular}{cccccccc}
    \verb+\pm+       & $\pm$  & \verb+\otimes+      &  $\otimes$ & 
    \verb+\cong+     & $\cong$  & \verb+\imath+     &  $\imath$     \\
    \verb+\mp+       & $\mp$  & \verb+\oslash+      &  $\oslash$ &  
    \verb+\subset+   & $\subset$  & \verb+\jmath+   &  $\jmath$     \\
    \verb+\div+      & $\div$  & \verb+\odot+       &  $\odot$     & 
    \verb+\supset+   & $\supset$  & \verb+\ell+     &  $\ell$         \\
    \verb+\ast+      & $\ast$  & \verb+\leq+        &  $\leq$       & 
    \verb+\subseteq+ & $\subseteq$  & \verb+\aleph+ &  $\aleph$     \\
    \verb+\times+    & $\times$  & \verb+\geq+      &  $\geq$       & 
    \verb+\supseteq+ & $\supseteq$  & \verb+\nabla+ &  $\nabla$     \\
    \verb+\bullets+  & $\bullet$  & \verb+\equiv+   &  $\equiv$   & 
    \verb+\in+       & $\in$      & \verb+\|+       &  $\|$         \\
    \verb+\circ+     & $\circ$  & \verb+\ll+        &  $\ll$         & 
    \verb+\ni+       & $\ni$    & \verb+\partial+   &  $\partial$  \\
    \verb+\star+     & $\star$  & \verb+\gg+     &  $\gg$         & 
    \verb+\emptyset+ & $\emptyset$  & \verb+\wedge+ &  $\wedge$     \\
    \verb+\setminus+ & \backslash & \verb+\sim+ &  $\sim$       & %setminus与unicode-math包冲突,会显示空白
    \verb+\forall+   & $\forall$  & \verb+\vee+   &  $\vee$         \\
    \verb+\oplus+    & $\oplus$  & \verb+\simeq+    &  $\simeq$   & 
    \verb+\infty+    & $\infty$  & \verb+\cup+    &  $\cup$         \\
    \verb+\ominus+   & $\ominus$  & \verb+\approx+   &  $\approx$ & 
    \verb+\exists+   & $\exists$  & \verb+\cap+   &  $\cap$   
  \end{tabular}
  \caption{常用数学符号}
  \label{tab:3.1}
\end{table}

\begin{ii}
我们盘点了latexsym和amssymb包中的近450个符号(参见附录C)%TODO
。目的不是介绍它们!表\ref{tab:3.1}是标准符号中的一部分,我们认为它们可能是最常用的那一批——除了完全偶然出现的$\aleph$\yz{
  aleph,希伯来文字母表的第一个字母。
}。也许这证明了作者的数学水平不太高。
\end{ii}

\subsubsection{省略号}

为了节省篇幅,数学式中经常使用省略号。省略号有三种,指令\verb|\dots|可以生成点“放置在”基线上的省略号:

\begin{codelist}[3.3]{
$C=\{\vec{c}_0,\vec{c}_1,\dots,
    \vec{c}_N\}$
为$N$个颜色的集合。
}
\begin{verbatim}
$C=\{\vec{c}_0,\vec{c}_1,\dots,
    \vec{c}_N\}$
为$N$个颜色的集合。\end{verbatim}
\end{codelist}

指令\verb|\cdots|生成的省略号圆点上下居中,就像等号一样:

\begin{codelist}[3.4]{
  $\vec{\mu}=\frac{1}{N}
(\vec{c}_0+\vec{c}_1+\cdots+\vec{c}_N)$
为$N$个颜色的平均值。
}
\begin{verbatim}
$\vec{\mu}=\frac{1}{N}
(\vec{c}_0+\vec{c}_1+\cdots+\vec{c}_N)$
为$N$个颜色的平均值。\end{verbatim}
\end{codelist}

最后,指令\verb|\vdots|和\verb|\ddots|主要在矩阵中使用(参见3.6节、例3.15)。这两个指令分别可以生成$\vdots$和$\ddots$这两种省略号。

\subsubsection{箭头}

用于生成箭头的指令可以使用以下简单的方法来记忆:

\begin{itemize}
  \item 所有指令均以\dm{arrow}结尾;
  \item 必须带有前缀\dm{left}或\dm{right},表示方向;
  \item 可以带有前缀\dm{long},表示加长;
  \item 指令的第一个字母可以改为大写,表示箭头使用双线;
  \item 可以连写\dm{left}和\dm{right},表示双向箭头。
\end{itemize}

综上:

\begin{center}
  \begin{tabular}{lll|lll}
    \verb+\rightarrow+ & 表示  &$\rightarrow$ &
    \verb+\Longleftarrow+ & 表示 & $\Longleftarrow$ \\
    \verb+\Leftarrow+ & 表示 &$\Leftarrow$ &
    \verb+\Longleftrightarrow+ & 表示 &$\Longleftrightarrow$
  \end{tabular}
\end{center}

\subsubsection{希腊字母}

可以以一种极简单的方式使用希腊字母:打出它们的名字。也就是说,\verb|\alpha|表示$\alpha $,\verb|\pi|表示$\pi $。将指令的第一个字母改为大写,表示将对应希腊字母改为大写:\verb|\Gamma|表示$\Gamma $。注意,不是所有大写希腊字母都有对应的指令,如果要将$\alpha $改为大写,直接使用字母A即可(指令\verb|\Alpha|不存在)。

\subsubsection{实数集}

科技文档的作者常常会面临一个“至关重要”的问题:“我们应当如何打出代表实数集的字母`R'?”关于这个问题,这里分享几个观点。从历史上看,似乎最初的数学资料上将实数符号排版为加粗的形式(“令$x \in \mathbf{R}$”),老师们会使用粉笔反复在字母“R”上描几遍,来代表这个符号。这种比较烦琐的方法促成了我们“更熟悉”的写法:“令$x \in \mathbbm{R}$”。因此,出现了不同的流派:$\mathbf{R}$、ℝ,等等。如果你想自己选择,那么有如下包和指令供你选择:

\begin{itemize}
  \item \textsf{bbm}提供的指令\verb|\mathbbm{R}|可以生成$\mathbbm{R}$、指令\verb|\mathbbmss{R}|可以生成$\mathbbmss{R}$,等等;
  \item \textsf{bbold}提供的指令\verb|\mathbbm{R}|可以生成$\mathbbm{R}$;
  \item \textsf{amssymb}提供的指令\verb|\mathbb{R}|可以生成ℝ、指令\verb|\mathbf{R}|可以生成$\mathbf{R}$。
\end{itemize}

\section{函数}

\subsection{标准函数}

要生成经典的数学函数(如对数函数、三角函数等),需要使用\LaTeX 预装的函数来实现,这里是一个示例:

\begin{codelist}[3.5]{
$\sin^2x + \cos^2 x=1$
}
\begin{verbatim}
$\sin^2x + \cos^2 x=1$\end{verbatim}
\end{codelist}

如果不使用\LaTeX 函数:

\begin{codelist}[3.6]{
$sin^2x + cos^2x=1$
}
\begin{verbatim}
$sin^2x + cos^2x=1$\end{verbatim}
\end{codelist}

二者的区别在于,\LaTeX 会将字符串\dm{cos}视为一系列变量(因此生成意大利体),而将函数\verb|\cos|生成为罗马体的“cos”。另一个区别是对可能存在的下标的处理(参见以下\verb|\max|的示例)。以下函数都是\LaTeX 的标准数学函数。

\begin{itemize}
  \item 各种三角函数:\verb|\sin|、\verb|\cos|、\verb|\tan|。在前面加\dm{arc},可以得到对应的反函数。在后面加\dm{h},可以得到双曲三角函数。
  \item 自然对数和常用对数\yz{
    指以10为底的对数,标准的写法为$\lg$。本书以原书习惯为准,约定使用$\log$。
  }分别使用函数\verb|\ln|和\verb|\log|。
  \item 函数\verb|\sup|、\verb|\inf|、\verb|\max|、\verb|\min|、\verb|\arg|可以用于如下形式的数学式中:
  
  \begin{codelist}[3.7]{
    \begin{displaymath}
      T=\arg \max_{t<0} f(t)
    \end{displaymath}
  }
\begin{verbatim}
\begin{displaymath}
  T=\arg \max_{t<0} f(t)
\end{displaymath}
  \end{verbatim}
  \end{codelist}
\end{itemize}

注意搭配\verb|\max|使用下标操作符\dm{\_}的结果。

\subsection{积分、求和和其他极限}

\LaTeX 使用一套简单的语法来生成\emph{积分}、\emph{求和}等内容,具体如下:

\begin{dmd}
\backslash \codereplace{操作}\verb|_{|\codereplace{下界}\verb|}^{|\codereplace{上界}\verb|}|
\end{dmd}

其中\codereplace{操作}可以是\dm{sum}、\dm{prod}、\dm{int}、\dm{lim}之一,\codereplace{上界}和\codereplace{下界}会排列在操作符号的周围。例如:

\begin{codelist}[3.8]{
  对此等比数列求和:
\begin{displaymath}
  \sum_{i=0}^{n}q^i=
  \frac{\quad 1-q^{n+1}}{1-q}
\end{displaymath}
}
\begin{verbatim}
  对此等比数列求和:
\begin{displaymath}
  \sum_{i=0}^{n}q^i=
  \frac{1-q^{n+1}}{1-q}
\end{displaymath}\end{verbatim}
\end{codelist}

类似地,使用指令\verb|\prod|可以生成求积符号$\prod$。
以下是使用积分的示例:

\begin{codelist}[3.9]{
对于$x>0$,定义自然对数如下:
\begin{displaymath}
  \ln(x)=\int_{1}^{x}\frac{1}{t}
  \,\mathrm{dt}
\end{displaymath}
}
\begin{verbatim}
对于$x>0$,定义自然对数如下:
\begin{displaymath}
  \ln(x)=\int_{1}^{x}\frac{1}{t}
  \,\mathrm{dt}
\end{displaymath}\end{verbatim}
\end{codelist}

指令\verb|\,|可以在“dt”前插入很小的空格(参见3.5.1小节)。你如果更喜欢\emph{线积分},可以使用\verb|\oint|,这个指令可以生成符号$\oint$。好了,这里会给出一个关于极限的示例,相信你可以看懂:

\begin{codelist}[3.10]{
  $f(x)$在$x_0$处存在极限$\ell$:
\begin{displaymath}
  \lim_{x\rightarrow x_0}f(x)=\ell
\end{displaymath}
}
\begin{verbatim}
$f(x)$在$x_0$处存在极限$\ell$:
\begin{displaymath}
  \lim_{x\rightarrow x_0}f(x)=\ell
\end{displaymath}\end{verbatim}
\end{codelist}

希望你能注意到示例中漂亮的$\ell$。为了巩固一下关于两种数学模式的知识,这里给出相同的数学式,但它们这次会镶嵌在行文中:求和,$\sum_{i=0}^{n}q^i=\frac{1-q^{n+1}}{1-q}$;求积分,$\ln(x)=\int_{1}^{x}\frac{1}{t}\,\mathrm{dt}$;求极限,$\lim_{x\rightarrow x_0}f(x)=\ell$。

\section{重叠的符号}

\subsection{操作符\dm{not}}

操作符\verb|\not|可以生成特定关系的“否定”样式:

\begin{codelist}[3.11]{
  令实数$x \not\in I$……
}
\begin{verbatim}
  令实数$x \not\in I$……\end{verbatim}
\end{codelist}

\verb|\not|的输出结果就是在其下一个符号上加上一道“斜杠”。\textbf{注意},这个操作符并不能总是呈现出完美的效果,例如\verb|$\not\longrightarrow$|会显示为$\not\longrightarrow$。但对于宽度合适的符号,它给出的结果还能令人满意。

\subsection{“变音符号”}

对于特殊的数学概念,经常需要\jz{
  实际上,一些名副其实的大数学家很喜欢这种符号上面的小帽子。一些人甚至还喜欢在上面叠两层、三层……
}在符号上加“变音”符号。以下是可用的符号示例:

\begin{center}
  \begin{tabular}{lc@{\quad}lc@{\quad}lc}
    \verb+\hat{x}+  &$\hat{x}$  & \verb+\check{x}+&$\check{x}$&
    \verb+\breve{x}+&$\breve{x}$\\
    \verb+\acute{x}+&$\acute{x}$& \verb+\grave{x}+&$\grave{x}$&
    \verb+\tilde{x}+&$\tilde{x}$\\ 
    \verb+\bar{x}+  &$\bar{x}$  & \verb+\dot{x}+&$\dot{x}$&
    \verb+\ddot{x}+ &$\ddot{x}$\\  
  \end{tabular}
\end{center}


\subsection{向量}

有两种\jz{
由埃迪·索德雷(Eddie Saudrais)开发的包\textsf{esvect}可以为向量生成更好看的箭头。
}方式可以得到向量:

\begin{itemize}
  \item 对于小些的符号,可以使用\verb|\vec|,因为这个指令是用于添加“变音”符号的。
  \item 对于其他情况,可以使用\verb|\overrightarrow|。
\end{itemize}

\begin{codelist}[3.12]{
设$\overrightarrow{A\!B}$在基底
$(\vec{\imath},\vec{\jmath})$
下定义。
}
\begin{verbatim}
设$\overrightarrow{A\!B}$在基底
$(\vec{\imath},\vec{\jmath})$
下定义。\end{verbatim}
\end{codelist}

注意,\verb|$\vec{A\!B}$|会显示为$\vec{A\!B}$(关于\verb|\!|的用途,参见3.5.1小节)。此外,指令\verb|\imath|和\verb|\jmath|分别可以生成不带点的字母i和j:$\imath$、$\jmath$。

\subsection{指令\dm{stackrel}}

指令\verb|\stackrel|可以将两个符号叠放在一起:

\begin{dmd}
\verb|\stackrel{|\codereplace{符号$_1$}\}\{\codereplace{符号$_2$}\}
\end{dmd}

\codereplace{符号$_1$}会置于\codereplace{符号$_2$}上方。例如:

\begin{dmd}
\verb|x\stackrel{f}{\longmapsto}y|
\end{dmd}

以上代码会生成$x\stackrel{f}{\longmapsto}y$。

\section{两个重要原则}

为了掌握\LaTeX 生成数学式的方法,需要知道以下两个原则。

\begin{description}
  \item[空格] \LaTeX 会忽略数学式中夹带的空格,因此\verb|$x+1$|和\verb|$x + 1$|生成的结果是相同的。\LaTeX 会在它认为最合适的地方添加空格。
  \item[文本] 任何的符号组都会被当作同一系列变量或函数对待,因此\verb|$x=t avec t>0$|\yz{
    此问题几乎只在西文排版中出现,因此保留原文。avec可理解为“且其中”。
  }会生成“$x=t avec t>0$”,而不是你所期待的“$x=t$ avec $t>0$”
  \jz{
    数学式中夹带文本的问题只会在使用\dm{displaymath}系列的环境时显现出来。毕竟,使用“\dm{\$x=t\$ avec \$t>0\$}”总是可以的!
  }。
\end{description}

在了解了两个原则后,来看看入门如何处理相关的问题。

\subsection{数学模式的空格}

首先需要知道,\LaTeX 选择添加空格的方式一般是正确的。然而,如果有一天你非要去\linebreak\rlap{ㄨㄨㄨㄨ}吹毛求疵,表\ref{tab:3.2}可以帮助你在数学式中插入空格。在该表格中,我们在两个$\Box$符号之间夹入不同的空格指令,来展示它们的效果。

\begin{table}[ht]
  \begin{center}
    \begin{tabular}{|ll|ll|ll|ll|}
      \hline
      \verb+\!+ & $\Box\!\Box$ &
      \emph{无指令} & $\Box\Box$ &
      \verb+\,+ & $\Box\,\Box$ &
      \verb+\:+ & $\Box\:\Box$ \\
      \hline
      \verb+\;+ & $\Box\;\Box$ &
      \verb*+\ + & $\Box\ \Box$ &
      \verb|\quad| & $\Box\quad\Box$ &
      \verb|\qquad| & $\Box\qquad\Box$ \\
      \hline
    \end{tabular}
    \caption{数学模式中的空格}
    \label{tab:3.2}
  \end{center}
\end{table}

对于那些关注“毛”“疵”的人,要强调一下,本书在等比数列的示例(参见3.3.2小节)中偷偷了在分子上添加了一些空格,以让分式中的两个$q$稍微对齐。如果按照默认的生成方式,结果会是这样的:

\begin{displaymath}
  \sum_{i=0}^{n}q^i=
  \frac{1-q^{n+1}}{1-q}
\end{displaymath}

不知道这个关于$q$的故事是否为你带来了更敏锐的观察力。

\subsection{数学模式中的文本}

在数学式中插入文本,最简单的方法是将文本“装箱”,并适当地插入空格:

\begin{codelist}[3.13]{
设数列$(u_n),(v_n)$:
\begin{displaymath}
  u_n=\ln n\quad
  \mbox{且}\quad v_n=(1+\frac{1}{n})^n
  \label{ex-maths-suite}
\end{displaymath}
}
\begin{verbatim}
设数列$(u_n),(v_n)$:
\begin{displaymath}
  u_n=\ln n\quad
  \mbox{且}\quad v_n=(1+\frac{1}{n})^n
  \label{ex-maths-suite}
\end{displaymath}\end{verbatim}
\end{codelist}

你可以在4.4.1小节找到关于指令\verb|\mbox|的细节。如果你已经在考虑使用包\textsf{amsmath},相比于使用\verb|\mbox|,也可以考虑使用指令\verb|\text|。

\section{阵列(array):简单且高效}

阵列环境\dm{array}可以满足生成大多数数学式的需求。正如其名,它可以将对象排列成一行行、一列列的样子。实际上,它和环境\dm{tabular}对应。也正如\dm{tabular}一样,\dm{array}也不会换行。

\subsection{阵列的原理}

关于阵列环境的语法,可以回忆一下\dm{tabular},有:

\begin{dmd}
\verb|\begin{array}[|\codereplace{垂直位置}]\{\codereplace{格式}\verb|} ... \end{array}|
\end{dmd}

其中,\codereplace{格式}指明各列的对齐方式:\dm{c}表示居中,\dm{l}表示左对齐,\dm{r}表示右对齐。可选的参数\codereplace{垂直位置}可以明确整个表格的垂直位置\celan{\S 2.2.4}。与表格中相同,我们使用以下指令:

\begin{itemize}
  \item 使用\verb|&|分隔不同列;
  \item 使用\verb|\\|换行。
\end{itemize}

\begin{codelist}[3.14]{
  设$A=\begin{array}{rc}
    -1&1\\
    3&4
  \end{array}$为数字阵列……
}
\begin{verbatim}
设$A=\begin{array}{rc}
  -1&1\\
  3&4
\end{array}$为数字阵列……\end{verbatim}
\end{codelist}

以下示例使用了省略号:

\begin{codelist}[3.15]{
  \begin{displaymath} A=\left[\begin{array}{ccc}
    a_{00} & \dots & a_{0n}\\
    \vdots & \ddots & \vdots\\
    a_{n0} & \dots & a_{nn}
  \end{array}\right]\end{displaymath}
}
\begin{verbatim}
\begin{displaymath} A=\left[\begin{array}{ccc}
  a_{00} & \dots & a_{0n}\\
  \vdots & \ddots & \vdots\\
  a_{n0} & \dots & a_{nn}
\end{array}\right]\end{displaymath}\end{verbatim}
\end{codelist}

\subsection{阵列和定界符号}

我们经常需要\dm{array}来生成矩阵,这需要借助\emph{定界符号}的辅助。定界符号是一类特殊的括号,可能是方括号、花括号等,可以将数学对象包裹其中。其语法如下:

\begin{dmd}
\verb|\left|\codereplace{定界$_1$} \codereplace{对象} \verb|\right|\codereplace{定界$_2$}
\end{dmd}

其中,\codereplace{定界$_1$}和\codereplace{定界$_2$}为定界符号,\codereplace{对象}为其包裹的数学对象。

较常用的定界符号如下:

\begin{center}
  \begin{tabular}{cccc}
    \verb+(+ 和 \verb+)+               & $(\Pi)$             &
    \verb+[+ 和 \verb+]+               & $[\Pi]$             \\
    \verb+\{+ 和 \verb+\}+             & $\{\Pi\}$           &
    \verb|\lfloor| 和 \verb|\rfloor| & $\lfloor\Pi\rfloor$ \\
    \verb|\lceil| 和 \verb|\rceil|   & $\lceil\Pi\rceil$   &
    \verb|\langle| 和 \verb|\rangle| & $\langle\Pi\rangle$  \\
    \verb+|+                           & $|\Pi|$             &
    \verb+\|+                         & $\|\Pi\|$           
  \end{tabular}
\end{center}

使用定界符号的好处是,这种符号可以自动适应它包裹的对象的尺寸:

\begin{codelist}[3.16]{
  设$I=
\left[\begin{array}{cc}
  1&0\\0&1
\end{array}\right]$
为单位矩阵。
}
\begin{verbatim}
设$I=
\left[\begin{array}{cc}
  1&0\\0&1
\end{array}\right]$
为单位矩阵。\end{verbatim}
\end{codelist}

我们同样可以使用定界符号重写示例3.13,来改变括号的尺寸:

\begin{codelist}[3.17]{
  设数列$(u_n),(v_n)$:
\begin{displaymath}
  u_n=\ln n\quad\mbox{et}
  \quad v_n=\left(1+\frac{1}{n}\right)^n
\end{displaymath}
}
\begin{verbatim}
设数列$(u_n),(v_n)$:
\begin{displaymath}
  u_n=\ln n\quad\mbox{et}
  \quad v_n=\left(1+\frac{1}{n}\right)^n
\end{displaymath}\end{verbatim}
\end{codelist}

\begin{exclamation}
对于每一个指令\verb|\left|,都应该有一个指令\verb|\right|与其对应。然而,左侧和右侧分别使用的符号不必是配套的。
\end{exclamation}

以下示例中使用了\verb|\right.|,表示我们不需要右侧的符号:

\begin{codelist}[3.18]{
  设$ S_i=\left\{\begin{array}{rl}
    -1  & \mbox{若$i$为偶数,}    \\
    1  & \mbox{否则。}
  \end{array}\right.$
}
\begin{verbatim}
设$ S_i=\left\{\begin{array}{rl}
  -1  & \mbox{若$i$为偶数,}    \\
  1  & \mbox{s否则。}
\end{array}\right.$\end{verbatim}
\end{codelist}

\subsection{说话的方式简单点……}

包\textsf{amsmath}中提供了两种环境——\verb|pmatrix|(p代表parenthèse,即圆括号)和\verb|bmatrix|(b代表英文的\emph{braket},即方括号),可以简单地插入矩阵:

\begin{codelist}[3.19]{
  \begin{displaymath}
    \bar{\bar{\sigma}}=\begin{bmatrix}
      \sigma_{11} & \sigma_{12} \\
      \sigma_{21} & \sigma_{22} \\
    \end{bmatrix}
  \end{displaymath}
}
\begin{verbatim}
\begin{displaymath}
  \bar{\bar{\sigma}}=\begin{bmatrix}
    \sigma_{11} & \sigma_{12} \\
    \sigma_{21} & \sigma_{22} \\
  \end{bmatrix}
\end{displaymath}\end{verbatim}
\end{codelist}

\section{方程和环境}

本节会介绍\LaTeX 中可以生成数学式的三种标准环境。

\subsection{环境\dm{displaymath}}

你如果已经阅读到这里,应该已经明白,\dm{displaymath}可以打断当前段落,并居中显示一行公式。\verb|\begin{displaymath}...\end{displaymath}|的一种简略的写法是:
\verb|\[...\]|。例如:

\begin{codelist}[3.20]{
  色度距离:\[
  \Delta E=\sqrt{
   \Delta L^{*2}+ \Delta a^{*2}
  +\Delta b^{*2}} \]
}
\begin{verbatim}
色度距离:\[
  \Delta E=\sqrt{
   \Delta L^{*2}+ \Delta a^{*2}
  +\Delta b^{*2}} \]\end{verbatim}
\end{codelist}

\subsection{方程环境\dm{equation}}

方程环境\dm{equation}的作用与\dm{displaymath}相同,但可以为数学式编号:

\begin{codelist}[3.21]{
  请牢记,若$a>0$且$b>0$,有
\begin{equation}
  \ln(ab)=\ln(a)+\ln(b)
\end{equation}
}
\begin{verbatim}
请牢记,若$a>0$且$b>0$,有
\begin{equation}
  \ln(ab)=\ln(a)+\ln(b)
\end{equation}\end{verbatim}
\end{codelist}

\begin{ii}
文档类中的选项\dm{leqno}可以将方程的编号放在左侧。方程环境的选项\dm{fleqn}可以将方程居左,而非居中显示。
\end{ii}

\subsection{多行数学式}

\begin{ii}
在本书的某个旧版本中,作为介绍标准环境的结尾,我们曾引入环境\dm{eqnarray}。这个环境可以生成多行数学式,但\textbf{要知道这是错误的}。关于这个话题,有一些资料(如[12]或[14])会向你解释如何生成“正确”的文档。要坚定不移地相信,使用\dm{eqnarray}(以及很多其他工具)\textbf{是在造孽}。无论如何,如果你没有禁住诱惑而向\dm{eqnarray}让步,那么终有一天,审判会顺着搜索引擎降临,任何忏悔都将无济于事,任何赦免都将无法拯救你。勿谓言之不预也。
\end{ii}

我们在此介绍包\textsf{amsmath}中的环境\dm{align}:

\begin{itemize}
  \item 使用\verb|\\|换行;
  \item 每行数学式都会编号,除非指令\verb|\nonumber|出现在该行中;
  \item 使用两个操作符\verb|&|\jz{
    因为共有三列。(\textsl{译注}:原文如此。)
  }来对齐。
\end{itemize}

\begin{codelist}[3.22]{
\begin{align}%TODO align会产生奇怪的空行
  (a+b)^2 & =  (a+b)(a+b)\nonumber\\
          & =  a^2+b^2+2ab
  \end{align}
}
\begin{verbatim}
\begin{align}
  (a+b)^2 & =  (a+b)(a+b)\nonumber\\
          & =  a^2+b^2+2ab
\end{align}\end{verbatim}
\end{codelist}

\begin{ii}
带有星号的环境\dm{align*}可以使所有行都不编号。要想引用环境\dm{align}中的多行,需要插入同样数量的\verb|\label|,分别对应相应行。
\end{ii}

若要为占用多行的方程编号,可以使用环境\dm{split}(同样由\textsf{amsmath}提供):

\begin{codelist}[3.23]{
  \begin{equation}
    \begin{split}
      (a+b)^2 & =  (a+b)(a+b)\\
      & =  a^2+b^2+2ab
    \end{split}
  \end{equation}
}
\begin{verbatim}
\begin{equation}
  \begin{split}
    (a+b)^2 & =  (a+b)(a+b)\\
    & =  a^2+b^2+2ab
  \end{split}
\end{equation}\end{verbatim}
\end{codelist}

\section{数学模式的风格}

\subsection{字体}

\LaTeX 支持多种用于在数学模式中切换字体的指令\yz{
  本节疑似原书作者的包和译稿使用的xeCJK都有冲突,仅针对使用法文书写的环境。此小节暂时按原文誊写。% todo conflict
}。默认情况下,所有符号或字符序列(除了特定函数)在最终文档中都会以意大利体呈现。在一些情况下,强制切换字体风格会很有用。实现方法具体如下:

\begin{center}
  \begin{tabular}{ll}
    \verb|设$\mathit{A\in\Phi}$| & 设$\mathit{A\in Φ}$\\
    \verb|设$\mathrm{A\in\Phi}$| & 设$\mathrm{A\in Φ}$\\
    \verb|设$\mathbf{A\in\Phi}$| & 设$\mathbf{A\in Φ}$\\
    \verb|设$\mathsf{A\in\Phi}$| & 设$\mathsf{A\in Φ}$\\
    \verb|设$\mathtt{A\in\Phi}$| & 设$\mathtt{A\in Φ}$\\
    \verb|设$\mathcal{A\in\Phi}$| & 设$\mathcal{A\in\Phi}$\\
  \end{tabular}
\end{center}

\begin{exclamation}
指令\verb|\mathcal|只接受大写拉丁字母作为变量,否则结果会展示为乱码。例如:\\
\verb|\mathcal{abcd\Gamma}|

上述指令的运行结果为$\dashv \lfloor \rfloor \lceil -$。
\end{exclamation}

\subsection{符号的字号}

\LaTeX 会区分四种数学式写作\emph{风格}。\LaTeX 生成数学式时,会根据当前的“处境”选择模式。

\begin{description}
  \item[文本] 适用于行文间插入的数学式。
  \item[方程] 适用于\emph{方程}格式下的数学式。
  \item[角标] 适用于角标。
  \item[子角标] 适用于角标的角标。 
\end{description}

每种模式都可以明确地使用以下声明激活:

\begin{itemize}
  \item 使用\verb|\textstyle|切换文本模式;
  \item 使用\verb|\displaystyle|切换方程模式;
  \item 使用\verb|\scriptstyle|切换角标模式;
  \item 使用\verb|\scriptscriptstyle|切换子角标模式;
\end{itemize}

以下示例阐明了如何在\emph{方程}模式中强制使用\emph{文本}模式,以及相反的操作:

\begin{codelist}[3.24]{
两种形式的积:$\prod_{1}^{n}f_i$
和$\displaystyle\prod_{1}^{n}f_i$
相反操作:
\[ \prod_{1}^{n}f_i
\mbox{ 和 }\textstyle\prod_{1}^{n}f_i \]
}
\begin{verbatim}
两种形式的积:$\prod_{1}^{n}f_i$
和$\displaystyle\prod_{1}^{n}f_i$
相反操作:
\[ \prod_{1}^{n}f_i
\mbox{ 和 }\textstyle\prod_{1}^{n}f_i \]\end{verbatim}
\end{codelist}

\subsection{创建新操作符}

想象在一个场景中,你需要创建一个特殊的操作符,称作“burps”。只需要通过如下形式生成:

\begin{codelist}[3.25]{
\newcommand{\burps}{
\mathop{\textrm{burps}}}
$x=\burps_i f(i)$
}
\begin{verbatim}
\newcommand{\burps}{
\mathop{\textrm{burps}}}
$x=\burps_i f(i)$\end{verbatim}
\end{codelist}

再看一个例子。为了让“反正弦函数”(默认显示为arcsin)依法国的习惯显示,可以按如下示例操作:

\begin{codelist}[3.26]{
  $\theta = \arcsin x$
\renewcommand{\arcsin}{%
  \mathop{\textrm{Arcsin}}\nolimits}
$\theta = \arcsin x$
}
\begin{verbatim}
$\theta = \arcsin x$
\renewcommand{\arcsin}{%
  \mathop{\textrm{Arcsin}}\nolimits}
$\theta = \arcsin x$\end{verbatim}
\end{codelist}

指令\verb|\nolimits|可以使相关操作符不再将参数显示为上标或下标的形式,正如操作符\verb|\lim|、\verb|\int|等所做的那样\yz{
  此句疑似说反了。
}。此外,前文的两个示例使用了指令\verb|\newcommand|和\verb|\renewcommand|,相关问题请参见4.5节。

最后,还有一种方式可以达到类似的目的。如果你已经加载了\textsf{amsmath}包,并在文前部分进行了以下声明:

\begin{dmd}
\verb|\DeclareMathOperator*{\vlunch}{vlunch}|
\verb|\DeclareMathOperator{\zirgl}{Zirgl}|
\end{dmd}

那么可以实现以下操作:

\begin{codelist}[3.27]{
  \[
  x=\mathop{\rm vlunch}\limits_{i} f(\theta)
  \]
  其中$\theta =$ Zirgl $y$。
}
\begin{verbatim}
\[x=\vlunch_i f(\theta)\]
其中$\theta = \zirgl y$。\end{verbatim}
\end{codelist}

\section{小结}

本章介绍了用于生成数学式的基本方程。对于大多数科学文档,这些指令已经足够使用了。如果你不得不起草一份充斥这复杂数学式的文档,仅靠\LaTeX 的宏可能不能满足需求。因此,著名的\emph{美国数学协会(英:American Mathematical Society)}孵化了称为\AmS\TeX 的包(通过\verb|\usepackage{amsmath}|使用),可以生成尤其“奇形怪状”的数学式。
% \begin{appendices}
%     \section{附录}
% \end{appendices}
\end{document}