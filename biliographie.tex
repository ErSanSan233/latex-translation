\chapter*{参考文献}
\addcontentsline{toc}{chapter}{参考文献}

\begin{enumerate}
    \renewcommand{\labelenumi}{[\theenumi]}
    \item the UK List of TEX Frequently Asked Questions on the Web\\
    包含丰富的英文信息,可通过\wz{http://www. tex.ac.uk/cgi-bin/texfaq2html}访问。
    \item Jacques André. Petites leçons de typographie. 1990.\\
    可以在\wz{http://jacques-andre.fr}找到该文档。这是一篇有趣的关于排版的文章,包含了很多关于大写用法、标点符号、下线及法文符号使用的例子。
    \item W. Appel, E. Chevalier, E. Cornet, Desreux S., Fleck J.J., and Pichaureau P.. \LaTeX pour l'impatient. In \emph{Technique et pratique}. H \& K, 2007.
    \item Denis Bitouzé and Jean-Côme Charpentier. \LaTeX . In \emph{Collection Synthex}. Pearson Education France, September 2006.
    \item M. Goossens, S. Rahtz, and F. Mittelbach. \emph{The \LaTeX \ Graphics Companion}. Addison-Wesley, 1997.\\
    由《\LaTeX 伴侣》作者编写,一本关于广义上的图形使用的书,包含使用\LaTeX 绘图的包的探索,以及关于PostScript字体的使用的介绍。
    \item Michel Goossens, Franck Mittelbach, and Alexander Samarin. \emph{The \LaTeX \ companion}. \linebreak Addison-Wesley, 1994.\\
    关于\LaTeXe 和其包的\textbf{唯一}圣经。该书是所有想要理解\LaTeX 内嵌函数的用户的必读书,包含关于以下内容的十分详细的信息:自定义默认版式的方法、字体的使用、大量包,等等。
    \item D. E. Knuth. \emph{The Art of Computer Programming}, volume 1–3. Addison-Wesley, 1997–98.\\
    有关“编程的艺术的”3卷图书。第4卷正在准备阶段。这一套书本科学界认定为本世纪最重要的图书之一(对此,参见\wz{http://www.amsci.org/amsci/bookshelf/ centurylist.html};对于更多关于“TAOCP”的信息,参见关于高德纳的网页\wz{http://www-cs-staff.stanford.edu/\linebreak\~\/knuth/taocp.html})。
    \item Donald E. Knuth. \emph{The \TeX Book. Addison-Wesley}, 1988.\\
    关于\TeX 的\textbf{唯一}圣经。该书充满了“危险的转折”,十分详细第解释了\TeX 的内部机制。这是本相当难读的参考书,并且没有为初学者准备有关\TeX 的介绍——我认为是这样。
    \item Leslie Lamport. \emph{\LaTeX : A Document Preparation System}. Addison-Wesley, 2e edition, 1994.\\
    \LaTeX 的作者写的书,第2版中包含了\LaTeXe 。显然,这是一本很好的入门书。该书结尾带有指令的参考指南。
    \item Vincent Lozano. Tout ce que vous avez toujours voulu savoir sur UNIX sans jamais oser le demander, 2006. \wz{http://www.enise.fr/cours/info/unix}
    \item Lars Madsen. Avoid eqnarray ! . \emph{The Prac\TeX Journal}, (4), 2006.\\
    一篇整理不使用该环境的理由的文章,应当可以在\wz{http://home.imf.au.dk/daleif}找到。
    \item Yves Perrousseaux. \emph{Manuel de typographie française élémentaire}. Atelier Perrousseaux, 1995.\\
    关于排版学的一本由教育意义的“小”书,包含有趣的发展史,以及全世界使用的关于排版学的规则清单。
    \item Mark Trettin. Une liste des péchés des utilisateurs de \LaTeX. 2004.\\
    该文档更著名的名称是“l2tabu”,探讨“过时的指令和扩展,以及一些其他错误”。
\end{enumerate}