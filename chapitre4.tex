\chapter{成为小魔仙}

\begin{quote}
    羔羊揭开第七印的时候,天上寂静约有二刻……——《圣经·启示录》8:1
\end{quote}

在继续探索\LaTeX 这个庞大且出色的系统前,有必要停下脚步来了解一些重要的概念。实际上,为了成为能在组成这个系统的大量文件中大展身手的“赫尔克里·波洛”\yz{
    Hercule Poirot,阿加莎·克里斯蒂所著系列侦探小说中的主角。
},掌握这些概念是基本的要求。在本章,我们介绍有关计数器、长度、空白和字盒的内容。如果你在使用\LaTeX 时,除了顺从地使用它提供给你的格式外,还想要其他版式,这4个概念会很有用。

\begin{exclamation}
    本章涉及的概念有些微妙的细节,比较难以把握\jz{
        作者本人也不能保证掌握了全部内容……
    }。我们建议你实际上手体验一下,因为这里介绍的工具一方面可以为你生成最令人满意的结果,另一方面也能让你掉最多的头发,基本上跟直接在你头上薅差不多。
\end{exclamation}
\section{计数器}

文档中所有带有编号的对象都由\emph{计数器}生成。计数器可以递增、递减,也可以重设为0,等等。我们也可以根据自己的需要来创建计数器。

\subsection{可用的计数器}

计数器主要与标题、页码、浮动环境(环境\dm{figure}和\dm{table})、方程(环境\dm{equation})、页面底部的注释、编号列表(环境\dm{enumerate})息息相关。

表\ref{tab:4.1}列出了\LaTeX 中基础的计数器的名称。可以注意到,它们基本上都由相关对象的名称组成。计数器\dm{enumi}~\dm{enumiv}分别与环境\dm{enumerate}中的第1层~第4层相关。\dm{mpfootnote}是环境\dm{minipage}中脚注的计数器,会在4.4.3小节提到。

\begin{table}
  \centering
  \ttfamily
  \begin{tabular}{|l|l|l|l|}
    \hline
    part       & paragraph    & figure     & enumi\\
    chapter    & subparagraph & table      & enumii\\
    section    & page         & footnote   & enumiii\\
    subsection & equation     & mpfootnote & enumiv\\
    subsubsection &&&\\
    \hline
  \end{tabular}
  \caption{\LaTeX 的计数器}
  \label{tab:4.1}
\end{table}

\subsection{操作}

在接下来的段落中,我们介绍几个操作计数器的基本工具。注意,计数器是\emph{全局}变量,因此以下描述的三个指令也会在全局范围内生效。同时,也需要注意,其中的变量都是\emph{整数}。

\subsubsection{创建计数器}

可以通过以下指令\emph{创建}计数器:

\begin{dmd}
\backslash newcounter\{\codereplace{计数}\}[\codereplace{父计数}]
\end{dmd}

该指令可以创建一个新的计数器,名为\codereplace{计数}。参数\codereplace{父计数}是非强制的,如果配置,则每次\codereplace{父计数}递增时,\codereplace{计数}都会归零。

\subsubsection{指定计数}

可以通过以下方法为计数器指定一个值:

\begin{dmd}
\backslash setcounter\{\codereplace{计数}\}\{\codereplace{值}\}
\end{dmd}

其中,\codereplace{计数}代表我们想要指定值的计数器,\codereplace{值}即具体指定的值。

\subsubsection{增值}

可以通过以下指令使计数器的值增加或减少:

\begin{dmd}
  \backslash addtocounter\{\codereplace{计数}\}\{\codereplace{值}\}
\end{dmd}

其中,若\codereplace{值}为正数(对应地,负数),则可以使计数器的值增加(对应地,减少)。为了展现该指令的效果,我们在本书的文档中添加以下一行指令:

\begin{dmd}
\verb|\addtocounter{footnote}{357}|
\end{dmd}

\addtocounter{footnote}{357}
可以看到页面下方脚注的编号变化\jz{
  虽然这个值变了这么多属实有些荒谬……
}了。为了接下来的脚注恢复正确的序号,我们在源码中插入以下指令:

\begin{dmd}
  \verb|\addtocounter{footnote}{-357}|
\end{dmd}

\addtocounter{footnote}{-357}
可以看到,现在的脚注序号恢复正常\jz{
  不正常我遭天打雷劈!
}了。

\subsection{显示}

使用如下指令可以将计数显示出来:

\begin{dmd}
\backslash the\codereplace{计数器名}
\end{dmd}

实际上,所有可以显示计数器的指令或环境都调用了类似的指令。这样一来,我们有如下指令:

\begin{itemize}
  \item \verb|\thepage|,在此处可以生成“\thepage ”,会在每次换页时调用;
  \item \verb|\thefootnote|,在此处可以生成“\thefootnote ”,会被\verb|\footnote|调用;
  \item \verb|\thesubsection|,在此处可以生成“\thesubsection ”,会被指令\verb|\subsection|调用;
  \item ……
\end{itemize}