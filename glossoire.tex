\chapter{术语字典}

\begin{leglossaire}

\item[编译(compilation)]
尽管从科学的角度上,这个术语不是很严格,但我们将从\LaTeX 源文件翻译为DVI或PDF格式的环节称为编译。

\item[主文件(document maître)]
即在将文档内容拆分为多个文件时,包含\verb+begin{document}+的那个源文档。

\item[源文档(document source)]
包含文本和\LaTeX 指令的文本文档。不要丢失该文档,因为它是制作纸质版、电子版等文件的源头,正如C语言的源文件是可执行程序的源头一样。

\item[DVI]
\emph{设备无关}文件的格式,由克努特制定,目的是从源文档创建一个格式与平台或介质无关的文档。

\item[辅助文件(fichiers auxiliaires)]
编译时由\LaTeX 生成的一批文件。它们的文件名与源文档相同,只不过扩展名由表明其角色的3个字母组成。

\item[格式(format)]
存储在扩展名通常为\dm{.fmt}的文件中的一组指令或宏。其中最著名的一组就是\TeX 和\LaTeX 的格式\dm{plain}。

\item[宏(macro)]
使得将\LaTeX 的复杂内容压缩为简单命令的工具。宏也可以称作指令,正如编程语言中的例程。

\item[PDF]
指\emph{便携文档格式},是由Adobe公司创造的文件格式,其目的是轻松地在一个系统和另一个系统间交换文档。有多种方式可以由\LaTeX 源码创建PDF(参见附录A)。

\item[PostScript]
由Adobe公司定义的语言,目的是描述以印刷的为目的文档。该语言包含低层级的原语,可以由软件解释来在印刷前实现预览,或直接由打印机装载的电路解释来生成要打印的图像。

\item[参考(références)]
以符号形式操作段落、数学式、章等内容的编号的系统,可以避免在修改版式时引入更新所带来的难题。

\item[\LaTeX]
由莱斯利·兰波特在\TeX 之上定义的宏的集合。目前在使用的版本是\LaTeXe。

\item[\TeX]
\LaTeX 作为一组宏所依附的底层引擎。\TeX 在版本3.14159时变得稳定。在发布新版本时,克努特会添加一位小数。

\end{leglossaire}