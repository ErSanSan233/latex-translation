\chapter{新玩具}

\begin{quote}
    我属我的良人,他也恋慕我。我的良人,来吧,你我可以往田间去。你我可以在村庄住宿!我们早晨起来往葡萄园去,看看葡萄发芽开花没有,石榴放蕊没有。我在那里要将我的爱情给你。

    \hfill《圣经·雅歌》7:11
\end{quote}

在本章,我会介绍专为本书创建的工具。为了看懂本章定义的大部分指令和环境,你需要阅读过第9章和第10章的内容……本章中,我们会涉及带有“危险”标记的提示的创建方法、各章开头的首字下沉、摘要、术语字典、带有当前章号的标签,以及并列展示\LaTeX 代码和效果的环境。

\section{一些小手工}

\subsection{参数和排版转换}

在涉及信息语言的文档中,需要突出显示参数、指令,以及函数。例如,我们需要这类效果:

\begin{codelist}[11.1]{
    为了编译文件\codereplace{文件}:
\begin{flushleft}
    \ttfamily latex \codereplace{文件}
\end{flushleft}
}\begin{verbatim}
为了编译文件\bwarg{文件}:
\begin{flushleft}
    \ttfamily latex \bwarg{文件}
\end{flushleft}
\end{verbatim}
\end{codelist}

指令\verb|\bwmarg|可以以\textsl{倾斜字体}来输出其参数,并且置于“⟨”和“⟩”之间。这种括号可以在数学模式下分别由指令\verb|\langle|、\verb|\rangle|生成。此外,你无疑注意到了,我们可以使用下标符号,就像这样:

\begin{codelist}[11.2]{
复制文件:
\begin{flushleft}\ttfamily
    cp \codereplace{文件$_1$} ...
    \codereplace{文件$_n$}
    \codereplace{文件$_{dst}$}
\end{flushleft} 
}\begin{verbatim}
复制文件:
\begin{flushleft}\ttfamily
    cp \bwarg[1]{文件} ...
    \bwarg[n]{文件}
    \bwarg[dst]{文件}
\end{flushleft}
\end{verbatim}
\end{codelist}

指令\verb|\bwarg|\jz{
    这个名称代表“黑白”参数(« black \& white » argument)……
}的定义如下:

\begin{dmd}
\begin{verbatim}
\newcommand{\marg}[2][]{% 
    {\normalfont%
        \textsl{$\langle$#2%
            % 如果选用了可选参数 
            \ifthenelse{\equal{#1}{}}{} 
            {$_\mathit{#1}$}% 显示下标
            $\rangle$}}}%
\end{verbatim}
\end{dmd}

指令\verb|\normalfont|的效果是回到文档的默认字体。这解释了为什么示例11.1中的“\codereplace{文件}”没有使用打字机字体显示。

在本文档的电子版(也就是需要通过屏幕阅读的版本)中,我们决定使用\emph{颜色}而不是字符“⟨”和“⟩”。这样一来,可以定义指令\verb|\colarg|:

\begin{dmd}
\begin{verbatim}
\newcommand{\colarg}[2][]{{% 
    \normalfont\color{blue!90}#2% 蓝色 
    \ifthenelse{\equal{#1}{}}{}{$_\mathit{#1}$}}}
\end{verbatim}
\end{dmd}

接下来,借助一个巧妙放置的布尔值\celan{\S 9.3.1},我们可以定义一个通用的指令\verb|\marg|,以调用其中的一个版本(黑白版本或彩色版本):

\begin{dmd}
\begin{verbatim}
\ifversionenligne
    \let\marg\colarg
\else
    \let\marg\bwarg
\fi
\end{verbatim}
\end{dmd}

这个结构调用了\TeX 的指令\verb|\let|,在9.2.3小节有明确的介绍。

\subsection{关于索引的生成}

本书中,凡是行文中提到指令、环境、包、文档类型等时,都会调用对应的特殊指令,来自动在索引中插入一条入口\yz{
    妈的怎么不早说。算了先不做索引了,全书翻译完之后再调整吧。
}。举例来说,这样一来的效果为:

\begin{codelist}[11.3]{
借助包\textsf{varioref},可以
使用指令\dm{\backslash vref}……
}\begin{verbatim}
借助包\ltxpack{varioref},可以
使用指令\ltxcom{vref}……
\end{verbatim}
\end{codelist}

指令\verb|\ltxpack|的定义如下。首先,以下内容定义的指令\verb|\ltx@pack|可以以非衬线字体展示包名:

\begin{dmd}
\begin{verbatim}
\newcommand{\ltx@pack}[1]{%
    \upshape\textsf{#1}}
\end{verbatim}
\end{dmd}

接下来,我们像这样定义:

\begin{dmd}
\begin{verbatim}
\newcommand{\ltxpack}[1]{%
    \ltx@pack{#1}% 
    \protect\index{扩展们!\protect\texttt{#1}}% 
    \protect\index{#1@\protect\textsf{#1 扩展}}}
\end{verbatim}
\end{dmd}

其中调用了前面定义的指令,并在索引中插入了两个入口,一个遵循“\codereplace{包名}扩展”的格式,另一个作为“扩展们”的\celan{\S 6.3}子入口。这里,指令\verb|\protect|的作用是避免指令\verb|\ltxpack|本身作为作为另一条指令的参数时带来的麻烦。遵循同样的思路,我们可以定义指令\verb|\ltxcom|。首先:

\begin{dmd}
\begin{verbatim}
\newcommand{\ltx@com}[1]{%
    \texttt{\symbol{92}#1}}
\end{verbatim}
\end{dmd}

以上指令可以以打字机字体生成指令名,并在前面加上字符\dm{\backslash}。\verb|\symbol|是\LaTeX 指令,此处用于插入所选字体下的第92个字符(恰好为反斜杠)。因此,我们最终有如下定义:

\begin{dmd}
\begin{verbatim}
\newcommand{\ltxcom}[1]{%
    \ltx@com{#1}% 
    \index{#1@\protect\texttt{\symbol{92}#1}}}
\end{verbatim}
\end{dmd}

该指令调用了上一条指令,在索引中插入一个入口。此处需要提炼出的思想是,定义可以自动在索引中插入入口的指令可能很有帮助。举例来说,我们也许可以定义这样一条指令:

\begin{dmd}
\begin{verbatim}
\newcommand{\jargonanglais}[1]{% 
    \emph{#1}%
    \index{#1}}
\end{verbatim}
\end{dmd}

该指令可以在设置英文术语格式的同时将其插入索引——比如某个特殊的索引。同样,如果在文档中经常出现,我们就可以定义一条指令,将其插到索引中。例如,在本书中,我们有这样的定义:

\begin{dmd}
\begin{verbatim}
\newcommand{\postscript}{% 
    PostScript% 
    \protect\index{PostScript}}
\end{verbatim}
\end{dmd}

\subsection{跳转提示}

本书的纸质版中四处分散着跳转阅读的提示,例如这个提醒你去看\celan{\S D.3.5}术语字典(glossaire)的提示——幸亏我们这里在说跳转阅读,否则这个术语字典就这里的内容毫无关联了\jz{
    如果你没完全跟上的话,我是想说,这里其实没有术语字典的什么事……
}。实现跳转提示符号的指令被赐名为\verb|\voir|,接收两个参数:

\begin{dmd}
\backslash voir\{\codereplace{目标标签}\}\{\codereplace{跳转对象文本}\}
\end{dmd}

例如,前面的跳转是这样生成的:

\begin{dmd}
\verb|\voir{chap-glossaire}{glossaire}|
\end{dmd}

这个指令的设计过程中,全部的“难点”在于怎样让三角形根据页面奇偶性来改变朝向。借助包\textsf{chngpage},这一难点得以功课。请参阅9.3.3小节。

余下的工作就是展示三角形。为了达成这一目的,我们定义了两个指令,分别生成侧栏的跳转提示和正文中的标记。由此,\verb|\voir|的格式如下:

\begin{dmd}
\begin{verbatim}
\newcommand{\voir}[3][\S]{% 
    \checkoddpage% 
    \ifcpoddpage
        \v@irpageimpaire{#1}{#2}{#3}}{% 奇数页的跳转提示
    \else
        \v@irpagepaire{#1}{#2}{#3}}} % 偶数页的跳转提示
    \fi
\end{verbatim}
\end{dmd}%TODO 此处括号似乎不配对

我们注意到,除了两个必需的参数外,该指令还可以接收一个可选参数,它默认定义为段落标记(\S)。指令\verb|\v@irpageimpaire|和\verb|\v@irpagepaire|是对称的,作用如下:

\begin{enumerate}
    \item 将“朝向正确”的三角形放置在文本中,作为跳转提示对象;
    \item 在侧栏生成带有跳转目标的注解。
\end{enumerate}

三角形

