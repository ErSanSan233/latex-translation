\chapter{生成过程的笔记}

我在这里整理了一些内容,它们对于探索本文档的源文件很有用,例如如何编译、使用哪个\LaTeX 发行版、文件的组织方式,等等。

\section{此时的发行版}

本书在发行版Ubuntu 12.04上使用发行版\TeX live 2013编译。我们也欢迎在其他系统上的任何成功经验……此外,对于生成PDF格式的文档,我们只测试了计算机现代字体族的“CM-Super”版本。

\section{文档源文件}

\subsection{结构}

本书的源文件依以下原则组织:

\begin{itemize}
    \item \LaTeX 源代码位于目录\dm{corps}中,每章一个文件;
    \item 风格定义(\dm{sty}及\dm{cls})位于目录\dm{pngs}中;
    \item 目录\dm{texs}包括文档中需要包含的\LaTeX 源代码(信件、传真模板,包含调用Psfrag或Pstricks的代码);%TODO textsf
    \item 目录\dm{figs}包含\textsf{xfig}源文件;
    \item 与索引、参考文献、术语字典有关的内容存储在目录\dm{bibidx}中。
\end{itemize}

\textsf{xfig}源文件和\LaTeX 文件的特定“端”由makefile翻译为PDF格式,无论是否封装PostScript。根据其使用的引擎不同(\dm{latex}或\dm{pdflatex}),makefile将其存储于以下位置:

\begin{itemize}
    \item 目录\dm{epss}中;
    \item 目录\dm{pdfs}中;
    \item 目录\dm{pss}中;
\end{itemize}

\subsection{风格}

文件\dm{framabook.cls}包含了本书文档类型的定义。该文件调用了一系列“商用型”和“自制”包,对于后者,有针对以下内容的文件:

\begin{itemize}
    \item 各种“新玩具”:标签、提示框、摘要、术语字典、带有标题的字盒(\dm{titlebox.sty})、示例、首字下沉、跳转阅读(\dm{voir.sty})、引用,以及题记;
    \item 全书摘要;
    \item 文档的全局尺寸设定;
    \item 页眉和页脚的风格;
    \item 章节等层级的风格;
    \item 文档使用的其余杂项指令(位于文件\dm{manumac.sty}中)。
\end{itemize}

除非另有说明,否则这些文件的命名与其内容高度相似。

\section{编译}

在主文件\dm{framabook.tex}中,我们可以指明文档类型选项:

\begin{itemize}
    \item \dm{versionenligne}可以将生成文件中的超链接显示为彩色;
    \item \dm{versionpapier}致力于生成用于印刷并裁切成书的文件版本。
\end{itemize}

\begin{exclamation}
目前,本书基于而组织对\emph{不同于A4}的纸张尺寸的支持而组织。因此,如果你没有裁切机,生成的文档看起来会挺丑的。可以通过以下网站获取适用于在A4纸上打印的版本:\wz{http://www.enise.fr/cours/info}。
\end{exclamation}

\subsection{Makefile}

源文件目录树的根部包含一个关于本书Makefile文件,应该将其复制:

\dmh{cp Makefile.frama Makefile}

\subsection{图片}

图片可以借助以下指令编译:

\begin{dmd}
make figs
\end{dmd}

\begin{exclamation}
需要部署与xfig有关的软件transfig以将源文件翻译为\dm{pdf}或\dm{eps}。网站\linebreak \wz{http://cours.enise.fr/info/latex}提供了一个已经翻译过的图片档案,可供下载……
\end{exclamation}

\subsection{DVI和PostScript}

%TODO 加线
\begin{verbatim}
latex framabook
make bibindex
latex framabook
latex framabook
dvips framabook -o framabook
\end{verbatim}

\subsection{PDF}

没什么特别的:

%TODO 加线
\begin{verbatim}
ex framabook 
make bibindex 
pdflatex framabook 
pdflatex framabook
\end{verbatim}

\subsection{春季大扫除}

\begin{tabbing}
1234567890123456789012345\= \kill
\dmh{make cleanfigs}\> \leftarrow 抹除所有eps、pdf……\\
\dmh{make cleantex}\> \leftarrow 抹除所有辅助文件\\
\dmh{make cleandocs}\> \leftarrow 抹除所有生成的文件(dvi、ps、pdf……)
\end{tabbing}
